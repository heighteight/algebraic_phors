
In this section we provide an overview on our approach to probabilistic termination via algebraic generating functions.

\subsection{Algebraic Power Series from Finitary PHORS}

The weighted relational semantics arises from linear logic, and is indeed based on a \emph{precise count} of the number of times that a program may use each of its inputs during any of its reductions. 
For example, the interpretation of an order-2 program $M:(o\to o)\to (o\to o)$ yields a generating function which of the form
$$
 \model{M}(y,x)=\sum_{n=0}^{\infty}a_n y^n x
$$ 
where the real coefficient $a_n$ indicates the probability that 
$M$ terminates through a reduction that uses its functional input $y$ exactly $n$ times (and $x$ once).
For instance, the program 
$$M=\mathrm{Fix}(\lambda hyx. x\oplus hy(yx))$$
which generates the probabilistic tree 
$$x\oplus (yx\oplus (y^2x\oplus (y^3x\oplus \dots$$ 
yields the power series 
$$ \model{M}(y,x)=\sum_{n=0}^\infty \frac{1}{2^{n+1}}y^n x=\frac{x}{2-y},$$
as the probability of terminating using $y$ exactly $n$ times is $\frac{1}{2^{n+1}}$.

Observe that we are silently assuming, for simplicity, that order 1 programs $P:o\to o$ are always \emph{linear}, that is, they may use their input precisely once. This implies that any such program is interpreted by a unique coefficient (the probability of $P$ terminating using its input once), and that the variable $x$ above may indeed only occur with exponent 1. While this is a simplification, it is not a very strong one, since, as we will see in Section ??,  order-1
PHORS are always \emph{affine}, that is, they may use their input variables \emph{at most} once.




%
%\begin{align*}
%Hfx&= H(B\circ f)x \oplus_{p_1} (f\circ f)(Bx)\\ 
%Bx&= x\oplus_{p_2} \Omega\\
%S&= HB e
%\end{align*}
%
%
%Language associated:
%$$
%\{ 
%a^{n}b^{2n}\mid n\in \mathbb{N}, n\geq 1
%\}
%$$
%
%\begin{align*}
%Hfgx&= H(A\circ f)(B\circ g)x \oplus_{p_1} (f\circ g\circ f)x\\ 
%Ax&= x\oplus_{p_2} \Omega\\
%Bx&= x\oplus_{p_3} \Omega\\
%S&= HAB e
%\end{align*}
%
%Language associated:
%$
%\{ 
%c^{n+1}a^{n}b^{n}a^n\mid n\in \mathbb{N}\}
%$

Consider now a program defined by the following equations:
\begin{equation}\label{eq:phors1}
\begin{aligned}
Hyx&=( H(A\circ y)x \oplus_{a}
H(B\circ y)x )\oplus_{a}y(yx)\\
Ax&= x\oplus_{b} \Omega\\
Bx&=x\oplus_{c} \Omega\\
S&=HIe
\end{aligned}
\end{equation}

This is indeed an example of PHORS: the upper case letters are called \emph{non-terminal} symbols, the execution of the program starts from the order 0 non-terminal $S$ by applying instances of the equations, read from left to right, as well as probabilistic choices, until the unit constant $e$ is, eventually, produced. 
Notice that $I$ and $\Omega$ stand, respectively, for the identity and diverging terms, and that $a,b,c,d$ stand for rational biases for the probabilistic choice operators.


We can canonically associate an infinite tree with the program above by considering binary function symbols $a,b,c$ with each choice operator. The word language (defined as in \cite{}) consisting of all the finite branches of this tree is then 
$$
\mathcal L_M=\{ a^{2|w|+2}ww\mid w\in \{b,c\}^*\}.
$$
Notice that this language is not context-free, as it contains an arbitrary word repeated twice.

In the relational interpretation, each order-2 non-terminal symbol $N$ yields a sequence of coefficients $N_n$ (and a corresponding generating function  $\model{N}(y,x)=\sum_nN_ny^n x$), and each order-1 non-terminal $N$ yields a unique real coefficient. 
The equations \eqref{eq:phors1} translate then naturally into a system of \emph{polynomial} equations over such coefficients. On the one hand, we immediately get $A\cdot x= bx$ and $ B\cdot x=cx$, that is, $A=b, B=c$; for the $H_n$ we can solve using $\model{H}(y,x)=\sum_n H_n y^n x$, which yields
\begin{align*}
H_2y^2 x&= 
\alpha
H_2y^2x
+ \beta y^2x,\\
H_ny^n x&= 0 \qquad (n\neq 2),
\end{align*}
where $\alpha=(a^2b^2+a(1-a)c^2)$ and $\beta=1-a$.
In other words, $H_n=0$ for all $n\neq 2$, while $z=H_2$ 
%is a root of the polynomial $\mathscr p\in \BB R[x,y][z]$
%$$
%\mathscr p(x,y)(z)=\delta xy^2 \cdot z+ (1-a)xy^2,
%$$
%which 
can be easily computed $H_2=\frac{a-1}{a^2b^2+a(1-a)c^2-1}$. 
Notice that this implies that a reduction of $H$, \emph{independently of the number of its unfoldings}, will always end up using its functional input precisely twice.

Now, observe that from the equations above it follows 
$$
\model{H}(y,x)= \alpha\cdot \model{H}(y,x)+\beta y^2x
$$
that is, the power series $\model{H}(y,x)$ is a solution of the polynomial equation
$\mathscr p(x,y,z(y,x))=0$, where
$\mathscr p(x,y)(z)=(\alpha-1)\cdot z+ \beta y^2x$, so it is algebraic.


This example illustrates one fact that is general and one that only works in a \emph{restricted} class of situations. The general fact is that, for \emph{any} PHORS, the relational interpretation produces a set of polynomial equations defining the real coefficients interpreting each non-terminal symbol; however, this system will in general be \emph{infinite}: already at order 2, as we saw, we obtain sequences of coefficients $(a_n)_{n\in \mathbb N}$, each with its own equation. 
We can thus generally translate a PHORS into a system of polynomial equations over \emph{countably many} unknowns.

In the example, though, we realized that, out of all the coefficients $H_n$, only $H_2$ may be different from zero. More generally, whenever we realize that, out of the countably many unknowns of the (relational interpretation of the) program, only \emph{finitely many} are non-zero, we obtain a finite system of polynomial equations, which implies that each such coefficient is an algebraic real number, and that, globally, the power series interpreting each non-terminal are themselves algebraic. 


How can we enforce, then, a program to translate into finitely many coefficients? The solution comes, again, from linear logic: as we saw, each coefficient represents reductions using the inputs a fixed number of times; what if we may provide a \emph{finite bound} on the number of uses that a program may make of each of its inputs?

A first natural idea is to restrict ourselves to linear, or even affine, programs. This is indeed the approach taken in \cite{}. However, as the example above suggests, one may well allows programs to use their inputs more than once, even a very large number of times, as soon as we can provide a fixed bound for this number (as the final number of coefficients will rely - even though exponentially, see ??? - on it). 
Indeed, a standard and well-studied way to impose finite bounds is via \emph{(affine) graded exponentials} \cite{} $!_nA$, where a program of type $M:\ !_nA\multimap B$ is forced to use its input \emph{at most} $n$ times. Interestingly, as we will see, this restriction continues to hold in presence of fixpoints: if $A$ is a finitely graded higher-order type and $M: A\to A$ is a program which may use its input $A$ \emph{unrestrictedly}, then its fixpoint $\mathrm{Fix}M:A$ is still ``finitary''.
%: its interpretation still has a number of unknowns, which can be traced as the points of some algebraic variety, as we'll see.


For example, the non-terminal $H$ in our example could be typed as 
$H:\ !_2(!_1o\multimap o)\multimap (!_1o\multimap o)$, since, as we saw, any reduction of $H$ uses its functional argument twice.



At the same time, observe that, while our example is non-linear, it can well be \emph{linearized}: we can design some affine PHORS that generates the same infinite tree, for example:
\begin{align*}\label{eq:phors2}
Ly_1y_2x&=( L(A\circ y_1)(A\circ y_2)x \oplus_{a}
L(B\circ y_1)(B\circ y_2)x )\oplus_{a}y_1(y_2x)\\
Ax&= x\oplus_{b} \Omega\\
Bx&=x\oplus_{c} \Omega\\
S&=HIIe
\end{align*}
Notice that the unique functional variable $y$, that was used twice, is now replaced by \emph{two} functional variables $y_1,y_2$, used once.
Linearization is a well-known procedure in linear logic which, intuitively, corresponds to unfolding a (finite) exponential $!_nA$ as an $n$-fold tensor $A\otimes\dots \otimes A$. 
As a consequence, the word languages that are obtained via PHORS typed via graded exponential coincide with those obtained via the affine PHORS of \cite{}. 
However, notice that the linearized PHORS may well have a number of variables that is exponential with respect to those of the original, non-linear one. 


\subsection{Deciding AST and PAST for Finitary PHORS}

We now show how \emph{almost sure termination} (AST) and \emph{positive almost sure termination} (PAST) can be decided for a PHORS typable via graded types.
Let us recall that AST and PAST are, respectively, the problems to know whether a program terminates with probability 1, and whether its expected number of steps before termination is finite. 

Given a PHORS typable via graded types, we have seen that, for each non-terminal symbol $N$, the associated power series $\model{N}$ contains finitely many non-zero terms $N_ky^k$. As all such (finitely many) coefficients are given in terms of the other ones via polynomial equations, the set of their possible values forms an algebraic variety.
As a consequence, the interpretation of the source non-terminal $S:o$, which is obtained by combining other non-terminals as well as a unique constant $e:o$, can be expressed as a polynomial combination of the coefficients. 
At this point, it is not difficult to design a formula in the decidable \emph{first order theory of the reals} \cite{} that expresses that such a polynomial combination of algebraic reals is equal to 1, and thus to decide AST.

We can also obtain a first-order formula expressing PAST as follows.
In the relational semantics a choice $M\oplus_a N$ with bias $a$ is interpreted as a convex sum $a\model M+(1-a)\model N$; in order to count the number of reduction steps we can multiply each such convex sum by a fresh parameter $w$, yielding $w(a\model M+(1-a)\model N)$. 
Logically, this corresponds to adding a fresh linear variable $w:o\multimap o$ and replacing each choice $M\oplus_a N$ by $w(M\oplus_a N)$. 
Notice then that, with this new variable in the game, the type of the source non-terminal $S$
is now $(o\multimap o)\multimap o$, that is, $S$ 
 is now interpreted as a formal power series 
$\model{S}(w)=\sum_{n=0}^\infty S_nw^n$, where $S_n$ is the probability of termination using the fresh parameter $w$ exactly $n$ times, that is, making exactly $n$ probabilistic choices. We will show in Section \ref{} that the power series $\model{S}(w)$ remains algebraic even after this \emph{parameterization} via $w$.

As a paradigmatic example, consider the order-1 PHORS
\begin{align*}
Fx&= F(Fx)\oplus_{\frac{1}{2}} x & S&=Fe,
\end{align*}
corresponding to a simple random walk. Parameterizing $ F(Fx)\oplus_{\frac{1}{2}} x$ as $w( F(Fx)\oplus_{\frac{1}{2}} x)$ yields the algebraic power series $C(w)$ given by 
$$
C(w)= \frac{1}{2}\cdot {(wC^2(w)+w)},
$$
whose solution gives precisely the generating function of the {Catalan numbers}
$C(w)=\sum_{n=0}^\infty C_nw^n = \frac{1-\sqrt{1-4w}}{2w}$ (it is indeed well-known that $C_n$ counts the number of $n$-step paths for the simple random walk)


Now, observe that the expected number of steps (i.e.~of probabilistic choices) is precisely given by $\model{S}'(1)$, where $\model{S}'$ is the \emph{derivative} of $\model{S}$:
$$
\model{S}'(w)=\sum_{n=1}^\infty n\cdot S_{n}w^{n-1}.
$$
The derivative $a'(x)$ of an algebraic power series $a(x)$ is still an algebraic power series,
and there is a well-known method to express $a'(x)$ as a polynomial in $a(x)$ with rational function coefficients \cite{}. All this allows us to express PAST (i.e.~$\model{S}'(1)<\infty$) via some first-order formula over the real numbers, and thus to decide it.




\subsection{Towards Non-Finitary PHORS via Parameterization}

As we saw, a PHORS typable via graded types may use its functional inputs only a finite number of times, independently of the number of its unfoldings. 
On the other hand, in Section ?? we explore a \emph{parameterization method} that can be used to show the algebraicity also of PHORS that may use their functional inputs an \emph{arbitrary} number of times.

Indeed, an instance of this method underlies our previous discussion of PAST, which relied on showing that the generating function $\model S(w)=\sum_{n=0}^\infty S_nw^n$ for the number of steps to termination of an algebraic PHORS is itself an algebraic power series, in which the variable $w$ may appear arbitrarily many times.

The same idea can then be used to show the algebraicity also for PHORS which are not finitary. For instance, the following order-2 PHORS
\begin{align*}
Lyx&=Ly(Lyx)\oplus_a y(yx),\\
Bx&= x\oplus\Omega,\\
S&= LBe,
\end{align*}
can be seen as a functional and non-linear variant of the simple random walk, in which the function variable $y$ may be used an arbitrary (even) number of times. 

What makes this example, as well as the previous ones, work, is the remark that the non-terminal $L$ uses the functional variable $y$ as a parameter, that is, we are never asked to compute values of $L$ over, say, $y^2$ or $2y+1$. 
Formally, this can be captured by looking at this PHORS as an algebraic formal power series 
with coefficients taken \emph{themselves} from a semiring of formal power series. In the relational semantics, this corresponds to lifting the underlying continuous semiring of weights from the (extended) positive reals $\BB R_{\geq 0}^\infty$ to the semiring of formal power series
$\fps{\BB R_{\geq 0}^\infty}{y}$ in the variable $y$, as explained for instance in \cite{}.
At the same time, we show that all such cases can be captured by extending our graded type system with a new rule that allows to introduce \emph{infinite} grades $!_\infty A$ in a controlled way. 



%
%\begin{align*}
%Hfx&= (p_1H(p_3fx)+p_2H(p_4fx) + (1-p_1-p_2)f^2x\\
%&= (p_1p_3+p_1p_4)Hfx+(1-p_1-p_2)f^2x
%\end{align*}
%
%So $y=H(f,x)$ is solution of the polynomial equation $\mathscr p(f,x,y)=0$, where
%$$
%\mathscr p(f,x,y)= (p_1p_3+p_1p_4-1)y+(1-p_1-p_2)f^2x.
%$$
%
%The solution sequence $(H_n)_{n\in\mathbb N}$ is the following: 
%\begin{align*}
%H_n&=0 \quad (n\neq 2)
%\\
%H_2&=\sum_{k=0}^\infty\sum_{k=m+n}(p_1p_3)^m(p_2p_4)^n\\
%&= \sum_{m,n}(p_1p_3)^m(p_2p_4)^n\\
% &=\sum_{m=0}^\infty(p_1p_3)^m\cdot \sum_{n=0}^\infty(p_2p_4)^n\\
% &=\frac{1}{1-p_1p_3}\cdot \frac{1}{1-p_2p_4}
%\end{align*}
%which yields the probability of termination $\sum_nH_n=\frac{1}{1-p_1p_3}\cdot \frac{1}{1-p_2p_4}$.
%\begin{itemize}
%
%\item fundamental idea: when studying the relational interpretation of a probabilistic program, we can look at it as a formal power series, whose coefficients can be seen as \emph{countably many} unknowns. These are infinite since one has to consider trajectories that may use inputs an arbitrary number of times (as in the example above)
%
%%\item fixpoints corresponds then to imposing a system of \emph{polynomial equations} over such unknowns; yet, solving a system of infinitely many polynomial equations is far beyond what can be hoped!
%
%\item yet, what it we impose a \emph{bound} on the number of uses that the program can do of each of its inputs?
%This is a well-known approach that uses \emph{(affine) graded types} $!_nA\multimap B$, meaning ``produce $B$ using an input $A$ at most $n$ times'' 
%
%\item under such constraints, the number of unknowns to find becomes \emph{finite} (although possibly very large). In a similar way, when computing the semantics of $\mathrm{fix}M:A$, where $A$ is finitely graded but $M:A\To A$ may use its input $A$ unboundedly, we show that we obtain a system of \emph{finitely} many polynomial equations, whose solution set forms then an \emph{algebraic variety}.
%
%\item Using standard algebraic reasoning, the interpretation of this fixpoint is thus shown to be a power series $F$ that is solution of a \emph{unique} polynomial equation $p(x,F(x))=0$, yielding the decidability of AST. 
%
%
%\end{itemize}
