% !TEX root = main.tex


In this section we introduce the class $\Gfin$, extending the PAHORS of \cite{DBLP:conf/lics/LiMO22}, of PHORS with algebraic generating function and decidable AST and PAST.
% We will show that both the AST and PAST problems are decidable for the PHORS typable in this system.

%
%$\Gfin$ are an extension of the affine PHORS of (Li et al.) with \emph{finitely graded} exponentials $!_kA$ ($k\in \mathbb N$), allowing for a program to use a given input \emph{no more} than some fixed number of times.
%% In the next section we will show how to extend this system with \emph{infinite} exponentials $!_\infty A$, enabling programs to use their inputs even an \emph{unbounded} number of times.

\subsection{Syntax of $\Gfin$ }



We use types defined by the grammar below:
\begin{align*}
\varphi,\psi&:= \1^n\mid \  !_k\varphi\multimap \varphi
%\\
%\xi&:=\ !_k \varphi
 \qquad (k\in\mathbb N).
\end{align*}
We let $\tau$ be an abbreviation for the affine function type $!_1o\multimap o$. 
We define an order relation $\varphi\sqsubseteq \psi$ between types by induction by
\begin{align*}
\infer{\1^n\sqsubseteq\1^n}{}
% \quad
% \infer{\varphi\&\psi\sqsubseteq\varphi'\&\psi'}{\varphi\sqsubseteq \varphi', \psi\sqsubseteq\psi'}
\quad \infer{\xi\multimap\varphi\sqsubseteq\xi'\multimap\varphi'}{\xi\sqsubseteq \xi', \varphi\sqsubseteq\varphi'}
\quad
\infer{!_k\varphi\sqsubseteq !_h\psi}{k\leq h \quad\varphi\sqsubseteq\psi}
\end{align*}

The typing rules will use two kinds of contexts (as in \cite{DBLP:conf/lics/LiMO22}): 
\begin{varitemize}
\item non-linear contexts, noted $\nonterm,\nonterm'$ will be used for non-terminal symbols; these are finite sets of type bindings of the form $L:\varphi$; intuitively, we put no restrictions on the number of times a non-terminal is used;


\item graded contexts, noted $\Delta,\Delta'$ will be used for the arguments to be passed to the non-terminales; they are finite sets of type bindings of the form $x:_k \varphi$. Intuitively, the binding $x :_k \varphi$ means that we can use $x$ with type $\varphi$ \emph{at most} $k$ times.
\end{varitemize}

We define by induction the operations $\Delta+\Delta',k\Delta$ on graded contexts as follows:
{\small
\begin{align*}
\Delta +\emptyset &= \Delta, \\
(\Delta,x:_{k}\varphi) + (\Delta',x:_{h}\varphi) &= (\Delta+\Delta'),x:_{k+h}\varphi;\\
%\Delta \vee\emptyset &= \Delta, \\
%(\Delta,x:_{k}\varphi) \vee (\Delta',x:_{h}\varphi) &= (\Delta+\Delta'),x:_{\max\{k,h\}}\varphi;\\
k\emptyset&=\emptyset,\\
k(\Delta,x:_{h}\varphi)  &= k\Delta,x:_{kh}\varphi,
\end{align*}
}
where, as in e.g.~\cite{10.1145/3009837.3009890}, we are assuming that $\Delta$ and $\Delta'$ agree on each variable (i.e.~they contain the same type bindings but may disagree on the corresponding coefficients). 

A type judgement is an expression of the form $\nonterm\mid\Delta\vdash t:\varphi$.
% or
%$\emptyset;\emptyset\vdash^{\mathrm{fix}} t:\varphi$. The second form of type judgements are called \emph{PHORS-typings}. Notice that in a PHORS-typing we require the context to be empty.
The typing rules are illustrated in Fig.~\ref{fig:typerules}.
\begin{definition}[$\Gfin$]
A \emph{finitely graded PHORS} (noted $\Gfin$) is a triple $\gphors=(\nonterm, \C R, S)$, where $\nonterm$ is a finite set of typed non-terminals, $S\in\nonterm$ is the {start symbol} such that $\nonterm(S)=\1$, and $\C R$ is a function that associates each $L\in\nonterm$ with a derivation $\nonterm\mid\emptyset\vdash \lambda x_1.\dots. x_n.t:\varphi$
%(t_L \oplus_p t_R):\varphi$ 
such that $\varphi\sqsubseteq \nonterm(L)$, $t$ contains no $\lambda$-abstraction and $\nonterm\mid x_1,\dots, x_k\vdash t:\1$.
%t_L \oplus t_R:\1$. 
\end{definition} 

Fom any $\Gfin$ we can canonically extract the underlying (unrestricted) PHORS.
Observe first that the rules (and indeed the derivations) of $\Gfin$ can be seen as \emph{quantitative refinements} of those of $\PLY$. First, every type $\varphi$ is (uniquely) a \emph{refinement} of some simple type $T$: the refinement relation $\varphi\tri T$ is defined by
$\1^n\tri\1^n$ and 
by $\varphi\tri T , \psi\tri U\Rightarrow !_k\varphi\multimap\psi\tri T\to U$.
%
%\[
%\infer{\1\tri\1}{}\qquad
%\infer{\varphi\&\psi\tri T\& U }{\varphi\tri T & \psi\tri U}\qquad
%\infer{!_k\varphi\multimap\psi\tri T\to U }{\varphi\tri T & \psi\tri U}
%\]
Notice that $\varphi\sqsubseteq \psi$ implies $\varphi\tri T\Leftrightarrow \psi\tri T$. 
Next, for every derivation $\pi:\nonterm\mid\Delta \vdash t:\varphi$, 
by replacing each $\psi$ by $U$, where $\psi\tri U$, we obtain, by induction, a simply typed derivation $\Gamma\vdash t:T$, where $\nonterm\mid\Delta\tri \Gamma$ and $\varphi\tri T$. 




%Differently from (Li et al.), we require terms to be always of the form $t_L\oplus_p t_R$. This poses no syntactic restriction, since a term $t$ can always be turned into $t\oplus_p t$, but will ensure that the corresponding FAS is proper.

Moreover, as affine implication is expressed by $!_1\varphi\multimap \psi$, 
%and every derivation $\nonterm\mid x_1,\dots, x_k\vdash t :\1$ yields a semantically equivalent derivation  $\nonterm\mid x_1,\dots, x_k\vdash t \oplus_1 t :\1$, 
it is immediate that all PAHORS are $\Gfin$, so all our results translate automatically to them. In particular, as the PHORS from Example \ref{ex:phors1} is affine, it is also a $\Gfin$.




%
%\begin{proposition}
%Every PAHORS is a $\Gfin$.
%\end{proposition}

%
%While most rules are standard graded variants of the rules of the simply typed $\lambda$-calculus (see Gaboardi, Mazza etc.), fixpoints are introduced by a
% \emph{finitary fixpoint rule} $Y_{\mathrm{fin}}$, which produces a PHORS-typings. As we'll see, this rule ensures that the finiteness condition of Definition \ref{def:finitary} holds for the corresponding fps.
%Notice that the graded context in $Y_{\mathrm{fin}}$ is required to be empty.
%



\begin{figure}
\fbox{
\begin{minipage}{.43\textwidth}
\resizebox{.99\textwidth}{!}{
\begin{minipage}{1.1\textwidth}
\begin{align*}
	\begin{array}{ccc}
		\infer[ax1]{\nonterm\mid \Delta\vdash x:\varphi }{x:_p\varphi\in \Delta,\quad 1 \leq p} & &
		\infer[ax2]{\nonterm\mid\Delta \vdash L:\varphi }{ L:\varphi  \in \nonterm}\\[7pt]
%		\infer[!]{\Gamma;k\Delta \vdash t: !_k \varphi}{\Gamma;\Delta \vdash t: \varphi \quad k\in \mathbb N}
%		 & \qquad &
%		 \infer[\oplus]{\Gamma, \Delta \vdash t \oplus t' : \1 }{\Gamma; \Delta \vdash t:\1 \quad \Gamma; \Delta \vdash t':\1} \\[5pt]
%		 
		\infer[\lambda]{\nonterm\mid \Delta \vdash \lambda x. t : !_k \varphi \multimap \psi }{\nonterm \mid \Delta, x:_k \varphi \vdash t : \psi}
		&  &
		\infer[@]{\nonterm\mid \Delta + k\Delta' \vdash tu : \psi}
		{\nonterm\mid\Delta \vdash t :!_k\varphi \multimap \tau \qquad \nonterm\mid \Delta' \vdash u :\varphi }\\[9pt]
		\infer[\langle \rangle]{\nonterm\mid \Delta \vdash \langle t_1, \dots,t_n \rangle : \1^n}{\nonterm\mid \Delta \vdash t_i : \1 &  i=1,\dots, n}&  &
		\infer[\pi_i]{\nonterm\mid \Delta \vdash \pi_i t: 1}{\nonterm\mid \Delta \vdash t: 1^n &  i=1,\dots, n}
%		\\[5pt]
%		\infer[\lambda \infty]{\Gamma; \Delta \vdash t: \tau \to \sigma}{\Gamma, L: \tau; \Delta \vdash t: \sigma} 
	\end{array}
	\end{align*}
	\begin{align*}
		 \infer[\oplus]{\nonterm\mid \Delta \vdash t \oplus t' : \1 }{\nonterm\mid \Delta \vdash t:\1 \quad \nonterm\mid \Delta \vdash t':\1} 
	\end{align*}
%		\begin{align*}
%			\infer[ Y_{\mathrm{fin}}]{\nonterm\mid\emptyset \vdash^{\mathrm{fix}}  Y \lambda \langle L_1,\dots, L_n\rangle.t:\psi }{\nonterm,\langle L_1,\dots, L_n\rangle:\varphi\mid \emptyset \vdash t :  \psi \quad \psi \sqsubseteq \varphi  }
%	\end{align*}
	\end{minipage}
	}
\end{minipage}
}
\caption{Typing rules of $\Gfin$.}
\label{fig:typerules}
\end{figure}


\begin{example}\label{ex:nonlin}
The $\gphors$ from \eqref{eq:phors1} in Section 2 is not affine, but is a $\Gfin$: 
$A,B$ are linear and $Hyx$ may use the functional variable $y$ at most two times, and
we can type  $A,B:\tau, H:!_2\tau\multimap \tau\mid \emptyset \vdash S:o$.

%, with $\nonterm(A)=\nonterm(B)=\tau$ and 
%$\nonterm. 
%
% so they can be given type $A,B:\tau$; moreover, 
% since  
% we can type
\end{example}


\begin{example}[a non-algebraic PHORS]\label{ex:nonalg}
The PHORS $\gphors$ below:
\begin{align*}
Lfx &= L(f\circ f) x\oplus_{p} fx \\
S&= L\ \mathrm{id}\  e
\end{align*}
is \emph{not} a $\Gfin$: if we try to type $t_L=L(y\circ y) x\oplus_{p} fx$ we obtain
$L:\varphi \mid \emptyset\vdash \lambda y.\lambda x.t_L: \psi$, where $\psi=\ !_2\tau\multimap \tau\not\sqsubseteq \ !_1\tau\multimap\tau=\varphi$. 
Indeed, the generating function $s^L(y_{\N},x)\in \fps{\Rsemiring}{\N+1}$ of $L$ satisfies 
$s^L(y_\N,x)=\frac{1}{2}(
y_1x+s^L( y^2_\N,x)
)$, which yields the solution $s^L(y_\N,x)=\sum_i\frac{1}{2^{i+1}}y^{2^i}x$, that is not algebraic (cf. Remark~\ref{rem:hadamard}). 
\end{example}



%
% there exists a derivation  the following is easily proved by induction:
%\begin{proposition}[$\Gfin$ refines $\PLY$]
%For every derivation $\pi:\nonterm\mid\Delta \vdash_{\Gfin}t:\varphi$, given $\nonterm\mid\Delta\tri \Gamma$ and $\varphi\tri T$, there exists a derivation $\pi^*:\Gamma\vdash_{\lambda_{\oplus} }t:T$.
%\end{proposition}


While the $\Gfin$ are not linear, they can be \emph{linearized}:
we can turn them into PAHORS with same semantics and branch language. Notice, however, that the resulting affine PHORS might have size exponential in the original one.

%
%
%
%, the set of corresponding branch languages do indeed coincide: this means that for any $\Gfin$ it is possible to find a PAHORS defining the same branch language.

Define the \emph{size} of a $\gphors$ as the number of its non-terminals times the maximum size of any of its terms, i.e.~
$\|\gphors\|=|\nonterm|\times\max\{|t_L|\mid L\in\nonterm\}$, and let $\partial\gphors$ be the maximum grade occurring in $\gphors$.

\begin{theorem}[Linearization]
For every $\Gfin$ $\gphors=(\nonterm, \C R, S)$ there exists a PAHORS $\AFF\gphors=(\nonterm, \C R',S)$ with size $\|\AFF\gphors\|\leq \partial\gphors{\|\gphors\|}$ and such that 
$a_{\gphors}=a_{\AFF{\gphors}}$ and 
$\C L(\gphors)=\C L(\AFF\gphors)$.

\end{theorem}
\begin{proof}[Proof sketch]
Define $\AFF{\varphi}$ as $
\AFF{(o^n)}=o^n$ and 
$\AFF{(!_k\varphi\multimap \psi)}={\AFF{\varphi}\multimap\dots\multimap\AFF{\varphi}}\multimap\AFF\psi$, with $\AFF{\varphi}$ repeated $k$ times. 
Then, by induction, any derivation of $\nonterm\mid \Delta\vdash t:\varphi$ yields a derivation of $\AFF{\nonterm}\mid\Delta'\vdash t:\AFF \varphi$ in the affine system of (Ong et al.), where $|\Delta'|\leq \partial\gphors|\Delta|$: an order-$n$ $\Gfin$ yields then an order-$n$ PAHORS in which every equation $Lx_1\dots x_n$ translates into a new equation $Lz_1^1\dots z_{n}^{k_n}$, where, if $\nonterm(L)=!_{k_1}\varphi_1\multimap\dots \multimap !_{k_i}\varphi_i\multimap\dots\multimap o^n$, the variable $x_i$ is replaced by $k_i$ variables $z_i^1,\dots, z_{i}^{k_i}$.
\end{proof}


\begin{example}
The linearization of the $\Gfin$ from Example \ref{ex:nonlin} yields the PAHORS below:
{\small
\begin{align*}\label{eq:phors2}
Ly_1y_2x&=( L(A\circ y_1)(A\circ y_2)x \oplus_{a}
L(B\circ y_1)(B\circ y_2)x )\oplus_{a}y_1(y_2x)\\
Ax&= x\oplus_{b} \Omega\\
Bx&=x\oplus_{c} \Omega\\
S&=HIIe
\end{align*}
}
Notice that the unique functional variable $y$, that was used twice, is now replaced by \emph{two} functional variables $y_1,y_2$, used once.
\end{example}

%In a $\Gfin$, the degree $\partial \gphors$ can be exponential in $\|\gphors\|$, so that $\AFF\gphors$ has size exponential in $\|\gphors\|$:

\begin{example}
$\|\AFF\gphors\|$ may be exponential in $\|\gphors\|$:
take the order-2 $\Gfin$ $\gphors$ given by
{\small
\begin{align*}
F_1fx&= f(fx)\\
F_2f x&= F_1(f\circ f) x \oplus_{p} x\\
%F_2f x&= F_3(f\circ f) x \oplus_{p} x\\
&\vdots\\
F_{n}fx&= F_{n-1}(f\circ f)x\oplus_{p}x\\
S&=F_n \ \mathrm{id} \  e
\end{align*}
}
As $\nonterm(F_n)=\ !_{2^{n}}\tau\multimap\tau$, 
$\|\AFF{\gphors}\| \leq \partial\gphors\|\gphors\|=
2^{\C O(\|\gphors\|)}\|\gphors\|$.
\end{example}



\subsection{From $\Gfin$ to FAS}

We will now show how, for any $\Gfin$ $\gphors$, the interpretation of the underlying PHORS yields a finitary FAS, whose minimal solution is thus an algebraic family.

 For any set $\Sigma$  and $k\in \N$, let $!_k\Sigma\subseteq !\Sigma$ be the set of multisets $\mu$ of maximal multeplicity $k$ (i.e $\max_{x \in \Sigma} \mu(x)\leq k)$. $!_k\Sigma$ corresponds to the monomials over $x_\Sigma$ of degree at most $k$ in each variable. 
 
% \begin{definition}
%Given a $X$-family of fps $s_X\in \fps{\Rsemiring}{\Sigma}^X$, $\Sigma'\subset \Sigma$ and $X'\subset X$, let $s_X\big\vert_{\Sigma'}^{X'} \in \fps{\Rsemiring}{\Sigma}^X$ be the $X$-family $t_X$ defined by $t_i= s_i\vert_{\Sigma'}$, for all $i\in X'$, and $t_i=0$ otherwise.
%
%A $X$-family of fps $s_X\in \fps{\Rsemiring}{X}^X$ is \emph{finitary} if there exists $Y\subset X$ such that $(s_X)\big\vert_Y^Y$ is a finite family of polynomials.
% \end{definition}

 
   Define, for every type $\varphi$, $\model{\varphi}$ as $\model{o^n}=1$ and $\model{!_k\varphi\multimap \psi}=!_k\model{\varphi}\times\model{\psi}$. We extend this to contexts by $\model{\emptyset}=0$ and $\model{\nonterm,x:\varphi}=\model{\nonterm}+\model{\varphi}$ (irrespectively of grades). Notice that $\varphi\tri T$ implies $\model{\varphi}\subseteq\model{T}$, but the variable sets $\model{\varphi}$ are always finite. 
      
   Our general idea is the following: given a $\Gfin$ $\gphors=(\nonterm, \C R, S)$ and supposing $\nonterm\tri T$, from the derivations $\C R(L)$ of $\nonterm\mid\emptyset \vdash t_L:\nonterm(L)$, we deduce a corresponding simply typed derivation of $T\vdash t_{\gphors}:T$ yielding a family
   $\model{t}^{\Rsemiring}\in \fps{\Rsemiring}{\model{T}}^{\model{T}}$.  
  At the same time, we will show that the finite set $\model{\nonterm}\subset\model{T}$ yields a stable and polynomial pair, and we will conclude that $\model{t_{\gphors}}$ is finitary, hence, by Corollary \ref{cor:propertoalg}, algebraic.    
   
   The fundamental ingredient is the following lemma:
   
  \begin{lemma}[stability lemma]\label{lemma:stability}
  Let $\Rsemiring=\fps{\Rinf}{\Sigma}$.
For every derivation $\nonterm\mid\Delta\vdash t:\varphi$ refining a simply typed derivation
$\Gamma\vdash t: T$, the pair $(\model{\nonterm}+\model{\Delta}, \model{\varphi})$ is stable and polynomial for $\model{t}^{\Rsemiring}_{\model\Gamma}$.
  
  
  
  \end{lemma}
  \begin{proof}[Proof sketch]
  Let a \emph{finitary} variable be any point in the interpretation of ${\nonterm},{\Delta},{\varphi}$, and a \emph{non-finitary} variable be any other point in the interpretation of $\Gamma$ and $T$.
   One argues, by induction on the derivation, for the following two claims:
   \begin{varenumerate} 
   \item for all non-finitary variables $\sigma$, $\left(\model{t}^\Rsemiring\right)_\sigma\big\vert_{\model{\nonterm}+\model{\Delta}}=0$;
   \item for all finitary variables $\sigma$, 
   $\left(\model{t}^\Rsemiring\right)_\sigma\big\vert_{\model{\nonterm}+\model{\Delta}}$ has 
   finite degree in each variable $x_{L:\nonterm(L)}$ and 
   degree $\leq k$ in each variable $x_{f:_k\psi}$.
   \end{varenumerate}
%   In the end, condition 1.~implies stability and condition 2.~implies polynomiality.
   \end{proof}
%   Given $\model{\varphi}\tri T$ and $\model{\psi}\tri U$, 
%  every fps $s\in \fps{\Rsemiring}{\model{T}}^{\model U}$ 
%can be restricted to a finitary family of fps
%$s\big\vert^{\model\psi}_{\model\phi}\in \fps{\Rsemiring}{\model{\phi}}^{\model \psi}$ in finitely many variables.
%We will use this idea to interpret the $\Gfin$.
%
%Now, for a term $t:\varphi$ typable in $\Gfin$, where $\varphi\tri T$, we will define its \emph{finitary interpretation} $\model{t:\varphi\tri T}^{\Rsemiring}$, intuitively, as the \emph{restriction} to its finitary environment $\model\nonterm$ and type
% $\model{\varphi}$, of the interpretation $\model{t:T}^{\Rsemiring}$ of the underlying simply typed term.
%%
%%Recall that the interpretation $\model{\Gamma\vdash t:T}^{\Rsemiring}$ of a $\PLY$-derivation in $\Qrelkleisli{\Rsemiring}$ is defined by induction on the rules, associating the axiom rule $\Gamma, x:T\vdash x:T$ with the projection $\Qrelkleisli{\Rsemiring}(\model{\Gamma}+\model{T},\model{T})$, the rule introduction $\oplus$ with weighted sums, and all other rules with the corresponding operation on $\Qrelkleisli{\Rsemiring}$ given by cartesian closure.
%More precisely, given a $\Gfin$-derivation $\pi:\nonterm\mid\Delta\vdash t:\varphi$, given $\nonterm\tri\Gamma_{\nonterm}, \Delta\tri\Gamma_{\Delta}$ and $\varphi\tri T$, we define 
%\[
%\model{t:\varphi\tri T}^{\Rsemiring}\in\Qrelkleisli{\Rsemiring}(\model{\Gamma_{\nonterm}}+\model{\Gamma_{\Delta}},\model{T})
%\]  
% by induction as follows:
%\begin{varitemize}
%
%\item if $\pi$ is an instance of $ax1$ or $ax2$, 
%$\model{t:\varphi\tri T}^{\Rsemiring}:=\left(
%\model{t:T}^{\Rsemiring}\right)\Big\vert_{\model{\nonterm}+\model{\Delta}}^{\model{\varphi}}
%$, where, given a $X$-family of fps $s_X\in \fps{\Rsemiring}{Y}^X$, $X'\subset X$ and $Y'\subset Y$, $(s_X)\vert_{Y'}^{X'}=(s_{X'})\vert_{Y'}\in \fps{\Rsemiring}{Y'}^{X'}$ is the $X'$-family of restrictions of $s_{X'}$ to the variables in $Y'$;
%
%\item all other rules are interpreted as the corresponding simply-typed rules.
%
%\end{varitemize}
%
%
%
%The following result ensures that the interpretation of a $\Gfin$-derivation coincides with the restriction of the corresponding simply typed derivation to its finite domain and codomain:
%
%
%\begin{proposition}\label{prop:finiok}
%Given $\nonterm\mid\Delta\vdash t:\varphi$, with $\varphi\tri T$,  
%\[
%\model{t:\varphi\triangleleft T}^{\Rsemiring}=\left(\model{t:T}^{\Rsemiring}\right)\Big\vert^{\model{\varphi}}_{\model{\nonterm}+\model{\Delta}}.
%\]
%\end{proposition}
%
%For every $\Gfin$ $\gphors=(\nonterm, \C R,S)$, we can define the interpretation of each non-terminal $L_i$ as we did for PHORS, i.e.~ 
%$
%\model{L_i}^{\Rsemiring}:=  \model{\pi_i(Yt_{\gphors})}=\pi_i\left(\fix \model{t_{\gphors}}\right)\in \Rsemiring^{\model{\nonterm(L_i)}}.
%$ 
%
%The following is at this point an immediate consequence of the remark that the type $\1$ is the only refinement of itself:
%% the finitary interpretations $\model{t:\1\tri \1}^{\Rsemiring}$ correctly capture $\mathbb P(t\downarrow)$ and $\mathbb E(t\downarrow)$.
%
%\begin{corollary}
%For every $\Gfin$ $\gphors=(\nonterm, \C R,S)$, 
%\[
%\model{S}^{\Rinf}=\mathbb P(\gphors\downarrow),
%\qquad
%\left(\model{S}^{\fps{\Rinf}{z}}\right)'(1)=\mathbb E(\gphors\downarrow).\]
%\end{corollary}
%
%
%
%
An immediate consequence of Lemma \ref{lemma:stability} is the following:
%
%At this point, observe that a consequence of Proposition \ref{prop:finiok} and of the finiteness of the types of $\Gfin$ is that the fps $\model{t:\varphi\tri T}^{\Rsemiring}$ must have a finite support. This leads to the follows:
%


\begin{proposition}
For every $\Gfin$ $\gphors=(\nonterm,\C R,S)$, with $\nonterm\tri T$, the fps
$\model{t_{\gphors}}^{\Rsemiring}\in \fps{\Rsemiring}{\model{T}}^{\model{T}}$ is finitary (via $\model\nonterm$).
% derivations of $\nonterm\mid\emptyset\vdash t_L:\varphi$, with $\varphi\sqsubseteq \nonterm{L}\tri T$, the fps
%$\model{M:\psi\tri T}^{\Rsemiring}\in \fps{\Rsemiring}{\model{T}+\model{\Gamma}}^{\model{T}}$ is finitary.
\end{proposition}

Thanks to Proposition \ref{prop:fintoalg} and the observation that all coefficients of $\model{t_\gphors}^{\Rinf}$ are rational, we obtain then:
\begin{theorem}
	\label{Gfin-algebraic-theorem}
	For all $\Gfin$ $\gphors=(\nonterm, \C R, S)$, $(a_L)_{L \in \nonterm}$ is a $\Q^{+}$ fixpoint algebraic family with parameter $z$
\end{theorem}
%Now observe that the term $t_\gphors$ is in the form $\langle t_{1, L} \oplus_{p_n}  t_{1, R} \dots  t_{n, L} \oplus_{p_n} t_{n_R} \rangle$. We thus have that every fps $s(x_{\model{\nonterm},} z)$ belonging to $\model{t_\gphors}^{\fpp{\Rinf}{z}}$ is divisible by $z$. In particular $s(0_{\model{\nonterm}} 0)=0$ and for every variable $x \in x_{\model{\nonterm}}$ the monomial $x$ does not appear in $s$: hence, $\model{t_\gphors}^{\fpp{\Rinf}{z}}$ is a proper algebraic family. Then, by \ref{cor:propertoalg}, each $\model{L}^{\fps{\Rinf}{z}}$ is an algebraic power series:
\begin{corollary}[$\Gfin$ are algebraic]
	\label{Gfin-algebraic-corollary}
	 For all $\Gfin$ $\gphors=(\nonterm, \C R, S)$,
	\begin{varenumerate}
\item for each $L \in \nonterm$, $a_L(z)$ is an algebraic power series.  
\item $\mathbb P(\gphors\downarrow)$ is a $\mathbb Q^{+}$ algebraic number and 
 $\sum_i\mathbb P(\gphors\downarrow_i)z^i$ is a ${\mathbb Q^{+}}$-algebraic power series.
\end{varenumerate}
\end{corollary}
%\begin{proof}
%	For (1), notice that $\model{S}^{\Rinf}=\model{S}^{\fps{\Rinf}{z}}$
%\end{proof}
As a consequence of Remark $\ref{remark:semilinear}$ we also obtain:
\begin{corollary}
	For all $\Gfin$ $\gphors=(\nonterm, \C R, S)$, the set $\{n \mid \:  S \text{ terminates in } n \text{ steps}  \}$
	is semilinear
\end{corollary}
\begin{remark}
	By interpreting  $t_L \oplus t_R$ as $x \model{t_L} + \overline x \model{t_L}$ (cf.~Remark \ref{rem:tropical}), the previous corollary could be strengthened to say that the set
	$\{(n,m) \mid \:  S \text{ terminates with } n \text{ left steps and } m \text{ rigth steps}\}$ is also semilinear.
	This could also been deduced by the fact that the branch language of a linear HORS is \emph{multiple context-free} \cite{DBLP:conf/mfcs/ClairambaultM19}, and languages of this class are known to be semilinear \cite{DBLP:conf/acl/Vijay-ShankerWJ87}, but it is still worth noticing that we reproved this fact via generating functions.
\end{remark}
%\begin{example}
%Consider again $\gphors$ from Example \ref{ex:nonlin}.
%
%
%
%
%\end{example}
%
%\begin{remark}
%
%We could have also defined the interpretation of $\Gfin$ via the well-known fact that the family of functors $(!_k)_{k\in \N}:\Qrel{\Rsemiring}\to\Qrel{\Rsemiring}$ defines a \emph{graded linear exponential comonad} (cite Katsumata etc.), yielding a truly finite semantics in which $\model{!_k\varphi}=!_k\model{\varphi}$. 
%%
%%. This would interpret a type $\varphi$, with $\varphi\tri T$ as a suitable finite subset of $\model{T}$. In our semantics we still have $\model{\varphi\tri T}=\model{T}$, although 
%However, we chose to keep the infinitary semantics $\model{\varphi\tri T}=\model{T}$ as we wish to look at $\Gfin$ as a tool to extract information the power series arising from (a subset of the terms of) $\lambda_{\oplus}Y$. 
%
%\end{remark}

We conclude by deducing the decidability of AST and PAST:
\begin{theorem}
	AST and PAST are decidable for any $\Gfin$.	
\end{theorem}
\begin{proof}[Proof sketch]
The  proof of Theorem \ref{Gfin-algebraic-theorem} is constructive: it gives a way to build effectively a FAS $(\vec x = p_i(\vec x))_{1 \leq i \leq n}$ that has $(a_L(z))_{L \in \nonterm}$ as its (minimal) solution.
% by computing, for each $L \in \nonterm$, the interpretation $\model{\mathcal{R}(L)}^{\Rinf}$. 
We can then check if $\mathbb P(\gphors\downarrow) = a_\gphors(1)=1$ via the first order Existential Theory of the Reals $\Etor$ \cite{DBLP:journals/jacm/EtessamiY09}: first, we create a first order formula $\phi( \vec x)$ expressing the fact that $\vec x$ satisfies $\vec x = p_i(\vec x)$, then we can express the fact that $x_1=1$ by the formula $\exists \vec{x}(x_1=1 \land \phi(\vec x) \land \forall \vec y ((\vec y < \vec x) \implies \neg \phi(\vec y)  ))$.

As for PAST, remember that PAST implies AST. Then, given a $\gphors=(\nonterm, \C R, S) \in \Gfin$, we can first test if it is AST: if it is not, then it is not PAST. If it AST, notice that, by Corollary \ref{Gfin-algebraic-corollary}, $a_\gphors(z)$ is an algebraic power series. Then, we can, as pointed out in Remark \ref{derivative-rational}, effectively find a rational function $r(z, y)$ such that $a'_\gphors(z)= r(z, a_{\gphors}(z)) $. But then $\mathbb E(\gphors\downarrow) =a'_\gphors(1)= r(1,1)$.
\end{proof}
%
%
%{\color{red}
%
%- rules of the type system
%
%- interpretation in the model
%
%- Theorem1: typable terms yields fps with finitely many non-zero coefficients, each coefficient is algebraic over $\mathbb Q$ 
%
%- Theorem2: typable terms yields an algebraic power series
%
%- Theorem3: AST is decidable (describe first-order formula for AST)
%
%- Example1: order1 random walk, 
%
%- Example2: $0^n1^n0^n$: order 2 non context-free but algebraic (both linear and non-linear)
%
%- Example3: non-algebraic $0^n 1^{2^n}$ is not typable
%
%}


