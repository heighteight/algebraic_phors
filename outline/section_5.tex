% !TEX root = main.tex


In this section we introduce a first type system $\Gfin$ that captures a class of PHORS whose corresponding generating functions are algebraic. We will show that both the AST and PAST problems are decidable for the PHORS typable in this system.

$\Gfin$ is an extension of the affine system of (Ong \& co) with \emph{finitely graded} exponentials $!_kA$ ($k\in \mathbb N$), allowing for a program to use a given input \emph{no more} than some fixed number of times.
 In the next section we will show how to extend this system with \emph{infinite} exponentials $!_\infty A$, enabling programs to use their inputs even an \emph{unbounded} number of times.

\subsection{The system $\Gfin$}


The types of $\Gfin$ are defined by the grammar below:
\begin{align*}
\varphi,\psi&:= \mathsf 1\mid \varphi\&\psi\mid\  \xi\multimap \varphi\\
\xi&:=!_k \varphi \qquad (k\in\mathbb N).
\end{align*}
We define an order relation $\varphi\sqsubseteq \psi$ between types by induction by
\begin{align*}
\infer{\1\sqsubseteq\1}{} \quad
 \infer{\varphi\&\psi\sqsubseteq\varphi'\&\psi'}{\varphi\sqsubseteq \varphi', \psi\sqsubseteq\psi'}
\quad \infer{\xi\multimap\varphi\sqsubseteq\xi'\multimap\varphi'}{\xi\sqsubseteq \xi', \varphi\sqsubseteq\varphi'}
\quad
\infer{!_k\varphi\sqsubseteq !_h\psi}{k\leq h \quad\varphi\sqsubseteq\psi}
\end{align*}

The system will use two kinds of contexts: 
\begin{varitemize}
\item non-linear contexts, noted $\nonterm,\nonterm'$ will be used for non-terminal symbols; there are finite sets of type bindings of the form $L:\varphi$; intuitively, we put no restrictions on the number of times a non-terminal is used;


\item graded contexts, noted $\Delta,\Delta'$ will be used for the arguments to be passed to the non-terminales; they are finite sets of type bindings of the form $x:_k \varphi$; Intuitively, the binding $x :_k \varphi$ means that we can use $x$ with type $\varphi$ \emph{at most} $k$ times.
\end{varitemize}

We define by induction the operations $\Delta+\Delta',k\Delta$ on graded contexts as follows:
\begin{align*}
\Delta +\emptyset &= \Delta, \\
(\Delta,x:_{k}\varphi) + (\Delta',x:_{h}\varphi) &= (\Delta+\Delta'),x:_{k+h}\varphi;\\
%\Delta \vee\emptyset &= \Delta, \\
%(\Delta,x:_{k}\varphi) \vee (\Delta',x:_{h}\varphi) &= (\Delta+\Delta'),x:_{\max\{k,h\}}\varphi;\\
k\emptyset&=\emptyset,\\
k(\Delta,x:_{h}\varphi)  &= k\Delta,x:_{kh}\varphi,
\end{align*}
where, as in (Gaboardi etc.) we are assuming that $\Delta$ and $\Delta'$ agree on each variable (i.e.~they contain the same type bindings but may disagree on the corresponding coefficients). 

A type judgement is an expression of either of the form $\Gamma;\Delta\vdash t:\varphi$ or
$\emptyset;\emptyset\vdash^{\mathrm{fix}} t:\varphi$. The second form of type judgements are called \emph{PHORS-typings}. Notice that in a PHORS-typing we require the context to be empty.
The typing rules are illustrated in Fig.~\ref{fig:typerules}. While most rules are standard graded variants of the rules of the simply typed $\lambda$-calculus (see Gaboardi, Mazza etc.), fixpoints are introduced by a
 \emph{finitary fixpoint rule} $Y_{\mathrm{fin}}$, which produces a PHORS-typings. As we'll see, this rule ensures that the finiteness condition of Definition \ref{def:finitary} holds for the corresponding fps.
Notice that the graded context in $Y_{\mathrm{fin}}$ is required to be empty.




\begin{figure}
\fbox{
\begin{minipage}{.43\textwidth}
\resizebox{.99\textwidth}{!}{
\begin{minipage}{1.1\textwidth}
\begin{align*}
	\begin{array}{ccc}
		\infer[ax1]{\nonterm\mid \Delta\vdash x:\varphi }{x:_p\varphi\in \Delta,\quad 1 \leq p} & &
		\infer[ax2]{\nonterm\mid\Delta \vdash L:\varphi }{ L:\varphi  \in \nonterm}\\[7pt]
%		\infer[!]{\Gamma;k\Delta \vdash t: !_k \varphi}{\Gamma;\Delta \vdash t: \varphi \quad k\in \mathbb N}
%		 & \qquad &
%		 \infer[\oplus]{\Gamma, \Delta \vdash t \oplus t' : \1 }{\Gamma; \Delta \vdash t:\1 \quad \Gamma; \Delta \vdash t':\1} \\[5pt]
%		 
		\infer[\lambda]{\nonterm\mid \Delta \vdash \lambda x. t : !_k \varphi \multimap \psi }{\nonterm \mid \Delta, x:_k \varphi \vdash t : \psi}
		&  &
		\infer[@]{\nonterm\mid \Delta + k\Delta' \vdash tu : \psi}
		{\nonterm\mid\Delta \vdash t :!_k\varphi \multimap \tau \qquad \nonterm\mid \Delta' \vdash u :\varphi }\\[9pt]
		\infer[\langle \rangle]{\nonterm\mid \Delta \vdash \langle t_1, t_2 \rangle : \varphi_1 \& \varphi_2}{\nonterm\mid \Delta \vdash t_i : \varphi_i \; i=1,2}&  &
		\infer[\pi_i]{\nonterm\mid \Delta \vdash \pi_i t: \varphi_i}{\nonterm\mid \Delta \vdash t: \varphi_1 \& \varphi_2}
%		\\[5pt]
%		\infer[\lambda \infty]{\Gamma; \Delta \vdash t: \tau \to \sigma}{\Gamma, L: \tau; \Delta \vdash t: \sigma} 
	\end{array}
	\end{align*}
	\begin{align*}
		 \infer[\oplus]{\nonterm\mid \Delta \vdash t \oplus t' : \1 }{\nonterm\mid \Delta \vdash t:\1 \quad \nonterm\mid \Delta \vdash t':\1} 
	\end{align*}
		\begin{align*}
			\infer[ Y_{\mathrm{fin}}]{\nonterm\mid\emptyset \vdash^{\mathrm{fix}}  Y \lambda \langle L_1,\dots, L_n\rangle.t:\psi }{\nonterm,\langle L_1,\dots, L_n\rangle:\varphi\mid \emptyset \vdash t :  \psi \quad \psi \sqsubseteq \varphi  }
	\end{align*}
	\end{minipage}
	}
\end{minipage}
}
\caption{Typing rules of $\Gfin$.}
\label{fig:typerules}
\end{figure}


\begin{example}
Catalan PHORS
\end{example}

\begin{example}
Non-linear PHORS from Section 2
\end{example}

\begin{example}
Non-algebraic PHORS
\end{example}

%
% there exists a derivation  the following is easily proved by induction:
%\begin{proposition}[$\Gfin$ refines $\PLY$]
%For every derivation $\pi:\nonterm\mid\Delta \vdash_{\Gfin}t:\varphi$, given $\nonterm\mid\Delta\tri \Gamma$ and $\varphi\tri T$, there exists a derivation $\pi^*:\Gamma\vdash_{\lambda_{\oplus} }t:T$.
%\end{proposition}


\subsection{Extracting Fas from $\Gfin$}

To define the relational interpretation of $\Gfin$ we could exploit the well-known fact that the family of functors $(!_k)_{k\in \N}:\Qrel{\Rsemiring}\to\Qrel{\Rsemiring}$ (recall that $!_kX$ is the subset of $!X$ formed by the multisets of cardinality at most $k$) defines a \emph{graded linear exponential comonad} (cite Katsumata etc.), and provides then a sound interpretation of the graded rules of $\Gfin$. 

%More precisely, define $\model{\1}=\1$, $\model{\varphi\&\psi}=\model{\varphi}\times\model{\psi}$ and $\model{!_k\varphi}=!_k\model{\varphi}$. Observe that all sets $\model{\varphi}$ are finite.
%Then, for any derivation $\nonterm\mid\Delta\vdash t:\varphi$, where $\nonterm=\{L_1:\varphi_1,\dots, L_n:\varphi_n\}$
%and $\Delta=\{x_1:_{k_1}\psi_1,\dots, x_m:_{k_m}\psi_m\}$, we obtain a morphism
%\[
%\model{\nonterm\mid\Delta\vdash t:\varphi}^{\Rsemiring}\in
%\Qrel{\Rsemiring}\left(
%\sum_i !\model{\varphi_i}+ \sum_j !_{k_j}\model{\psi_j}, \model{\varphi}
%\right)
%\]

However, we wish to look at $\Gfin$ as a tool to extract information about the power series arising from (a subset of) the PHORS, i.e.~of the terms typable in $\lambda_{\oplus}Y$. 
For this, observe first that the rules (and indeed the derivations) of $\Gfin$ can be seen as \emph{quantitative refinements} of those of $\PLY$. First, every type $\varphi$ is (uniquely) a \emph{refinement} of some simple type $T$: the refinement relation $\varphi\tri T$ is defined by:
\[
\infer{\1\tri\1}{}\qquad
\infer{\varphi\&\psi\tri T\& U }{\varphi\tri T & \psi\tri U}\qquad
\infer{!_k\varphi\multimap\psi\tri T\to U }{\varphi\tri T & \psi\tri U}
\]
Notice that $\varphi\sqsubseteq \psi$ implies $\varphi\tri T\Leftrightarrow \psi\tri T$. 
Next, for every derivation $\pi:\nonterm\mid\Delta \vdash_{\Gfin}t:\varphi$, 
by replacing each $\psi$ by $U$, where $\psi\tri U$, we obtain, by induction, a $\PLY$-derivation $\Gamma\vdash_{\lambda_{\oplus} }t:T$, where $\nonterm\mid\Delta\tri \Gamma$ and $\varphi\tri T$. 
We will say that $\pi$ \emph{refines} $\pi^\tri$ (noted $\pi\tri\pi^{\tri}$).


Now, for a term $t:\varphi$ typable in $\Gfin$, where $\varphi\tri T$, we will define its interpretation $\model{t:\varphi\tri T}^{\Rsemiring}$ as a suitably define \emph{restriction} of the interpretation $\model{t:T}^{\Rsemiring}$ of the corresponding $\lambda_{\oplus}Y$-derivation.
%
%Recall that the interpretation $\model{\Gamma\vdash t:T}^{\Rsemiring}$ of a $\PLY$-derivation in $\Qrelkleisli{\Rsemiring}$ is defined by induction on the rules, associating the axiom rule $\Gamma, x:T\vdash x:T$ with the projection $\Qrelkleisli{\Rsemiring}(\model{\Gamma}+\model{T},\model{T})$, the rule introduction $\oplus$ with weighted sums, and all other rules with the corresponding operation on $\Qrelkleisli{\Rsemiring}$ given by cartesian closure.

More precisely, given a $\Gfin$-derivation $\pi:\nonterm\mid\Delta\vdash t:\varphi$, given $\nonterm\tri\Gamma_{\nonterm}, \Delta\tri\Gamma_{\Delta}$ and $\varphi\tri T$, we define 
\[
\model{t:\varphi\tri T}^{\Rsemiring}\in\Qrelkleisli{\Rsemiring}(\model{\Gamma_{\nonterm}}+\model{\Gamma_{\Delta}},\model{T})
\]  
 by induction as follows:
\begin{varitemize}

\item if $\pi$ is an instance of $ax1$ or $ax2$, 
$\model{t:\varphi\tri T}^{\Rsemiring}:=\left(
\model{t:T}^{\Rsemiring}\right)\Big\vert_{\model{\nonterm}+\model{\Delta}}^{\model{\varphi}}
$, where, given a $X$-family of fps $s_X\in \fps{\Rsemiring}{Y}^X$, $X'\subset X$ and $Y'\subset Y$, $(s_X)\vert_{Y'}^{X'}=(s_{X'})\vert_{Y'}\in \fps{\Rsemiring}{Y'}^{X'}$ is the $X'$-family of restrictions of $s_{X'}$ to the variables in $Y'$;

\item all other rules are interpreted as the corresponding ones in $\lambda_{\oplus}Y$.

\end{varitemize}


The following result ensures that the interpretation of a $\Gfin$-derivation coincides with the restriction of the corresponding $\lambda_{\oplus}Y$-derivation to its finite domain and codomain:



\begin{proposition}\label{prop:finiok}
Given $\nonterm\mid\Delta\vdash t:\varphi$, with $\varphi\tri T$,  
\[
\model{t:\varphi\triangleleft T}^{\Rsemiring}=\left(\model{t:T}^{\Rsemiring}\right)\Big\vert^{\model{\varphi}}_{\model{\nonterm}+\model{\Delta}}.
\]
\end{proposition}
As the type $\1$ is the only refinement of itself, the finitary interpretations $\model{t:\1\tri 1}^{\Rsemiring}$ correctly capture $\mathbb P(t\downarrow)$ and $\mathbb E(t\downarrow)$.

\begin{corollary}
For all derivation $\emptyset\mid\emptyset\vdash t:\1$, $\model{t:\1\triangleleft\1}^{\Rinf}=\mathbb P(t\downarrow)$ and $\left(\model{t:\1\triangleleft\1}^{\fps{\Rinf}{z}}\right)'(1)=\mathbb E(t\downarrow)$.
\end{corollary}


At this point, observe that a consequence of Proposition \ref{prop:finiok} and of the finiteness of the types of $\Gfin$ is that the fps $\model{t:\varphi\tri T}^{\Rsemiring}$ must have a finite support. This ensures that the rule $Y_{\mathrm{fin}}$ always produces a fas:




\begin{proposition}
For all derivations of $\langle L_1,\dots, L_n\rangle:\varphi\mid\emptyset\vdash t:\psi$, with $\psi\sqsubseteq \varphi\tri T$, the fps
$\model{M:\psi\tri T}^{\Rsemiring}\in \fps{\Rsemiring}{\model{T}+\model{\Gamma}}^{\model{T}}$ is finitary.
\end{proposition}

Thanks to Proposition \ref{prop:fintoalg}, we obtain then:
\begin{theorem}[typable PHORS are algebraic]
For all PHORS $\gphors$ typable in $\Gfin$, $\model{t_S:\1\triangleleft\1}^{\Rinf}\in \Rinf$ is $\mathbb Q^{+}$-algebraic and 
 $\model{t_S:\1\triangleleft\1}^{\fps{\Rinf}{z}}\in \fps{\Rinf}{z}$ is $\fps{\mathbb Q^{+}}{z}$-algebraic.



\end{theorem}




\subsection{Decidability of AST and PAST}



%
%
%{\color{red}
%
%- rules of the type system
%
%- interpretation in the model
%
%- Theorem1: typable terms yields fps with finitely many non-zero coefficients, each coefficient is algebraic over $\mathbb Q$ 
%
%- Theorem2: typable terms yields an algebraic power series
%
%- Theorem3: AST is decidable (describe first-order formula for AST)
%
%- Example1: order1 random walk, 
%
%- Example2: $0^n1^n0^n$: order 2 non context-free but algebraic (both linear and non-linear)
%
%- Example3: non-algebraic $0^n 1^{2^n}$ is not typable
%
%}


