% !TEX root = main.tex


In this section we introduce a finitely graded $\Gfin$, a class of PHORS whose corresponding generating functions are algebraic. We will show that both the AST and PAST problems are decidable for the PHORS typable in this system.

$\Gfin$ are an extension of the additive PHORS of (Ong \& co) with \emph{finitely graded} exponentials $!_kA$ ($k\in \mathbb N$), allowing for a program to use a given input \emph{no more} than some fixed number of times.
 In the next section we will show how to extend this system with \emph{infinite} exponentials $!_\infty A$, enabling programs to use their inputs even an \emph{unbounded} number of times.

\subsection{$\Gfin$: Finitely Graded PHORS }


We use types defined by the grammar below:
\begin{align*}
\varphi,\psi&:= \1^n\mid \  !_k\varphi\multimap \varphi
%\\
%\xi&:=\ !_k \varphi
 \qquad (k\in\mathbb N).
\end{align*}
We let $\tau$ be an abbreviation for the affine function type $!_1o\multimap o$. 
We define an order relation $\varphi\sqsubseteq \psi$ between types by induction by
\begin{align*}
\infer{\1^n\sqsubseteq\1^n}{}
% \quad
% \infer{\varphi\&\psi\sqsubseteq\varphi'\&\psi'}{\varphi\sqsubseteq \varphi', \psi\sqsubseteq\psi'}
\quad \infer{\xi\multimap\varphi\sqsubseteq\xi'\multimap\varphi'}{\xi\sqsubseteq \xi', \varphi\sqsubseteq\varphi'}
\quad
\infer{!_k\varphi\sqsubseteq !_h\psi}{k\leq h \quad\varphi\sqsubseteq\psi}
\end{align*}

The typing rules will use two kinds of contexts: 
\begin{varitemize}
\item non-linear contexts, noted $\nonterm,\nonterm'$ will be used for non-terminal symbols; there are finite sets of type bindings of the form $L:\varphi$; intuitively, we put no restrictions on the number of times a non-terminal is used;


\item graded contexts, noted $\Delta,\Delta'$ will be used for the arguments to be passed to the non-terminales; they are finite sets of type bindings of the form $x:_k \varphi$; Intuitively, the binding $x :_k \varphi$ means that we can use $x$ with type $\varphi$ \emph{at most} $k$ times.
\end{varitemize}

We define by induction the operations $\Delta+\Delta',k\Delta$ on graded contexts as follows:
\begin{align*}
\Delta +\emptyset &= \Delta, \\
(\Delta,x:_{k}\varphi) + (\Delta',x:_{h}\varphi) &= (\Delta+\Delta'),x:_{k+h}\varphi;\\
%\Delta \vee\emptyset &= \Delta, \\
%(\Delta,x:_{k}\varphi) \vee (\Delta',x:_{h}\varphi) &= (\Delta+\Delta'),x:_{\max\{k,h\}}\varphi;\\
k\emptyset&=\emptyset,\\
k(\Delta,x:_{h}\varphi)  &= k\Delta,x:_{kh}\varphi,
\end{align*}
where, as in (Gaboardi etc.) we are assuming that $\Delta$ and $\Delta'$ agree on each variable (i.e.~they contain the same type bindings but may disagree on the corresponding coefficients). 

A type judgement is an expression of the form $\Gamma;\Delta\vdash t:\varphi$.
% or
%$\emptyset;\emptyset\vdash^{\mathrm{fix}} t:\varphi$. The second form of type judgements are called \emph{PHORS-typings}. Notice that in a PHORS-typing we require the context to be empty.
The typing rules are illustrated in Fig.~\ref{fig:typerules}.
\begin{definition}[$\Gfin$]
A \emph{finitely graded PHORS} (noted $\Gfin$) is a triple $\gphors=(\nonterm, \C R, S)$, where $\nonterm$ is a finite set of typed non-terminals, $S\in\nonterm$ is the {start symbol} such that $\nonterm(S)=\1$, and $\C R$ is a function that associates each $L\in\nonterm$ with a derivation $\nonterm\mid\emptyset\vdash \lambda x_1.\dots. x_n.(t_L \oplus_p t_R):\varphi$ such that $\varphi\sqsubseteq \nonterm(L)$, $t$ contains no $\lambda$-abstraction and $\nonterm\mid x_1,\dots, x_k\vdash t_L \oplus t_R:\1$. 
\end{definition} 
Since affine implication is expressed by $!_1\varphi\multimap \psi$ and every derivation $\nonterm\mid x_1,\dots, x_k\vdash t :\1$ can be changed into a derivation  $\nonterm\mid x_1,\dots, x_k\vdash t \oplus_1 e :\1$ without modifying the $\Qrelkleisli{\Rinf}$ interpretation of the term, it is immediate that all PAHORS are $\Gfin$:
\begin{proposition}
Every PAHORS is a $\Gfin$.
\end{proposition}

%
%While most rules are standard graded variants of the rules of the simply typed $\lambda$-calculus (see Gaboardi, Mazza etc.), fixpoints are introduced by a
% \emph{finitary fixpoint rule} $Y_{\mathrm{fin}}$, which produces a PHORS-typings. As we'll see, this rule ensures that the finiteness condition of Definition \ref{def:finitary} holds for the corresponding fps.
%Notice that the graded context in $Y_{\mathrm{fin}}$ is required to be empty.
%



\begin{figure}
\fbox{
\begin{minipage}{.43\textwidth}
\resizebox{.99\textwidth}{!}{
\begin{minipage}{1.1\textwidth}
\begin{align*}
	\begin{array}{ccc}
		\infer[ax1]{\nonterm\mid \Delta\vdash x:\varphi }{x:_p\varphi\in \Delta,\quad 1 \leq p} & &
		\infer[ax2]{\nonterm\mid\Delta \vdash L:\varphi }{ L:\varphi  \in \nonterm}\\[7pt]
%		\infer[!]{\Gamma;k\Delta \vdash t: !_k \varphi}{\Gamma;\Delta \vdash t: \varphi \quad k\in \mathbb N}
%		 & \qquad &
%		 \infer[\oplus]{\Gamma, \Delta \vdash t \oplus t' : \1 }{\Gamma; \Delta \vdash t:\1 \quad \Gamma; \Delta \vdash t':\1} \\[5pt]
%		 
		\infer[\lambda]{\nonterm\mid \Delta \vdash \lambda x. t : !_k \varphi \multimap \psi }{\nonterm \mid \Delta, x:_k \varphi \vdash t : \psi}
		&  &
		\infer[@]{\nonterm\mid \Delta + k\Delta' \vdash tu : \psi}
		{\nonterm\mid\Delta \vdash t :!_k\varphi \multimap \tau \qquad \nonterm\mid \Delta' \vdash u :\varphi }\\[9pt]
		\infer[\langle \rangle]{\nonterm\mid \Delta \vdash \langle t_1, \dots,t_n \rangle : \1^n}{\nonterm\mid \Delta \vdash t_i : \1 &  i=1,\dots, n}&  &
		\infer[\pi_i]{\nonterm\mid \Delta \vdash \pi_i t: 1}{\nonterm\mid \Delta \vdash t: 1^n &  i=1,\dots, n}
%		\\[5pt]
%		\infer[\lambda \infty]{\Gamma; \Delta \vdash t: \tau \to \sigma}{\Gamma, L: \tau; \Delta \vdash t: \sigma} 
	\end{array}
	\end{align*}
	\begin{align*}
		 \infer[\oplus]{\nonterm\mid \Delta \vdash t \oplus t' : \1 }{\nonterm\mid \Delta \vdash t:\1 \quad \nonterm\mid \Delta \vdash t':\1} 
	\end{align*}
%		\begin{align*}
%			\infer[ Y_{\mathrm{fin}}]{\nonterm\mid\emptyset \vdash^{\mathrm{fix}}  Y \lambda \langle L_1,\dots, L_n\rangle.t:\psi }{\nonterm,\langle L_1,\dots, L_n\rangle:\varphi\mid \emptyset \vdash t :  \psi \quad \psi \sqsubseteq \varphi  }
%	\end{align*}
	\end{minipage}
	}
\end{minipage}
}
\caption{Typing rules of $\Gfin$.}
\label{fig:typerules}
\end{figure}


As the PHORS from Examples \ref{ex:phors1}-\ref{ex:phors3} is affine, it is also a $\Gfin$.

\begin{example}\label{ex:nonlin}
Let $\gphors$ be the $\Gfin$ from \eqref{eq:phors1} in Section 2, 
While $A,B$ are affine, since $\nonterm(A)=\nonterm(B)=\tau$, 
we have $\nonterm(H)=\ !_2\tau \multimap \tau$, since $Hyx$ may use the functional variable $y$ at most two times. $\gphors$ is thus not a PAHORS.



\end{example}



%
% there exists a derivation  the following is easily proved by induction:
%\begin{proposition}[$\Gfin$ refines $\PLY$]
%For every derivation $\pi:\nonterm\mid\Delta \vdash_{\Gfin}t:\varphi$, given $\nonterm\mid\Delta\tri \Gamma$ and $\varphi\tri T$, there exists a derivation $\pi^*:\Gamma\vdash_{\lambda_{\oplus} }t:T$.
%\end{proposition}


While $\Gfin\supsetneq PAHORS$, the set of corresponding branch languages do indeed coincide: this means that for any $\Gfin$ it is possible to find a PAHORS defining the same branch language.

Define the \emph{size} of a $\gphors$ as the sum of the sizes of its terms, i.e.~$\|\gphors\|=|t_1|+\dots+ |t_n|$. Moreover, let $\partial\gphors$ be the maximum grade occurring in $\gphors$.

\begin{theorem}[Affinization of $\Gfin$]
For every $\Gfin$ $\gphors=(\nonterm, \C R, S)$ there exists a PAHORS $\AFF\gphors=(\nonterm, \C R',S)$, with same non-terminals, size $\|\AFF\gphors\|\leq \partial\gphors{\|\gphors\|}$ and such that $\C L(\gphors)=\C L(\AFF\gphors)$.

\end{theorem}
\begin{proof}[Proof sketch]
Define $\AFF{\varphi}$ as $
\AFF{(o^n)}=o^n$ and 
$\AFF{(!_k\varphi\multimap \psi)}={\AFF{\varphi}\multimap\dots\multimap\AFF{\varphi}}\multimap\AFF\psi$, with $\AFF{\varphi}$ repeated $k$ times. 
Then, by induction, any derivation of $\nonterm\mid \Delta\vdash t:\varphi$ yields a derivation of $\AFF{\nonterm}\mid\Delta'\vdash t:\AFF \varphi$ in the affine system of (Ong et al.), where $|\Delta'|\leq \partial\gphors|\Delta|$: an order-$n$ $\Gfin$ yields then an order-$n$ PAHORS in which every equation $Lx_1\dots x_n$ translates into a new equation $Lz_1^1\dots z_{n}^{k_n}$, where, if $\nonterm(L)=!_{k_1}\varphi_1\multimap\dots \multimap !_{k_i}\varphi_i\multimap\dots\multimap o^n$, the variable $x_i$ is replaced by $k_i$ variables $z_i^1,\dots, z_{i}^{k_i}$.
\end{proof}


\begin{example}
The affinization of the $\Gfin$ from Example \ref{ex:nonlin} yields the PAHORS below:
\begin{align*}\label{eq:phors2}
Ly_1y_2x&=( L(A\circ y_1)(A\circ y_2)x \oplus_{a}
L(B\circ y_1)(B\circ y_2)x )\oplus_{a}y_1(y_2x)\\
Ax&= x\oplus_{b} \Omega\\
Bx&=x\oplus_{c} \Omega\\
S&=HIIe
\end{align*}
Notice that the unique functional variable $y$, that was used twice, is now replaced by \emph{two} functional variables $y_1,y_2$, used once.
\end{example}

In a $\Gfin$, the degree $\partial \gphors$ can well be exponential in $\|\gphors\|$, so that $\AFF\gphors$ has size exponential in $\|\gphors\|$:

\begin{example}
Consider the order-2 $\Gfin$ $\gphors$ given by:
\begin{align*}
F_1fx&= f(fx)\\
F_2f x&= F_1(f\circ f) x \oplus_{p} x\\
%F_2f x&= F_3(f\circ f) x \oplus_{p} x\\
&\vdots\\
F_{n}fx&= F_{n-1}(f\circ f)x\oplus_{p}x\\
S&=F_n \ \mathrm{id} \  e
\end{align*}
For all $i=1,\dots,n$, $\nonterm(F_i)=\ !_{2^{i}}\tau\multimap\tau$, whence
$\|\AFF{\gphors}\| \leq \partial\gphors\|\gphors\|=
2^{\C O(\|\gphors\|)}\|\gphors\|$.
\end{example}



\subsection{Extracting Fas from $\Gfin$}


%More precisely, define $\model{\1}=\1$, $\model{\varphi\&\psi}=\model{\varphi}\times\model{\psi}$ and $\model{!_k\varphi}=!_k\model{\varphi}$. Observe that all sets $\model{\varphi}$ are finite.
%Then, for any derivation $\nonterm\mid\Delta\vdash t:\varphi$, where $\nonterm=\{L_1:\varphi_1,\dots, L_n:\varphi_n\}$
%and $\Delta=\{x_1:_{k_1}\psi_1,\dots, x_m:_{k_m}\psi_m\}$, we obtain a morphism
%\[
%\model{\nonterm\mid\Delta\vdash t:\varphi}^{\Rsemiring}\in
%\Qrel{\Rsemiring}\left(
%\sum_i !\model{\varphi_i}+ \sum_j !_{k_j}\model{\psi_j}, \model{\varphi}
%\right)
%\]

For this, observe first that the rules (and indeed the derivations) of $\Gfin$ can be seen as \emph{quantitative refinements} of those of $\PLY$. First, every type $\varphi$ is (uniquely) a \emph{refinement} of some simple type $T$: the refinement relation $\varphi\tri T$ is defined by:
\[
\infer{\1\tri\1}{}\qquad
\infer{\varphi\&\psi\tri T\& U }{\varphi\tri T & \psi\tri U}\qquad
\infer{!_k\varphi\multimap\psi\tri T\to U }{\varphi\tri T & \psi\tri U}
\]
Notice that $\varphi\sqsubseteq \psi$ implies $\varphi\tri T\Leftrightarrow \psi\tri T$. 
Next, for every derivation $\pi:\nonterm\mid\Delta \vdash_{\Gfin}t:\varphi$, 
by replacing each $\psi$ by $U$, where $\psi\tri U$, we obtain, by induction, a simply typed derivation $\Gamma\vdash_{\lambda_{\oplus} }t:T$, where $\nonterm\mid\Delta\tri \Gamma$ and $\varphi\tri T$. 
%We will say that $\pi$ \emph{refines} $\pi^\tri$ (noted $\pi\tri\pi^{\tri}$).

Define, for every type $\varphi$, $\model{\varphi}$ as $\model{o^n}=1$ and $\model{!_k\varphi\multimap \psi}=!_k\model{\varphi}\times\model{\psi}$. Notice that $\varphi\tri T$ implies $\model{\varphi}\subseteq\model{T}$.


Now, for a term $t:\varphi$ typable in $\Gfin$, where $\varphi\tri T$, we will define its interpretation $\model{t:\varphi\tri T}^{\Rsemiring}$, intuitively, as the \emph{restriction} of the interpretation $\model{t:T}^{\Rsemiring}$ to $\model{\varphi}$.
%
%Recall that the interpretation $\model{\Gamma\vdash t:T}^{\Rsemiring}$ of a $\PLY$-derivation in $\Qrelkleisli{\Rsemiring}$ is defined by induction on the rules, associating the axiom rule $\Gamma, x:T\vdash x:T$ with the projection $\Qrelkleisli{\Rsemiring}(\model{\Gamma}+\model{T},\model{T})$, the rule introduction $\oplus$ with weighted sums, and all other rules with the corresponding operation on $\Qrelkleisli{\Rsemiring}$ given by cartesian closure.
More precisely, given a $\Gfin$-derivation $\pi:\nonterm\mid\Delta\vdash t:\varphi$, given $\nonterm\tri\Gamma_{\nonterm}, \Delta\tri\Gamma_{\Delta}$ and $\varphi\tri T$, we define 
\[
\model{t:\varphi\tri T}^{\Rsemiring}\in\Qrelkleisli{\Rsemiring}(\model{\Gamma_{\nonterm}}+\model{\Gamma_{\Delta}},\model{T})
\]  
 by induction as follows:
\begin{varitemize}

\item if $\pi$ is an instance of $ax1$ or $ax2$, 
$\model{t:\varphi\tri T}^{\Rsemiring}:=\left(
\model{t:T}^{\Rsemiring}\right)\Big\vert_{\model{\nonterm}+\model{\Delta}}^{\model{\varphi}}
$, where, given a $X$-family of fps $s_X\in \fps{\Rsemiring}{Y}^X$, $X'\subset X$ and $Y'\subset Y$, $(s_X)\vert_{Y'}^{X'}=(s_{X'})\vert_{Y'}\in \fps{\Rsemiring}{Y'}^{X'}$ is the $X'$-family of restrictions of $s_{X'}$ to the variables in $Y'$;

\item all other rules are interpreted as the corresponding simply-typed rules.

\end{varitemize}


The following result ensures that the interpretation of a $\Gfin$-derivation coincides with the restriction of the corresponding simply typed derivation to its finite domain and codomain:



\begin{proposition}\label{prop:finiok}
Given $\nonterm\mid\Delta\vdash t:\varphi$, with $\varphi\tri T$,  
\[
\model{t:\varphi\triangleleft T}^{\Rsemiring}=\left(\model{t:T}^{\Rsemiring}\right)\Big\vert^{\model{\varphi}}_{\model{\nonterm}+\model{\Delta}}.
\]
\end{proposition}

For every $\Gfin$ $\gphors=(\nonterm, \C R,S)$, we can define the interpretation of each non-terminal $L_i$ as we did for PHORS, i.e.~ 
$
\model{L_i}^{\Rsemiring}:=  \model{\pi_i(Yt_{\gphors})}=\pi_i\left(\fix \model{t_{\gphors}}\right)\in \Rsemiring^{\model{\nonterm(L_i)}}.
$ 

The following is at this point an immediate consequence of the remark that the type $\1$ is the only refinement of itself:
% the finitary interpretations $\model{t:\1\tri \1}^{\Rsemiring}$ correctly capture $\mathbb P(t\downarrow)$ and $\mathbb E(t\downarrow)$.

\begin{corollary}
For every $\Gfin$ $\gphors=(\nonterm, \C R,S)$, 
\[
\model{S}^{\Rinf}=\mathbb P(\gphors\downarrow),
\qquad
\left(\model{S}^{\fps{\Rinf}{z}}\right)'(1)=\mathbb E(\gphors\downarrow).\]
\end{corollary}




At this point, observe that a consequence of Proposition \ref{prop:finiok} and of the finiteness of the types of $\Gfin$ is that the fps $\model{t:\varphi\tri T}^{\Rsemiring}$ must have a finite support. This leads to the follows:



\begin{proposition}
For every $\Gfin$ $\gphors=(\nonterm,\C R,S)$, the fps
$\model{t_{\gphors}}^{\Rsemiring}\in \fps{\Rsemiring}{\model{\nonterm}}^{\model{\nonterm}}$ is finitary.
% derivations of $\nonterm\mid\emptyset\vdash t_L:\varphi$, with $\varphi\sqsubseteq \nonterm{L}\tri T$, the fps
%$\model{M:\psi\tri T}^{\Rsemiring}\in \fps{\Rsemiring}{\model{T}+\model{\Gamma}}^{\model{T}}$ is finitary.
\end{proposition}

Thanks to Proposition \ref{prop:fintoalg} and the observation that all coefficients of $\model{t_\gphors}^{\Rinf}$ are rational, we obtain then:
\begin{theorem}
	\label{Gfin-algebraic-theorem}
	For all $\Gfin$ $\gphors=(\nonterm, \C R, S)$, $(\model{L}^{\Rinf})_{L \in \nonterm}$ is a $\Q^{+}$ fixpoint algebraic family with no parameters and $(\model{L}^{\fps{\Rinf}{z}})_{L \in \nonterm}$ is a ${\Q^{+}}$ fixpoint algebraic family with parameter $z$
\end{theorem}
Now observe that the term $t_\gphors$ is in the form $\langle t_{1, L} \oplus_{p_n}  t_{1, R} \dots  t_{n, L} \oplus_{p_n} t_{n_R} \rangle$. We thus have that every fps $s(x_{\model{\nonterm},} z)$ belonging to $\model{t_\gphors}^{\fpp{\Rinf}{z}}$ is divisible by $z$. In particular $s(0_{\model{\nonterm}} 0)=0$ and for every variable $x \in x_{\model{\nonterm}}$ the monomial $x$ does not appear in $s$: hence, $\model{t_\gphors}^{\fpp{\Rinf}{z}}$ is a proper algebraic family. Then, by \ref{cor:propertoalg}, each $\model{L}^{\fps{\Rinf}{z}}$ is an algebraic power series:
\begin{corollary}[$\Gfin$ are algebraic]
	\label{Gfin-algebraic-corollary}
	 For all $\Gfin$ $\gphors=(\nonterm, \C R, S)$,
	\begin{enumerate}
\item $\mathbb P(\gphors\downarrow)$ is a $\mathbb Q^{+}$ algebraic number and 
 $\sum_i\mathbb P(\gphors\downarrow_i)z^i$ is a ${\mathbb Q^{+}}$-algebraic power series.
\item for each $L \in \nonterm$, $\model{L_i f_1 \dots f_n : \1}^{\Rinf}$ is an algebraic power series.  
\end{enumerate}
\end{corollary}
\begin{proof}
	For (1), notice that $\model{S}^{\Rinf}=\model{S}^{\fps{\Rinf}{z}}$
\end{proof}
As a consequence of $\ref{remark:semilinear}$ we also obtain:
\begin{corollary}
	For all $\Gfin$ $\gphors=(\nonterm, \C R, S)$, the set $$\{n \mid \:  S \text{ terminates in } n \text{ steps}  \}$$
	is semilinear
\end{corollary}
\begin{remark}
	By interpreting  $t_L \oplus t_R$ by $z_1 \model{t_L}^{\Rinf} + z_2 \model{t_L}^{\Rinf}$, the previous corollary could be strengthened to say that the set:
	$$\{(n,m) \mid \:  S \text{ terminates with } n \text{ left steps and } m \text{ rigth steps}\}$$ is semilinear.
	This could also been deduced by the fact that the branch language of a linear HORS is multiple context-free (cite  Clairambault) and languages of this class are know to be semilinear (cite CHARACTERIZING STRUCTURAL DESCRIPTIONS PRODUCED BY VARIOUS.
	GRAMMATICAL FORMALISMS), but it is still worth noticing that it can be proved by means of generating function techniques.
\end{remark}
%\begin{example}
%Consider again $\gphors$ from Example \ref{ex:nonlin}.
%
%
%
%
%\end{example}

\begin{example}[a non-algebraic PHORS]
The PHORS $\gphors$ below:
\begin{align*}
Lyx &= L(y\circ y) x\oplus_{p} fx \\
S&= L\ \mathrm{id}\  e
\end{align*}
is \emph{not} a $\Gfin$: if we try to type $t_L=L(y\circ y) x\oplus_{p} fx$ we obtain
$L:\varphi \mid \emptyset\vdash \lambda y.\lambda x.t_L: \psi$, where $\psi=\ !_2\tau\multimap \tau\not\sqsubseteq \ !_1\tau\multimap\tau=\varphi$. 
Indeed, the generating function $s^L(y_{\N},x)\in \fps{\Rsemiring}{\N+1}$ of $L$ satisfies 
$s^L(y_\N,x)=\frac{1}{2}(
y_1x+s^L( y^2_\N,x)
)$, which yields the solution $s^L(y_\N,x)=\sum_i\frac{1}{2^{i+1}}y^{2^i}x$, that is not algebraic (cf. Remark~\ref{rem:hadamard}). 
\end{example}


\begin{remark}

We could have also defined the interpretation of $\Gfin$ via the well-known fact that the family of functors $(!_k)_{k\in \N}:\Qrel{\Rsemiring}\to\Qrel{\Rsemiring}$ defines a \emph{graded linear exponential comonad} (cite Katsumata etc.), yielding a truly finite semantics in which $\model{!_k\varphi}=!_k\model{\varphi}$. 
%
%. This would interpret a type $\varphi$, with $\varphi\tri T$ as a suitable finite subset of $\model{T}$. In our semantics we still have $\model{\varphi\tri T}=\model{T}$, although 
However, we chose to keep the infinitary semantics $\model{\varphi\tri T}=\model{T}$ as we wish to look at $\Gfin$ as a tool to extract information the power series arising from (a subset of the terms of) $\lambda_{\oplus}Y$. 

\end{remark}

\subsection{Decidability of AST and PAST}
The  proof of \ref{Gfin-algebraic-theorem} is constructive: it gives a way to build effectively a Fas $(\vec x = p_i(\vec x))_{1 \leq i \leq n}$ that has $(\model{L}^{\Rinf})_{L \in \nonterm}$ as its (minimal) solution by computing for each $L \in \nonterm$ the interpretation $\model{\mathcal{R}(L)}^{\Rinf}$, where $\mathcal{R}(L)$ is a fixpoint-free term. 
Once we have the Fas, we can check if $\mathbb P(\gphors\downarrow) = \model{S}^{\Rinf}$ by using the first order Existential Theory of the Reals $\Etor$ (cite Etessami-Yannakakis, 09): first, we create a first order formula $\phi( \vec x)$ expressing the fact that $\vec x$ satisfies $\vec x = p_i(\vec x)$, then we can express the fact that $x_1=1$ by the formula $\exists \vec{x}(x_1=1 \land \phi(\vec x) \land \forall \vec y ((\vec y < \vec x) \implies \neg \phi(\vec y)  ))$\\
As for PAST, remember in the first place that PAST implies AST. Then, given a $\gphors=(\nonterm, \C R, S) \in \Gfin$, we can in the first place test if it is AST: if it is not, then it not PAST. If it AST, notice that by \ref{Gfin-algebraic-corollary}, $\model{S}^{\fps{\Rinf}{z}}$ is an algebraic power series. Then, we can, as pointed out in remark \ref{derivative-rational}, find a rational function $r(z, y)$ such that ${\model{S}^{\fps{\Rinf}{z}}}' = r(z, \model{S}^{\fps{\Rinf}{z}}) $. But then $$E(\gphors\downarrow) = {\model{S}^{\fps{\Rinf}{z}'}}'(1)= r(1,1)$$ 
We thus obtain:
\begin{theorem}
	AST and PAST are decidable for $\Gfin$.	
\end{theorem}


%
%
%{\color{red}
%
%- rules of the type system
%
%- interpretation in the model
%
%- Theorem1: typable terms yields fps with finitely many non-zero coefficients, each coefficient is algebraic over $\mathbb Q$ 
%
%- Theorem2: typable terms yields an algebraic power series
%
%- Theorem3: AST is decidable (describe first-order formula for AST)
%
%- Example1: order1 random walk, 
%
%- Example2: $0^n1^n0^n$: order 2 non context-free but algebraic (both linear and non-linear)
%
%- Example3: non-algebraic $0^n 1^{2^n}$ is not typable
%
%}


