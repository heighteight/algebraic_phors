% !TEX root = main.tex

\paragraph*{Probabilistic Higher-Order Verification}


% Probabilistic programming provides a well-established family of methods and tools to specify probabilistic models as well as to perform statistical inference tasks using high-level, compositional, languages. For instance, languages like \emph{Church} or \emph{Anglican} enable the design of generative Bayesian models through finite lists of higher-order and recursive specifications. This provides the user with a user-friendly and high-level environment for stochastic reasoning, allowing them at the same time to implement general purpose inference methods via uniform and re-usable code, without having to adapt them to each model by hand.

% A key advantage of this approach to probabilistic modeling is the possibility of exploiting the theoretical background of programming language theory to design verification and model-checking methods for such languages in a way which is both compositional and grounded on solid mathematical foundations. In the deterministic setting, a vast literature has explored the model-checking of \emph{higher-order recursive schemes} (HORS) \cite{}, a specification language equivalent to the simply-typed $\lambda$-calculus enriched with a fixpoint operator, building over the paradigmatic result from \cite{} of the decidability, over such programs, of all properties expressible in monadic second order logic (MSO), including termination. 

Model checking is one of the most successful techniques for the verification of programs and systems, and it has found applications in a wide range of areas, leading to the development of widely used tools such as SPIN and PRISM. While in its original formulation model checking focused on finite-state systems modeled by Kripke structures, its scope of application has gradually expanded over time, in directions that are often very different from one another. In this paper, we are particularly interested in two such directions.

The first concerns the model checking of systems whose evolution is intrinsically \emph{probabilistic}. Along this line of research, the literature has produced a rich body of results, both positive and negative, regarding the decidability and tractability of the underlying verification problem (see \cite{} \warn for a recent survey). It is worth noting that when the property to be verified is a reachability problem, or analogously termination (appropriately generalized to a probabilistic setting), the verification problem remains decidable even beyond the finite-state case, as in, for example, recursive Markov chains or pushdown automata.

But there is also another generalization of model checking, likewise obtained by considering a broader class of systems than those traditionally studied, namely so-called \emph{higher-order model checking}. It has been known for about twenty years that this problem is decidable when the underlying specification is an arbitrary MSO formula and the program is an arbitrary \emph{higher-order recursion scheme} (HORS, \cite{} \warn), a specification language equivalent to the simply-typed $\lambda$-calculus enriched with a fixpoint operator. Problems such as termination or reachability are therefore decidable for a broad class of programs encompassing higher-order types and recursion. These positive results have nonetheless revealed
%been extensively examined with respect to extensions of the language, revealing 
a certain degree of fragility: relatively minor extensions suffice to render the problem undecidable.

What, then, happens when one considers model-checking problems for probabilistic \emph{and} higher-order programs? Recently, Dal Lago et al.~\cite{DBLP:journals/lmcs/KobayashiLG20} have introduced \emph{probabilistic higher-order recursive schemes} (PHORS), a probabilistic variant of HORS, which can be seen as the natural randomized variation on the theme of HORS. Unfortunately, verifying natural probabilistic counterparts of reachability and termination properties is in general a much more difficult task and is undecidable for PHORS, contrary to what happens for HORS and recursive Markov chains. For example, in a probabilistic setting, a natural property is \emph{almost sure termination} (AST), that is, termination with probability 1. Now, while termination is the quintessential \emph{semi-decidable} property for a Turing-complete language (and is even \emph{decidable} for HORS),  AST is a $\Pi^0_2$-complete in Turing-complete probabilistic languages \cite{DBLP:journals/acta/KaminskiKM19}, and its decidability fails for PHORS, already at order 2. For these reasons, the recent literature has focused on capturing sub-classes of PHORS for which AST and, possibly, the related \emph{positive almost sure termination} (PAST) - i.e.~the finiteness of the \emph{expected} number of steps to termination - could be shown decidable, and thus possibly amenable to verification and model-checking techniques. Notably, while \cite{DBLP:journals/lmcs/KobayashiLG20} established the decidability of AST for order-1 PHORS, Li et al.~\cite{DBLP:conf/lics/LiMO22} established the decidability of both AST and PAST for the \emph{affine} PHORS, i.e.~the recursive programs which are allowed to use each of their inputs \emph{at most} once during their reductions.

Despite the aforementioned results, the nature of the problem remains poorly understood, and a precise assessment of its computational difficulty is still lacking. For instance, it is currently unknown whether the decidability problem for PHORS remains $\Pi^0_2$-complete, or whether it can, while being undercidable, be placed lower in the arithmetical hierarchy. Similarly, the fragments of the PHORS language for which decidability results are known are incomparable, and no class of PHORS encompassing all of them has been identified. We therefore believe that it is important to investigate the nature of this problem in greater depth, which is the goal of this paper. In the remainder of this introduction, we present a new perspective that will allow us to shed fresh light on the problem.


%\begin{itemize}
%\item some well-known facts about deterministic HORS $\to$ decidability, model-checking
%
%\item probabilistic case $\to$ decidability of AST fails already at order 2
%
%\item the difficulty lies in computing or estimating the probability of termination, as this requires to \emph{count} over infinitely many reductions, even though these may exhibit a somehow finitary pattern
%
%\item Known results: decidability for \emph{affine} and \emph{order 1} PHORS, but undecidability already at order 2
%\end{itemize}


\paragraph*{Probabilistic Termination via Generating Functions}

A fundamental issue underlying the difficulty of estimating the probability of termination of higher-order programs is that this requires to \emph{count} probabilities over a typically infinite set of reductions, even though such reductions might all follow a somehow finitary pattern. 

Such counting problems are common in the field of combinatorics, in which they are addressed by studying the algebraic and analytic properties of the corresponding \emph{generating functions}, i.e.~of a formal power series that one can naturally associate with the sequence of numbers that one wishes to compute or estimate.   
A famous example is provided by the well-known \emph{Catalan numbers} $C_n=\frac{1}{n+1}\binom{2n}{n}$, where $C_n$ is the number of labeled binary trees with $n$ nodes. 
The generating function $c(x)=\sum_{n=0}^{\infty}C_n x^n$ is a power series that can be characterized as the solution of the algebraic equation $xc(x)^2-c(x)+1=0$, a property that can be used to shed light over several properties of the $C_n$, and notably to replace the infinite series by a closed (and computable) expression $c(x)=\frac{1+\sqrt{1-4x}}{2x}$.

%
%In combinatorics, these kind of situations are typical of \emph{counting sequences}, that is, of sequences $(a_n)_{n\in \BB N}$ that represent the cardinality $a_n$ of a certain class of entities which can be constructed out of $n$ basic elements (e.g.~the famous 
%A standard and powerful approach to study counting sequences $(a_n)_{n\in \BB N}$ is by investigating the analytic properties of the corresponding \emph{generating functions} $F(x)=\sum_{n=0}^\infty a_n x^n$, formal power series defining analytic functions over some open subset of the complex numbers.  

Starting from Chomsky and Sch\"utzenberger's (\cite{CHOMSKY1963118}), generating functions have been widely applied in formal language theory. For example, it is well-known that for all regular languages $L$, the associated generating function $L(x)=\sum_{n=0 }^{\infty}L_nx^n$, where $L_n$ counts the words of $L$ of length $n$, is \emph{rational}, that is, it can be written as a fraction $L(x)=p(x)/q(x)$ of two polynomials.
%
%
% defined by linear recurrence equations (like e.g.~the Fibonacci numbers $f_{n+2}=f_n+f_{n+1}$), and the corresponding generating function $\widehat L(x)=\sum_{n=0}^{\infty}L_nx^n$ is thus \emph{rational}, i.e.~a fraction $\frac{\mathscr{p}(x)}{\mathscr{q}(x)}$ of two polynomials.
When $L$ is context-free, instead, $L(x)$ is \emph{algebraic}, that is, is the solution, like $c(x)$ above, of some polynomial equation $p(x,L(x))=0$. 
%Finally, when $L$ is generated by an indexed grammar (cite), a class of languages sitting in between the context-free and the context-sensitive languages (and equivalent to the order-2 HORS), $L(x)$ may fail to be algebraic
%$y=L$ of some polynomial equation of the form $\mathscr p(x,y(x))=0$ (while this property fails for \emph{indexed grammars}, a class of grammars lying in the middle between the context-free and the context-sensitive ones, and equivalent to the order-2 HORS \cite{}).

%Knowing that the generating function of some language is rational or algebraic allows one to deduce several interesting properties about the language, like asymptotic estimations for the growth of their counting sequences or disproving 
%
%to providing, in some cases, closed forms for the corresponding generating functions, and thus exact methods for 

Now, suppose $\gphors$ is a PHORS, i.e.~a higher-order recursive probabilistic program, and that $\gphors$ makes, at each of its reduction steps, an unbiased choice between two different alternatives. 
Letting $\gphors_n$ be the number of \emph{distinct} terminating reductions of $\gphors$ with exactly $n$ steps, the power series $a_\gphors(z)=\sum_{n=0}^{\infty}\frac{\gphors_n}{2^n}z^n$ captures the probabilistic behavior of $\gphors$. Notably, $a_\gphors(1)$ precisely computes its probability of termination, and, as we'll see, the derivative $a_{\gphors}'(1)$ precisely captures the \emph{expected} number of reduction steps to termination.
 
Could we study, then, the complexity of the termination problem for a PHORS $\gphors$ by studying the algebraic or analytic properties of its generating function $a_\gphors(z)$?




%
% Indeed, given some probabilistic program $M$ of ground type, one can consider the sequence $(t_n)_{n\in \mathbb N}$ counting the number of distinct reductions of $M$ to termination making $n$ reduction steps; under the simplifying assumption that $M$ makes an unbiased probabilistic choice at each reduction step, considering the generating function $F(x)=\sum_{n=0}^{\infty}\frac{t_n}{2^n}x^n$, the value $F(1)$ precisely captures the probability of termination of $M$.
%Knowing whether $F(x)$ is rational or algebraic can then lead to methods to compute the value $F(1)$ explicitly, or in any case to obtain relevant information about it.
%

%
%\begin{itemize}
%
%\item probability of termination corresponds to a counting problem: for given $n$, how many terminating reductions with $n$ unfoldings?
%Calling $a_n$ this number, then $P(M\Downarrow)=\sum_n a_n\frac{1}{2^n}$.
%
%\item manipulating power series $\to$ theory of generating functions
%
%\item well-studied for counting problems: for a language $L$, let $L_n=\sharp\{\text{words $w\in L$ of length }n\}$, then
%\begin{itemize}
%\item for $L$ regular, $L_n$ is C-finite (i.e.~satisfies a linear recurrence equation, like Fibonacci numbers $F_{n+2}=F_{n+1}+F_n$)
%\item for $L$ context-free, $L_n$ is algebraic (i.e.~the function $L(x)=\sum_n a_nL^n$ is zero of a polynomial $p(x,L(x))=0$);
%\item for $L$ indexed grammar (equivalent to order 2) $L_n$ may be not algebraic.
%\end{itemize}
%
%
%\item remark that if the termination series $a(x)=P(M\Downarrow)=\sum_n a_nx^n$ is algebraic (i.e.~$p(x,a(x))=0$), then AST is decidable via the first-order theory of the reals. 
%
%\item Yet, all this is restricted to grammars and first-order languages, and HORS theory shows that climbing up to higher-orders makes the complexity of the underlying problems increase.
%\end{itemize}

\paragraph*{Generating Functions via Linear Logic}

Generating functions have been applied in the literature both for the study of
formal grammars \cite{DBLP:books/sp/KuichS86} and 
 \emph{first-order} programming languages (also in the probabilistic case, cite people \warn).  However, extending this natural and powerful approach to functional programs is challenging, as it is not obvious how to associate \emph{higher-order} functions  with sequences of real numbers, as required to induce a generating function. 


In this work we provide a solution to this problem, by presenting the first combinatorial study of the termination problem for {higher-order} recursive programs like the PHORS.
To do this, we exploit in an essential way two ideas coming from the toolbox of linear logic.
The first fundamental ingredient is the \emph{weighted relational model} \cite{DBLP:conf/lics/LairdMMP13}, which associates proofs in linear logic, as well as PCF programs, with families of \emph{formal power series} with coefficients in a continuous semiring.
We show how this model can be used to associate each PHORS with the corresponding generating function $a_\gphors(z)$ in an elegant and compositional way.

As mentioned before, linearity has been recognized as a key factor to ensure the decidability of the AST problem for PHORS (\cite{DBLP:conf/lics/LiMO22}).
The viewpoint of generating functions provides then a \emph{novel} way to look at this result:
using our interpretation via the weighted relational model, we will show that for all affine PHORS, while the corresponding language might \emph{not} be context-free, the corresponding generating function is, in fact, always algebraic.


 Actually, the combinatorial viewpoint allows us to go beyond the linear/affine restriction and consider even \emph{non-linear} PHORS typable via the discipline of \emph{graded exponentials} (cite \warn): this discipline allows one to define types of the form $!_n A \multimap B$, intuitively expressing functions from $A$ to $B$ using their inputs \emph{at most} $n$ times. 
 
 Hence, on the one hand, the combinatorial approach leads us to capture a \emph{larger} class of PHORS than \cite{DBLP:conf/lics/LiMO22}, including PHORS admitting both bounded and unbounded forms of duplication, for which the AST and PAST problems are decidable.
At the same time, it also suggests the \emph{robustness} of the class of PHORS individuated by \cite{DBLP:conf/lics/LiMO22}, as all PHORS typable in our discipline will be shown 
 \emph{equivalent} (in terms of both probabilistic termination and the corresponding language) to some affine PHORS, even though the latter may have size more than exponential compared to the original, non-linear, one.


%
%
%
%our approach provides a \emph{new} method, grounded on algebraic combinatorics, to understand (and to re-establish) the decidability of AST/PAST for affine PHORS; on the other hand, it
% 
% each such non-linear PHORS will be shownHence, on the one hand, 

%
%
% our approach naturally leads to consider finitary exponentials, as expressed by linear types of the form $!_n A \multimap B$, intuitively expressing functions from $A$ to $B$ using their inputs \emph{at most} $n$ times.
%
%
% 
%
%, which interprets proofs in linear logic (as well as $\lambda$-terms) with matrices valued over a continuous semiring. Following a folklore ideathis model can be formulated 
%
%The fundamental tool that  such languages with the theory of generating functions 
%
%
%Notably, leveraging a well-studied denotational semantics coming from linear logic, the \emph{weighted relational semantics} \cite{}, we show that any PHORS can be associated with a formal power series whose coefficients are defined by a countable system of polynomial equations. We then explore different type disciplines capturing classes of PHORS whose corresponding power series are algebraic, and we deduce from this the decidability of their associated AST and PAST problems.
%
%While our approach provides a new method, grounded on algebraic combinatorics, to re-establish the decidability of order-1 and affine PHORS, it also encompasses programs which are \emph{both} non-linear programs and of arbitrary order.
%
%Beyond the relational semantics, that, following \cite{}, we formulate in terms of formal power series, a second important ingredient that comes from linear logic is the theory of \emph{graded linear comonads:} while linearity (or, better, affinity) had already been recognized as a key factor to ensure decidability of probabilistic termination, our approach naturally leads to consider finitary exponentials, as expressed by linear types of the form $!_n A \multimap B$, intuitively expressing functions from $A$ to $B$ using their inputs \emph{at most} $n$ times.


%
%\begin{itemize}
%
%\item\textbf{Take home sentence:} 
%the main goal of this work is to show that the theory of generating functions can be applied to the study of the termination problem for higher-order probabilistic programs. 
%Using a well-known semantics from linear logic, the weighted relational model, we show that each PHORS can be associated with a formal power series whose coefficients are defined by a countable system of polynomial equations. 
%
%
%
%
%
%
%
%
%%\item it is a folklore fact that weighted Rel interprets higher-order terms as formal power series over a continuous semi-ring.
%
%\item example: 
%$$\lambda f.\mathrm{fix}(\lambda x.fx\oplus \mathrm{True}):(\mathtt{unit}\multimap \mathtt{unit})\To \mathtt{unit}$$
%has all reduction traces $\lambda f.f^n\mathrm{True}$ (each with probability $\frac{1}{2^n}$), and 
% yields the generating function $G(f)=\sum_n \frac{1}{2^n}f^n$ with closed form $G(f)=\frac{1}{1-\frac{f}{2}}$.
% 
% 
%
%\item while such power series are not computable in general, we use the theory of generating functions to capture \emph{decidable} fragments of this semantics. 
%
%
%
%\item Notably, we capture a class of PHORS (of arbitrary order) whose termination is expressed by an algebraic power series, and for which the AST problem is thus decidable. This class comprises and extends previously studied classes of decidable PHORS, like order 1 and affine PHORS.
%
%
%\item two key ingredients:
%	\begin{enumerate}
%	\item \emph{the (weighted) relational semantics}, a well-known model of linear logic and lambda-calculus associating proofs/terms with formal power series valued over a continuous semi-ring. 
%
%	\item \emph{graded linear comonads:} while linearity (or, better, affinity) had already been recognized as a key factor to ensure decidability of probabilistic termination, our approach naturally extends to finitary exponentials, as expressed by linear types of the form $!_n A \multimap B$, intuitively expressing functions from $A$ to $B$ using their inputs \emph{at most} $n$ times.
%	
%	
%	\end{enumerate}
%
%
%
%\end{itemize}
%

\paragraph*{Contributions}
Our contributions can be resumed as follows:
\begin{varitemize}

\item First, we show that the weighted relational semantics naturally associates each PHORS with generating functions capturing probability of termination and expected number of steps to termination, whose coefficients are implicitly defined via a countable system of equations. 

\item Then, we introduce the \emph{finitely bounded PHORS}, $\Gfin$, extending the affine PHORS of \cite{DBLP:conf/lics/LiMO22} with finitely bounded non-linearity, and show that the corresponding generating functions are algebraic, leading to the decidability of AST and PAST.

\item Finally, we introduce the \emph{infinitely bounded PHORS}, $\Ginf$, which extend the $\Gfin$ with a restricted use of \emph{unbounded} non-linearity, notably enabling the \emph{composition} of different PHORS. We show that this extension preserves both algebraicity and the decidability of AST and PAST.
% 
%e then introduce \emph{finitely graded PHORS}, $\Gfin$, extending the affine PHORS of (Ong) with bounded non-linearity, and show that the corresponding generating functions are algebraic, leading to the decidability of AST and PAST.
%
%a first type system with \emph{finite} grades and fixpoints (indeed a non-linear extension of the type system of Li et al.~\cite{}), and show that the PHORS typable in this system are interpreted by power series that are algebraic over $\mathbb Q$ (yielding the decidability of the corresponding AST and PAST problems).
%%Via well-known linearization techniques, each such typable PHORS can be shown \emph{equivalent} to an affine PHORS in the sense of [Ong], yet the latter may have size more than exponential with respect to the former.
%%% While this system types possibly non-linear PHORS of arbitrary order, it types \emph{all} order1 and affine PHORS.
%
%\item We then explore an extension of the first type system which enables the introduction of \emph{infinite} grades in a controlled way, 
% allowing us to capture even algebraic PHORS which may use their inputs an \emph{unbounded} number of times. This extension relies on a \emph{parameterization} method for the weighted relational semantics which we think may lead to further results in the future. 


\end{varitemize}
