\section{Proofs of section 4}
We prove this by the connection between the relational model and non idempotent type systems.
Take a standard non idempotent intersection intersection type system, for example:
$$
\infer[app]{\biguplus \Gamma_i \uplus \Delta \vdash ts: \tau}{ \Gamma_i \vdash s: b_i \; (i \in \{1, \dots n\})\quad \Delta \vdash t:[b_1, \dots b_n] \to \tau}
\qquad
\infer[\lambda]{\Gamma\vdash \lambda x. t: \sigma \to \tau}{\Gamma, x: \sigma \vdash t:\tau}
$$
$$\infer[fix]{\biguplus \Gamma_i \uplus \Delta \vdash YM: \tau}{\Gamma_i \vdash YM: b_i \; (i \in \{1, \dots n\})\qquad \Delta \vdash M: [b_1, \dots, b_n] \to \tau}$$
Then we obtain:
\begin{lemma}
	\label{lambdaYlinear}
	Suppose that we have a $\lambda$Y term $t$ with $FV(t)=\{x_1, \dots x_n\}$ and that $x_1:\1 \dots x_n:\1 \vdash_{\lambda Y} t: \1 $. Then, for any derivation in this intersection type system with conclusion  $ x_1: \sigma_1 \dots  x_n: \sigma_n \vdash t:  \1 $, one of the following is true:
	\begin{enumerate}
		\item For every $i$, $\sigma_i = \emptyset$
		\item there exists one $i$ such that $\sigma_i=[\1]$ and for each $j \neq i$, $\sigma_j=\emptyset$
	\end{enumerate}
\end{lemma}
\begin{proof} 
	If we have a typing of $ x_1: \sigma_1 \dots  x_n: \sigma_n \vdash t:  \1 $, $t$ has an head normal form, then take any head normal form $h$ of $t$ (which must satisfy  $ x_1: \sigma_1 \dots  x_n: \sigma_n \vdash h:  \1 $, hence also $ x_1:\1 \dots x_n:\1\vdash h:\1$) . Then either $L = x$ or $L=e$. In the first case, we can only derive $x_i:[1] \vdash x_i:1$, in the second we just have $x_i:\emptyset \vdash e:1$\\
\end{proof}
From this we can deduce:
\begin{proposition}\label{prop:affine}
	For all first-order terms $x_1:\1,\dots, x_n:\1\vdash_{\PLY}t:\1$, 
	$\model{t}^{\Rinf}(x_1,\dots, x_n)\in \fps{\Rinf}{x_1,\dots,x_n}$ is affine: there exists scalars $w_0,w_1,\dots, w_n\in\mathbb R_{\geq 0}$ such that
	\[
	\model{t}^{\Rsemiring}(x_1,\dots, x_n)
	=w_0+w_1x_1+\dots+w_nx_n,
	\]
	where $w_0=\mathbb P[t\to^* e]$ and
	$w_{i+1}=\mathbb P[t\to^* x_{i+1}]$.
\end{proposition}
\begin{proof}
	By the soundness of the weighted relational model with respect to intersection type (\cite{DBLP:journals/jacm/EhrhardPT18}, \cite{DBLP:journals/corr/abs-2501-15637}), the monomials $x_0^{i_0}, \dots x_n^{i_n}$ that appear with positive coefficient in $\model{t}^{\Rinf}(x_1,\dots, x_n)\in \fps{\Rinf}{x_1,\dots,x_n}$ correspond to derivations $x_1: [\1^{i_1}] \dots x_n: [\1^{i_n}] \vdash t: \1$. 
\end{proof}
\section{Proofs of section 6}
\subsection{Proof of theorem 6.2}
 We now define in a precise way the notion of linearization of a $\Gfin$. We remind that, following the terminology of \cite{DBLP:conf/lics/LiMO22}, by 'linear' here we mean what in a strict sense would be 'affine'.  First we define the linearisation of a type:
\begin{align*}
	&\Aff(\1^n) =\1 \quad n \geq 1\\
	&\Aff({!_0 \phi \multimap \psi}) = \Aff(\phi) \multimap \Aff(\psi) \\  
	&\Aff({!_k \phi \multimap \psi}) = \underbrace{\Aff(\phi) \multimap \dots \multimap \Aff(\phi)}_{k \textit{ times}} \multimap \Aff(\psi)
\end{align*}
We need now to define the linearisation of a the type derivation. In particular, to each derivation $\pi = \nonterm \mid \Delta \vdash \phi$ in $\Gfin$ we will associate a linear derivation $\Aff(\pi)$ of the form $\nonterm' \mid \Aff(\Delta) \vdash \phi$, where  $\nonterm'$ is obtained by $\nonterm $ by replacing every type binding $L:\varphi $ with $\AFF L: \Aff(\varphi)$ and $\Aff(\Delta)$ is obtained by $\Delta$ by replacing every type binding $x:_k \psi, \: k \geq 2$ with $k$ distinct bindings $x^{(1)}:\Aff(\psi) \dots x^{(k)}:\Aff(\psi)$
\begin{itemize}
	\item For $\pi = \infer[ax1]{\nonterm\mid \Delta\vdash x:\varphi }{x:_p\varphi\in \Delta,\quad 1 \leq p}$, $\Aff(\pi) \bydef \pi$ 
	\item For  $\pi = \infer[ax2]{\nonterm\mid\Delta \vdash L:\varphi }{ L:\varphi  \in \nonterm}$, $\Aff(\pi) \bydef \infer[ax2]{\nonterm' \mid\Delta \vdash \AFF L:\Aff(\varphi) }{ \AFF L:\Aff(\varphi)  \in \nonterm'} $ 
	\item If we have that $\pi$ ends with an application $\nonterm\mid \Delta + k\Delta' \vdash tu : \psi$ with premises $\nonterm\mid\Delta \vdash t :!_k\varphi \multimap \psi$ and $\nonterm\mid \Delta' \vdash u :\varphi$, suppose that we get inductively linearisations $\nonterm'\mid \Aff(\Delta) \vdash \Aff(t): \Aff(\varphi) \multimap \dots \Aff(\varphi) \multimap \Aff(\psi))$ (where $\Aff(\varphi)$ is repeated $k$ times)  and  $\nonterm' \mid \Aff(\Delta') \vdash \Aff(u) :\Aff(\varphi)$. Now we create $k$ different copies $u_1, \dots u_k$ of $u$ by replacing its free variables $x_1, \dots x_n$ with variables $x^{(i)}_1, \dots x^{(i)}_n, 1 \leq i \leq k$ and we create in the same way $k$ different copies of the derivation $\nonterm' \mid \Aff(\Delta') \vdash \Aff(u) :\Aff(\varphi)$. Now by $k$ linear applications we can type $\Aff(t) \Aff(u_1) \dots \Aff(u_k)$ with type $\Aff(\psi)$; this is the linearisation of $tu$
	\item If we have $\pi$ ends with an abstraction $\nonterm\mid \Delta \vdash \lambda x. t : !_k \varphi \multimap \psi$ with premises $\nonterm \mid \Delta, x:_k \varphi \vdash t : \psi$, then take its linearisation $\nonterm \mid \Aff(\Delta),x^{(1)}:\Aff(\varphi) \dots x^{(k)}:\Aff(\varphi) \vdash \Aff(t) : \Aff(\psi)$ and; by $k$ abstraction we then obtain a linear derivation of $\lambda x^{(1)} \dots x^{(k)}.\Aff(t): \Aff(\phi) \multimap \dots \Aff(\phi) \multimap \Aff(\psi))$,  (where $\Aff(\phi)$ is repeated $k$ times).
	\item In the case of $\pi_i$ and $\langle \rangle$, we can just apply them to their linearised premises and obtain a linearised conclusion.
\end{itemize} 
Now given a $ \gphors=(\nonterm, \C R, S)$, we define its linearisation $\AFF\gphors=(\nonterm, \C R',\AFF S)$ by $\C R'(L) \bydef \Aff(\C R(L))$ for each non terminal $L$. In particular, $\AFF{S}$ will have equation $\AFF S= \Aff(\C R(S))$ Call $\AFF{\redbigp{ }}$ the corresponding rewriting\\
To see that $\AFF \gphors$ generates the same tree as $\gphors$, we need the following lemmas:
\begin{lemma}
	Let $\nonterm \mid \Delta, x:_k  \vdash  t[x]$ be a term typable in $\Gfin$ and let $\nonterm' \mid \Aff(\Delta), x^{(1)}: \Aff(\varphi), \dots x^{(k)}: \Aff(\varphi) \vdash  \Aff t[x^{(1)}, \dots x^{(k)}]$ be its linearisation. Let $u$ be a term of type $\varphi$ and let $\Aff(u)$ be its linearisation; then, up to a renaming of the free variables:
	$$\Aff(t[x/u]) = \Aff(t)[x^{(1)}/\Aff(u)_1, \dots x^{(k)}/\Aff(u)_k] $$ 
	where $\Aff(u)_1 \dots \Aff(u)_k$ are obtained from $\Aff(u)$ by replacing its free variables with fresh ones.
\end{lemma}
\begin{proof}
	By induction on the definition of linearisation.
\end{proof}
Then we have
\begin{lemma} For each non terminal $L$ and $d \in \{l,r\}$
	$L t_1 \dots t_n  \redbigp{\mathsf d,p} t_L[x_1/t_1, \dots x_n/t_n]$ iff $\Aff(L t_1 \dots t_n)  \redbigp{\mathsf d,p} \Aff(t_d[x_1/t_1, \dots x_n/t_n])$
\end{lemma}
\begin{proof}
	We have that $\Aff(Lt_1 \dots t_n)= \AFF L \Aff(t_1)^{n_1} \dots \Aff(t_k)^{n_k} \AFF{\redbigp{\mathsf d,p}} \Aff(t_d)[x_1^{n_1}/(t_1)^{n_1}, \dots x_1^{n_1}/(t_k)^{n_k}]$, but this by the previous lemma is equal (all terms being closed, no renaming of the free variables is needed) to $\Aff(t_d[x_1/t_1, \dots x_n/t_n])$
\end{proof}  
Finally, notice that $e$ is the only normal form both for $\gphors$ and $\AFF \gphors$ and that the linearisation of $e$ is $e$. From this we see that there exists a bisimulation between the rewriting system of $\gphors$ and the one of $\AFF \gphors$, hence they generate the same tree.
\subsection{Proof of proposition 6.3}
To prove proposition 6.3, we prove first the following correctness lemma:
\begin{lemma}[Correctness of $\Gfin$ for ground terms]
	\label{lemma:correctness}
	Given a term $\nonterm\mid\Delta\vdash t: \1$, let $s \bydef \model{t: \1}_{}^{\Rsemiring} $ be its $\Qrelkleisli \Rsemiring$interpretation, which consists of a single power series. Then, $s_\sigma\big\vert_{\model{\nonterm}+ \model{\Delta}}$ is a polynomial. Moreover, for each $f:_k \psi \in \Delta$, the total degree of $s$ in the variables $x_{f:_k \psi}$ is at most $k$
\end{lemma}
We recall that given a polynomial $ P = \sum_\kappa \in !S a_\kappa x_S^\kappa$, and a subset $S' \subset S$, the total degree of $P$ in the variables $x_{S'}$ is $\max_{\kappa \mid a_\kappa \neq 0} (\sum_{i \in S'} \mu(i))$.
We don't prove directly the last lemma, as the induction hypothesis would be not sufficient, but we pass through an intermediate step that reminds of reducibility candidates.
\begin{definition}
	For each type $\varphi \tri T$, we define by induction a set ${\Com}(\varphi)$ such that $\model{\varphi} \subseteq \Com(\phi) \subseteq \model{T}$:
	\begin{enumerate}
		\item $\Com(\1^n)= \model{\1^n}$
		\item $\Com( !_k \psi \multimap \varphi \tri S \to T) \bydef \{(\mu, f) \in !\model{S} \times \model{T}\mid \supp \mu \subseteq \Com(\psi) \implies (|\mu| \leq k \land \Com(f))\}$
	\end{enumerate}
When $\Delta= f:_{k_1} \psi_1, \dots f:_{k_n} \psi_n$ is a context, $\Com(\Delta) \bydef \{(p_1, \dots p_n)\mid \forall i \: p_i \in \Com(\psi_i)\}$.
\end{definition}
Now we can prove by induction the following lemma:
\begin{lemma}
	Given a term $\nonterm\mid\Delta\vdash t: \1$, with $\varphi\tri T$, let $s_{\model{T}} \bydef \model{t: T}^{\Rsemiring} $ be its $\Qrelkleisli \Rsemiring$interpretation. We have that:
	\begin{enumerate}
		\item For each $\sigma \in \model{T} \setminus \Com(\model{\varphi})$, 
		\begin{equation}
			s_\sigma\big\vert_{\model{\nonterm}+ \Com(\model{\Delta})}=0
		\end{equation}
		\item For each $\sigma \in \Com(\model{\varphi})$, $s_\sigma\big\vert_{\model{\nonterm}+ \Com(\model{\Delta})} \in \fpp{\Rsemiring}{\model{\nonterm}+ \Com(\model{\Delta})}$. Moreover, for each $f:_k \psi \in \Delta$, the total degree of $s_\sigma$ in the variables $x_{f:_k \psi}$ is at most $k$
	\end{enumerate}
\end{lemma}
	\begin{proof}
		By induction on the type derivation $\nonterm\mid\Delta\vdash t:\varphi$. We immediately see this is true for the axioms. We discuss here the crucial cases: application and abstraction.\\ 
		Application: suppose we know that the thesis is true for $\nonterm\mid\Delta\vdash t:!_k \psi \multimap \varphi \tri T \to S$ and $\nonterm\mid\Delta'\vdash u: \psi \tri T$. We want to prove that it is true for $\nonterm\mid\Delta+ k\Delta'\vdash (tu): \varphi \tri T$ Let $s_{\model{S}} \bydef \model{tu: S}^\Rsemiring$ We know that by definition:
		$$s_\sigma\big\vert_{\model{\nonterm}+ \Com(\model{\Delta}+ \model{\Delta'})} = \left( \sum_{\mu \in !{\model{T}}} \model{t}^\Rsemiring_{\mu, \sigma} (\model{u}^\Rsemiring)^\mu \right) \Big\vert_{\model{\nonterm}+ \Com(\model{\Delta}+ \model{\Delta'})} $$

	When we restrict the RHS to variables from $\model{\nonterm}+ \model{\Delta}+ \model{\Delta'}$, we obtain:
	\begin{equation}
		\label{eq:sumrestrict}
	\sum_{\mu \in !{\model{T}}} \model{t}^\Rsemiring_{\mu, \sigma} \Big\vert_{\model{\nonterm}+ \Com(\model{\Delta})} (\model{u}^\Rsemiring)^\mu \Big\vert_{\model{\nonterm}+ \Com(\model{\Delta'})}
	\end{equation}
	If $\sigma \in \model{T} \setminus \Com(\model{\sigma})$, then we have three cases:
	\begin{enumerate}
		\item If $\supp \mu \not \subseteq \Com(\psi)$, then in the product $(\model{u}^\Rsemiring)^\mu$ there is a factor $(\model{u}^\Rsemiring)_m$ with $m \not \in  \Com(\psi)$; its restriction is then $0$ by inductive hypothesis and so the whole product is $0$
		\item If $\supp \mu \subseteq \Com(\psi)$, then $(\mu, \sigma) \not \in \Com(!_k \psi \multimap \varphi)$. Hence, by inductive hypothesis $\model{t}^\Rsemiring_{\mu, \sigma}=0$ and the product is $0$
	\end{enumerate}
	So we deduce that the restriction of $s_\sigma$ vanishes.\\
	Now we prove part (2): let $\sigma \in \model{!_k\psi \multimap \varphi}$. By inductive hypothesis, for all $\mu$ such $|\mu|>k$ we have by induction hypothesis (1) that $\model{t}^\Rsemiring_{\mu, \sigma} \Big\vert_{\model{\nonterm}+ \Com(\model{\Delta})}=0$, hence \ref{eq:sumrestrict} is a finite sum, each term being a polynomial by induction hypothesis (2); we conclude that the sum is a polynomial. Now, let $f:_k \xi \in \Delta + k\Delta'$; assume that $f:_n \xi \in \Com(\model{\Delta})$ and $f_m: \xi \in \Com(\model{\Delta'})$ and let $y$ be a variables in $x_{f:_k \xi}$. We have that the maximal degree of each $\model{t}^\Rsemiring_{\mu, \sigma} \Big\vert_{\model{\nonterm}+ \model{\Delta}}$ in $y$ is (by inductive hypothesis) at most $n$ and the degree of each   $(\model{u}^\Rsemiring)\Big\vert_{\model{\nonterm}+ \Com(\model{\Delta})}$ in $y$ is at most $m$. Since for all $\mu$ such $|\mu|>k$ we have  that $\model{t}^\Rsemiring_{\mu, \sigma} \Big\vert_{\model{\nonterm}+ \model{\Delta}}=0$, the maximal degree of a non zero product $\model{t}^\Rsemiring_{\mu, \sigma} \Big\vert_{\model{\nonterm}+ \Com(\model{\Delta})} (\model{u}^\Rsemiring)^\mu \Big\vert_{\model{\nonterm}+ \Com(\model{\Delta'})}$ is $n+km$.\\
	Abstraction: suppose we know that the thesis is true for $\nonterm\mid\Delta, x_k: \psi \vdash t: \varphi \tri T$, with $\psi \tri S$. We want to prove that it is true for $\nonterm\mid\Delta \vdash \lambda x. t: !_k \psi \multimap \varphi \tri S \to T$. If we write $\model{t: T}^\Rsemiring_t$, $t \in \model{T}$ as a fps $s_t=\sum_{\kappa \in !{S}} r_{\kappa,t} (x_{{x:S}})^\kappa$ (with $r_\kappa \in \fps{\Rsemiring}{\nonterm+\Delta} $), we know by definition that 
	$$\model{ \lambda x. t:  S \to T }^\Rsemiring_{\mu, t} \Big\vert_{\model{\nonterm}+ \Com(\model{\Delta})} = r_{\mu, t} \Big\vert_{\model{\nonterm}+ \Com(\model{\Delta})} $$
	Take $(\mu, t) \not \in \Com(!_k\model{\psi} \multimap \model{\varphi})$. This means that $\supp \mu \subseteq \Com(\psi)$, but either $t \not \in \Com(\phi)$ or $|\mu| > k$. In the first case, we know $s_t=0$. Since $\supp \mu \subseteq \Com(\phi)$, then $x_{x:S}\vert_{\model{\nonterm}+ \Com(\model{\Delta})^\mu \neq 0}$ (as none of its variables with positive degree is equated to $0$ by this restriction), and so we must have also $r_{\mu, t} \Big\vert_{\model{\nonterm}+ \Com(\model{\Delta})}=0$.  
	Now assume that $t \in \Com(\phi)$ and $\|\mu| > k$. By inductive hypothesis, we know that $s\vert_{\model{\nonterm}+ \Com(\model{\Delta})}$ is a polynomial of total degree at most $k$ in the variables $x_{x:S}$: hence $r_{\mu, t} \Big\vert_{\model{\nonterm}+ \model{\Delta}}=0$ for $\supp \mu \subseteq \Com(\psi)$ and $|\mu| > k$.
	 $r_{\mu, t} \Big\vert_{\model{\nonterm}+ \Com(\model{\Delta})}=0$. Point (2) is in this case obvious. 
\end{proof} 
From the previous lemma, lemma \ref{lemma:correctness} is obvious by taking $\varphi=\1$  and observing that if $s\Big\vert_{\model{\nonterm}+ \Com(\model{\Delta})}=0$, then  $s\Big\vert_{\model{\nonterm}+ \model{\Delta}}=0$ (as $\Delta \subseteq \Com(\Delta)$)
From this we obtain the following lemma about terms $\lambda x_1 \dots x_n.t: !_{k_1} \psi_1 \multimap \dots !_{k_n} \psi_n \multimap \1 $:
\begin{lemma}
	Given a term $$\nonterm\mid\emptyset \vdash : \lambda x_1 \dots x_n.t: !_{k_1} \psi_1 \multimap \dots !_{k_1} \psi_n \multimap \1$$ with $!_{k_n} \psi_1 \multimap \dots !_{k_1} \psi_n \multimap \1 \tri T$, let $s_{\model{T}} \bydef \model{t: T}^{\Rsemiring} $ be its $\Qrelkleisli \Rsemiring$interpretation. Then:
	\begin{enumerate}
		\item If $\sigma \not \in \model{T} \setminus \model{!_{k_1} \psi_1 \multimap \dots !_{k_1} \psi_n \multimap \1}$, then $s_\sigma\Big\vert_{\model{\nonterm}}=0$
		\item If  $\sigma \in \model{T} \setminus \model{!_{k_1} \psi_1 \multimap \dots !_{k_1} \psi_n \multimap \1}$, then $s \Big\vert_{\model{\nonterm}}$ is a polynomial. 
	\end{enumerate}
\end{lemma}
\begin{proof}
	By lemma \ref{lemma:correctness}, the interpretation of $\nonterm\mid x_1:_{k_1} \dots x_n:_{k_n} \vdash : t: \1$ is a polynomial of total degree at most $k_i$ in the variables $x_{x_i:_{k_i}: \psi_i}$
\end{proof}
\section{Proofs of section 7}
\subsection{Proof of Lemma \ref{lemma:composition}}
Here, we prove that what has been defined in Definition 7.2 as the composition of two $\Ginf$ really deserves this name.
Let  $\Ginf$ $\gphors=(\nonterm_1,\C P ,\C R_1,S_1)$ be a $\Ginf$. Recall that its $\Qrelkleisli{\Rsemiring}$ interpretation $\model{\gphors}$ is then an arrow in $\Qrelkleisli{\Rsemiring}(\model{P}, \model{N})$, which is a family $s_{\model{\nonterm}} \in \fps{\Rsemiring}{\model{\C P}}$ of fps.
\begin{lemma}
	Let  $\Ginf$ $\gphors_1=(\nonterm_1,\C P\cup\{w:_\infty\varphi),\C R_1,S_1)$ and
	$\gphors_2=(\nonterm_2,\C P,\C R_2,S_2)$ be two $\Ginf$ with $\nonterm_1 \cap \nonterm_2= \emptyset$ and $\nonterm_2(L) = \phi$. Let $\gphors_1(\gphors_2)$ be their composition along $L$. Then we have for each $L_1 \in \nonterm_1 $,
	$$\model{\gphors_1(\gphors_2)}_{L_1}^\Rsemiring(y_{\model{\C P}}) = \model{\gphors_1}^\Rsemiring_{\model{L_1}}(\model{\gphors_2}^\Rsemiring_{\model{L}}(y_{\model{\C P}}),y_{\model{\C P}} )$$
	and for each $L_2 \in \nonterm_2$,
	$$ \model{\gphors_1(\gphors_2)}_{L_2}^\Rsemiring(y_{\model{\C P}}) = \model{\gphors_2}_{L_2}^\Rsemiring(y_{\model{\C P}}) $$
\end{lemma}
\begin{proof}
	Let let respectively $s_{\model{\nonterm_1}}(y_\phi, y_{{\C P}})$, $r_{\model{\nonterm_2}}(y_{{\C P}})$ be the interpretations of $\gphors_1$ and $\gphors_2$. Then $s_{\model{\nonterm_1}}(y_\phi, y_{\nonterm{\C P}})$ is the minimal solution of the system $(S_1)$ of fixpoint equation:
	\begin{equation}
		\label{eq:fixg1}
		z_{\model{\nonterm_1}}= p_{\model{\nonterm_1}}(z_{\model{\nonterm_1}}, z_w,y_{\C P} )
	\end{equation}
	and  $r_{\model{\nonterm_2}}(y_{{\C P}})$  is the minimal solution of the system of fixpoint equations 
	\begin{equation}
		\label{eq:fixg2}
		z_{\model{\nonterm_2}}= q_{\model{\nonterm_2}}(z_{\model{\nonterm_1}}, y_{\C P} )
	\end{equation} 
	Then, $\model{\gphors_1(\gphors_2)}$ is the fixpoint solution of the system:
	\begin{equation}
		\label{eq:fixgcomp}
		\begin{cases}
			z_{\model{\nonterm_1}}= p_{\model{\nonterm_1}}(z_{\model{\nonterm_1}}, z_{\model{L: \phi}},y_{\C P} )\\
			z_{\model{\nonterm_2}}= q_{\model{\nonterm_2}}(z_{\model{\nonterm_1}}, y_{\C P} )
		\end{cases}
	\end{equation}
	Clearly, since $q_{\model{\nonterm_2}}$ do not involve any variable from other equations, the fixpoint iteration for \ref{eq:fixgcomp}, restricted to the variables $z_{\model{\nonterm_2}}$ will be equal to the fixpoint iterations for \ref{eq:fixg2}. Hence the minimal solution of \ref{eq:fixgcomp}, restricted to the variables $z_{\model{\nonterm_2}}$ will be  $r_{\model{\nonterm_2}}$. Let then $(t_{\model{\nonterm_1}},r_{\model{\nonterm_2}})$ be the full minimal solution of \ref{eq:fixgcomp}. We see that $t_{\model{\nonterm_1}}$ is then the a solution of $z_{\model{\nonterm_1}}= p_{\model{\nonterm_1}}(z_{\model{\nonterm_1}}, r_{\model{\nonterm_2}} ,y_{\C P} )$; it must be the minimal one, as if there were a solution $t'_{\model{\nonterm_1}} \leq t_{\model{\nonterm_1}}$, this would give a smaller solution of \ref{eq:fixgcomp}. This proves the lemma.
\end{proof}
\subsection{Proof of theorem 7.3 }
We consider a parametric bounded PHORS $\gphors=(\nonterm, \C P,\C R, S)$ and we build from it a $\Gfin$ with paramters $R(\gphors)$ . In the case that $\gphors$ is closed, from $R(\Gfin)$ we can extract a subset of nonterminals that form a (parameter free) $\Gfin$. Up to isomorphism (from a fps-theoretic viewpoint, up to commutation of the variables),we can always work in the case then non terminals have type $ !_{\infty} \psi_1 \multimap \dots\multimap  !_{\infty} \psi_j \multimap \varphi$. 
\begin{definition}
	The set $T(L)$ of substituable terms for a non terminal $L \in \nonterm, \: \nonterm(L)= !_{\infty} \psi_1 \multimap \dots\multimap  !_{\infty} \psi_j \multimap \varphi$ is defined as follows:
	$$T(L) \bydef \{[t_1, \dots t_n] \in \nonterm^* \: \mid \: \forall i \, (t_i \in \nonterm \cup \mathcal{P}\land (\nonterm \cup \mathcal{P}) t_i = \psi_i)  \}  $$
\end{definition}
Now we define the enlarged set $\nonterm'$ of non-terminals of finite graded type, that will be used in $R(\gphors)$.
\begin{definition}[Reduced nonterminals]
	Let $\nonterm'$ be defined as:
	$$\bigcup_{L \in \nonterm}\{\tilde{L}_\gamma : \varphi \mid \nonterm(L)=!_{\infty} \psi_1 \multimap \dots\multimap  !_{\infty} \psi_j \multimap \varphi, \;  \gamma \in T(L)\} $$
\end{definition}

Now we define a translation from $\Ginf$ applicative terms of finite type (i.e terms $t$ that can be typed as $\nonterm\mid\C P\mid\Delta\vdash t: \varphi $) to $\Gfin$ terms $R(t)$ that can be typed as $\nonterm' \mid \Delta \vdash R(t): \phi$. 
First we prove the following:
\begin{lemma}
	If $\nonterm\mid\C P\mid\Delta \vdash t: \tau$ is an applicative term of type $!_{\infty} \psi_{1} \multimap \tau$, then it is of the form $L t_{1} \ldots t_{k}$, where $L \in \mathcal{N}$ and $t_{i} \in \mathcal{N} \cup \mathcal{P}$.
\end{lemma}
\begin{proof}
	The only way an infinitary type can appear in $\Ginf$ applicative term is because of the axiom of a non-terminal with infinitary type. The only rule we can apply after this kinf of axiom is a sequence of $@_\infty$.
\end{proof}
Then we get: 
\begin{lemma}
	For every $\Ginf$ applicative term $\nonterm\mid\C P\mid\Delta\ \vdash t: \phi$ of finite type, there exists a $\Gfin$ applicative terms $\mathcal{N} \cup \mathcal{F}\mid \mathcal{P}\mid\Delta \vdash t': \phi$ such that $t=t^{\prime}\left[f_{1} / r_{1} \ldots f_{l} / r_{l}\right]$ and each $r_{i}$ is a $\Ginf$ applicative term of the form $ L t_{1} \ldots t_{k}$, where $L \in \nonterm$ and $t_{i} \in \nonterm \cup \mathcal{P}$.
\end{lemma}
\begin{proof} By induction, if $\mathcal{N}\mid \mathcal{P}\mid \Delta \vdash t: \tau$ is an axiom of finite type, $t^{\prime}=t$. If $\mathcal{N} \mid \mathcal{P} \mid \Delta \vdash L t_1 \dots t_k, \mathcal{N}(L)= !_{\infty} \psi_1 \multimap \ldots !_{\infty} \psi_k \multimap \varphi$, then $t=f_{1}, \mathcal{F}=\left\{f_{1}:  !_{\infty} \psi_1 \multimap \ldots !_{\infty} \psi_k \multimap \varphi\right\}$ and $r_{1}=L$.\\
If $t=t_{1} t_{2}$ and the last step was a finitary application, take the terms $t_{1}^{\prime}$ and $t_{2}^{\prime}$ whose existence is recursively ensured; we can assume that the free variables $f_{1}, \ldots f_{n}$ and $f_{1}^{\prime}, \ldots f_{n^{\prime}}^{\prime}$ and the contexts $\mathcal{F}$ and $\mathcal{F}^{\prime}$ are disjoint upon renaming and define $t=t_{1}^{\prime} t_{2}^{\prime}$.\\
\end{proof}
Now we define a translation from applicative terms typable in $\Ginf$ to terms typable in $\Gfin$: 
\begin{definition}
 Let $\nonterm\mid \mathcal{P}\mid \Delta \vdash t: \varphi$ be $\Ginf$ applicative term of finite type. Then we define its translation $R(t)$ to be the following $\Gfin$ applicative term that can be typed as $\mathcal{N}^{\prime} \mid \Delta \cup \mathcal{P} \vdash R(t): \varphi$. Given $t$, take $t^{\prime}$ (typable as $\mathcal{N} \cup \mathcal{F}\mid \mathcal{P}\mid\Delta \vdash t': \phi$) as defined in the last lemma. For each $r_{i}=L t_{1} \ldots t_{n}$ as in the last lemma, define $R(r_{i}):=\tilde{L}_{\left[t_{1} \ldots t_{n}\right]} \in \nonterm'$. Then define $R(t)$ to be the result of replacing every $f_{i}$ in $t^{\prime}$ by $R\left(r_{i}\right)$. 
\end{definition}
Now we are ready to define the translation from a closed $\Ginf$ $\gphors$ to a $\Gfin$ $R(\gphors)$ having the same value tree:
\begin{definition}
Now define a new PHORS $\mathcal{G}'$ with nonterminals $\N'$ as follows: for each $L \in \nonterm$ with equation $L \vec f = t_L$ and for each $\tilde L_\gamma \in \N'$ take the following equation:
$$\tilde L_\gamma x_1 \dots x_j = R(t_L[\vec g / \gamma]) $$
\end{definition}
This defines a finitary PHORS with parameters (as some of the substituable term in $\gamma$ might be parameters). 
\begin{definition}
	We define by induction the relation of dependency between non terminals: we say that a nonterminal $L$ depends over a nonterminal $G$ if either $G$ occurs in the rewriting rule of $L$ or there exists a non terminal $G'$  that occurs in the definition $L$ and $G'$ depends over $G$ 
\end{definition}
Let $L_{[p_1, \dots p_i]}$ be a reduced non terminal; we will say that it is open if at least one of the $p_i$ is a paramter. We will show now that in the recursive unfolding of a closed non terminals, only closed non terminals appear:
\begin{lemma}
	If $\gamma$ contains free parameter $p_1, \dots p_j$ and the rewriting rule of $\tilde L_{\gamma}$
	is $\tilde L_\gamma x_1 \dots x_j \redbigp{p, d}R(t_d[\vec g / \gamma]) $, then $R(t_L[\vec g / \gamma])$ only contains non-terminals $G_\gamma'$ where the parameters appearing in $\gamma'$ are amongst $p_1, \dots p_j$. In particular, if $L_\gamma$ is closed, all non terminals occuring in $R(t_d[\vec g / \gamma]) $ are closed.
\end{lemma}
\begin{proof}
	The only free variables in $t_d$ are paramters $g_1, \dots g_n$ and the $x_1, \dots x_j$. Since the only free parameters in $\gamma$ are $p_1, \dots p_j$, $FV(t_d[\vec g / \gamma]) \subseteq \{x_1, \dots x_n, p_1, \dots p_j\}$: it does contains only $p_1, \dots p_j$ as parameters, hence any reduced nonterminal in the translation $R(t_d[\vec g / \gamma])$ will only have this parameters. 
\end{proof}
%Hence, we see that $\{L_\gamma \mid \gamma \text{contains no paramters}\}$ is closed for the dependency relation, hence we can build a PHORS that only include these non-terminals. It will in particular include $S_{[]}$. So we can solve it in the finitary system. So we have that:
%\begin{enumerate}
%	\item If we are interested on the decidability of AST/PAST we can only look at the component of the PHORS containing $\tilde{S}_{[]}$ and closed under dependency, which is a finitary PHORS without paramters.
%	\item If we are interested about the algbraicity of a non-terminal $L \vec g \vec f$, we can look at the component of the PHORS containing $\tilde{L}_{[\vec g]}$ and closed under dependency: it will be a PHORS in the paramters $g_1 \dots, g_n$, hence $\tilde{L}_{[g_1, \dots g_n]}$ will be interpreted as an algebraic power series over $g_1, \dots g_n$.
%\end{enumerate}
%Example: take the PHORS:
%\begin{equation*}
%	\begin{cases}
%		& S = L G e\\
%		& L f k = H f k \oplus f k\\
%		& G l = l \oplus \Omega\\
%		&  H g n = L g n \oplus H g n
%	\end{cases}
%\end{equation*}
%the infinitary parameters being $f, g$. We obtain reduced non terminals. $\tilde{S}, \tilde{G}, \tilde{L}_G, \tilde{L}_f, \tilde{L}_g,  \tilde{H}_G, \tilde{H}_f, \tilde{H}_g, $ and the compnent of the reduced PHORS containing $\tilde{S}$ is :
%\begin{equation*}
%	\begin{cases}
%		\tilde{S} = \tilde{L}_G e\\
%		\tilde{L}_G k = R(H G k \oplus  G K) = \tilde{H}_G k \oplus \tilde{G}_{[]} k \\
%		\tilde G_{[]} l = l \oplus \Omega\\
%		\tilde{H}_G k = R(L G n \oplus H G n) = \tilde{L}_G n \oplus \tilde{H}_G n
%	\end{cases}
%\end{equation*}
%while the component containing $\tilde{L}_f$ is:
%\begin{equation*}
%	\begin{cases}
%		\tilde{L}_f k = R( H f k \oplus  f k) = \tilde{H}_f \oplus f \\
%		\tilde{H}_f n = R(L f n \oplus H f n) = \tilde{L}_f n \oplus \tilde{H}_f n
%	\end{cases}
%\end{equation*}
