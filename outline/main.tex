%\documentclass[sigplan,nonacm,screen,review, anonymous]{acmart}
\documentclass[sigconf,anonymous]{acmart}

\AtBeginDocument{%
  \providecommand\BibTeX{{%
    Bib\TeX}}}

% The preceding line is only needed to identify funding in the first footnote. If that is unneeded, please comment it out.
%Template version as of 6/27/2024

%\usepackage[nameinlink]{cleveref}
%\usepackage{enumitem}[shortlabels]
%\makeatletter
%\AtEndPreamble{%
%  \theoremstyle{acmdefinition}
%  \@ifundefined{rem}{%
%    \newtheorem{rem}[theorem]{Remark}
%  }{}%
%  \crefname{enumi}{}{}
%  \Crefname{enumi}{}{}
%  \crefname{lemma}{Lemma}{Lemmas}
%  \Crefname{lemma}{Lemma}{Lemmas}
%  \crefname{definition}{Definition}{Definitions}
%  \Crefname{definition}{Definition}{Definitions}
%}
%\makeatother


%\usepackage{cite}
%\usepackage{amsmath,amssymb,amsfonts, amsthm}
\usepackage{algorithmic}
\usepackage{graphicx}
\usepackage{textcomp}
\usepackage{xcolor}
\usepackage{stmaryrd}
\usepackage{hyperref}
%\usepackage{MnSymbol}
%\usepackage{mathrsfs}  




\usepackage{adjustbox}
\usepackage{proof}

\usepackage{bussproofs}\EnableBpAbbreviations
\usepackage{subcaption}
\usepackage{listings}
%\usepackage[scr=boondoxo,scrscaled=1.05]{mathalfa}


\usepackage{tikz}
\usetikzlibrary{cd, angles, shapes}
\tikzcdset{scale cd/.style={every label/.append style={scale=#1},
    cells={nodes={scale=#1}}}}
    

\newtheorem{example}{Example}[section]
\newtheorem{definition}{Definition}[section]
\newtheorem{problem}{Problem}[section]
\newtheorem{notation}{Notation}[section]

\newtheorem{remark}{Remark}[section]
\newtheorem{theorem}{Theorem}[section]
\newtheorem{conjecture}{Conjecture}[section]

\newtheorem{lemma}[theorem]{Lemma}
\newtheorem{falselemma}{False lemma}[section]

\newtheorem{proposition}[theorem]{Proposition}
\newtheorem{corollary}[theorem]{Corolllary}
\newtheorem{result}{Result}
\newenvironment{varitemize}
{
	\begin{list}{\labelitemi}
		{\setlength{\itemsep}{0pt}
			\setlength{\topsep}{0pt}
			\setlength{\parsep}{0pt}
			\setlength{\partopsep}{0pt}
			\setlength{\leftmargin}{15pt}
			\setlength{\rightmargin}{0pt}
			\setlength{\itemindent}{0pt}
			\setlength{\labelsep}{5pt}
			\setlength{\labelwidth}{10pt}
	}}
	{
	\end{list} 
}


\newcounter{numberone}


\newenvironment{varenumerate}
{
	\begin{list}{\arabic{numberone}.}
		{
			\usecounter{numberone}
			\setlength{\itemsep}{0pt}
			\setlength{\topsep}{0pt}
			\setlength{\parsep}{0pt}
			\setlength{\partopsep}{0pt}
			\setlength{\leftmargin}{15pt}
			\setlength{\rightmargin}{0pt}
			\setlength{\itemindent}{0pt}
			\setlength{\labelsep}{5pt}
			\setlength{\labelwidth}{15pt}
	}}
	{
	\end{list} 
}




\newcommand{\NINF}{\BB N^\infty}

\newcommand{\RS}{\BB R_{\geq 0}^\infty}
\newcommand{\TS}{\BB{T}}
\newcommand{\BS}{\{0,1\}}


\newcommand{\db}{{\mathsf{db}}}
\newcommand{\RR}{{\mathsf{RR}}}
\newcommand{\PL}{d_{\mathsf{PL}}}
\newcommand{\LEFT}{\Big\langle}
\newcommand{\RIGHT}{\Big\rangle}
\newcommand{\BAR}{\Big\vert}


\newcommand{\NNP}{NP^{\mathrm n}}
\newcommand{\VNP}{NP^{\mathrm v}}
\newcommand{\VN}{\mathbf{VN}}
\newcommand{\trop}{\mathsf t^!}

\newcommand{\COIN}[1]{\mathrm{coin}\ {#1}}

\newcommand{\TIT}{\B{P}_{\mathrm{trop}}}
\newcommand{\IT}{\B{P}}


\newcommand{\pPCF}{$\mathrm{PCF}\langle \vec X\rangle$}
\newcommand{\pPCFn}{$\mathrm{PCF}\langle X_1,\dots, X_n\rangle$}


\newcommand{\choice}[1]{\oplus_{#1}}
\newcommand{\Bool}{\mathrm{Bool}}
\newcommand{\True}{\mathrm{True}}
\newcommand{\False}{\mathrm{False}}
\newcommand{\YY}{\mathrm{Y}}
\newcommand{\FIN}{\mathbf{F}}
\newcommand{\NAT}{\mathbf{N}}
\newcommand{\SUCC}[1]{\mathsf{succ}\ #1}
\newcommand{\PRED}[1]{\mathsf{pred}\ #1}
\newcommand{\ZZERO}{{\mathsf 0}}
\newcommand{\ZERO}{\mathsf 0}
\newcommand{\ONE}{\mathsf 1}
\newcommand{\TWO}{{\mathsf 2}}
\newcommand{\ITE}[3]{\mathsf{ifz}(#1,#2,#3)}
%{\mathrm{if}\  #1 \ \mathrm{then } \ #2 \ \mathrm{else} \ #3}
\newcommand{\RED}[1]{\stackrel{#1}{\twoheadrightarrow}}
\newcommand{\CASE}[2]{\mathrm{case}_{#1} \ \mathrm{of } \ #2 }


\newcommand{\oracle}[2]{\{ #1 \mapsto #2\} }
\newcommand{\ORed}{\longrightarrow_{\mathsf e} }
\newcommand{\BRed}{\longrightarrow_{\beta} }
\newcommand{\WBRed}{\longrightarrow_{\mathsf{wh}\beta} }
\newcommand{\HRed}{\longrightarrow_{\mathsf{h}\beta} }

\newcommand{\PRed}{\longrightarrow_{\mathsf p} }
\newcommand{\ToTRed}{\longrightarrow_{\beta\mathsf{e}} }
\newcommand{\Norm}{\mathsf{Norm} }
\newcommand{\Red}{\mathsf{Red} }
\newcommand{\modd}[1]{\llbracket#1\rrbracket }

\newcommand{\FF}[1]{\mathfrak{#1}}


%math styles

\newcommand{\too}{\twoheadrightarrow}
\newcommand{\set}[1]{\{ #1 \}}

\newcommand{\fmset}[1]{!{#1}}
\newcommand{\fps}[2]{{#1}\{\!\!\{#2\}\!\!\}}
\newcommand{\fpp}[2]{{#1}\{#2\}}

\newcommand{\flc}[2]{{#1}\{#2\}}

\newcommand{\B}[1]{\mathbf{#1}}

\newcommand{\Pto}{\rightarrowtail}
\newcommand{\Pfrom}{\leftarrowtail}
\newcommand{\perm}{\longrightarrow_{\mathsf p}}
\newcommand{\PN}[1]{\mathsf{Perm}(#1)}
\newcommand{\FN}{\mathrm{FN}}
\newcommand{\SELECT}{\mathrm{merge}\ }

\newcommand{\DIST}[1]{\mathcal D(#1)}
\newcommand{\SDIST}[1]{\mathcal D_{\leq}(#1)}

\newcommand{\PROB}{\mathbf{P}}
\newcommand{\BOX}{\mathbf C}
\newcommand{\bone}{\mathcal b}
\newcommand{\btwo}{\mathcal c}
\newcommand{\bvar}{\mathcal x}
\newcommand{\bthree}{\mathcal d}
\newcommand{\bfour}{\mathcal e}
\newcommand{\model}[1]{\modd{#1}}

\newcommand{\an}[1]{{#1}\textbf{An}}
\newcommand{\rel}[1]{{#1}\textbf{Rel}}

\newcommand{\lto}{\multimap}
\newcommand{\lam}{\lambda}
\newcommand{\law}{\mathbb{L}}


\newcommand{\TT}[1]{\mathtt{#1}}
\newcommand{\D}[1]{\mathscr{#1}}

\newcommand{\C}[1]{\mathcal{#1}}
\newcommand{\BB}[1]{\mathbb{#1}}
\newcommand{\U}[1]{\underline{#1}}
\newcommand{\OV}[1]{\overline{#1}}
\newcommand{\F}[1]{\mathfrak{#1}}
\newcommand{\tr}{\trianglelefteq}
\newcommand{\To}{\Rightarrow}
\newcommand{\STo}{\stackrel{\sim}{\To}}
\newcommand{\cc}[1]{{\color{cyan}#1}}
\newcommand{\mm}[1]{{\color{magenta}#1}}
\newcommand{\rr}[1]{{\color{red}#1}}


\newcommand{\LY}{\lambda Y}
\newcommand{\PLY}{P\lambda Y}
\newcommand{\Gfin}{\text{BPHORS}^{<\infty}}
%P\lambda Y^{<\infty}}
\newcommand{\Ginf}{\text{BPHORS}^{\infty}}
%P\lambda Y^{\infty}}
\newcommand{\AFF}[1]{#1_{\mathrm{aff}}}


\newcommand{\nonterm}{\C N}
\newcommand{\tri}{\triangleleft}


\newcommand{\N}{\BB N}
\newcommand{\Q}{\BB Q}
\newcommand{\R}{\BB R}
\newcommand{\Qrel}[1]{#1\mathbf{Rel}}
\newcommand{\Qrelkleisli}[1]{#1\mathbf{Rel_!}}
\newcommand {\redp}[2]{\xrightarrow{#1}_{#2}}
\newcommand {\redbigp}[1]{\xrightarrow{#1}}
\newcommand {\redbigpcbn}[1]{\xrightarrow{#1}_{CbN}}
\newcommand {\redsp}[2]{\xRightarrow{#1}_{#2}}
\newcommand\Dir{d} % metavariable for directions, which range over {L,R}
\newcommand{\bydef}{\: := \:}
\newcommand{\0}{\mathtt{0}}
\newcommand{\1}{o}
\newcommand{\phors}{{(\mathcal{N}, \mathcal{T}, \mathcal{R}, S)}}
\newcommand{\gphors}{\mathcal{G}}
\newcommand{\Pterm}[1]{\mathbb{P}({#1} \downarrow)}
\newcommand{\ExpTime}[1]{\mathbb{E}({#1} \downarrow)}
\newcommand{\Prob}[1]{\mathsf{Prob{#1}}}
\newcommand{\len}{\mathsf{len}}
\newcommand{\fix}{\mathsf{fix}\:}
\newcommand{\Rsemiring}{\mathcal{R}}
\newcommand{\Qsemiring}{\mathcal{Q}}
\newcommand{\Rinf}{\mathbb{R}^{+\infty}_{\geq 0}}
\newcommand{\Qpos}{\mathbb{Q}_{\geq 0}}
\newcommand{\supp}{\mathsf{supp} \:}

\providecommand{\dotdiv}{% Don't redefine it if available
  \mathbin{% We want a binary operation
    \vphantom{+}% The same height as a plus or minus
    \text{% Change size in sub/superscripts
      \mathsurround=0pt % To be on the safe side
      \ooalign{% Superimpose the two symbols
        \noalign{\kern-.35ex}% but the dot is raised a bit
        \hidewidth$\smash{\cdot}$\hidewidth\cr % Dot
        \noalign{\kern.35ex}% Backup for vertical alignment
        $-$\cr % Minus
      }%
    }%
  }%
}


% TODO
\newcommand{\TODO}[1]{{\color{magenta}\bfseries{#1}}}





\begin{document}


%\def\BibTeX{{\rm B\kern-.05em{\sc i\kern-.025em b}\kern-.08em
%    T\kern-.1667em\lower.7ex\hbox{E}\kern-.125emX}}




\title{On Higher-Order Program Verification through Generating Functions and Linear Logic}


\author{{Anonymous Authors}}
%
%
%\author{\IEEEauthorblockN{1\textsuperscript{st} Davide Barbarossa}
%\IEEEauthorblockA{\textit{University of Bath} }\\
%Bath, UK \\
%db2437@bath.ac.uk
%\and
%\IEEEauthorblockN{2\textsuperscript{nd} Paolo Pistone}
%\IEEEauthorblockA{\textit{Universit\'e Claude Bernard Lyon 1} }\\
%Lyon, France \\
%paolo.pistone@ens-lyon.fr}



\begin{abstract}


\end{abstract}

\keywords{Quantum computation, lambda-calculus, compilation, geometry of interaction}

\maketitle









\section{Introduction}

% !TEX root = main.tex

\paragraph*{Probabilistic Higher-Order Verification}


% Probabilistic programming provides a well-established family of methods and tools to specify probabilistic models as well as to perform statistical inference tasks using high-level, compositional, languages. For instance, languages like \emph{Church} or \emph{Anglican} enable the design of generative Bayesian models through finite lists of higher-order and recursive specifications. This provides the user with a user-friendly and high-level environment for stochastic reasoning, allowing them at the same time to implement general purpose inference methods via uniform and re-usable code, without having to adapt them to each model by hand.

% A key advantage of this approach to probabilistic modeling is the possibility of exploiting the theoretical background of programming language theory to design verification and model-checking methods for such languages in a way which is both compositional and grounded on solid mathematical foundations. In the deterministic setting, a vast literature has explored the model-checking of \emph{higher-order recursive schemes} (HORS) \cite{}, a specification language equivalent to the simply-typed $\lambda$-calculus enriched with a fixpoint operator, building over the paradigmatic result from \cite{} of the decidability, over such programs, of all properties expressible in monadic second order logic (MSO), including termination. 

Model checking is one of the most successful techniques for the verification of programs and systems, and it has found applications in a wide range of areas, leading to the development of widely used tools such as SPIN and PRISM. While in its original formulation model checking focused on finite-state systems modeled by Kripke structures, its scope of application has gradually expanded over time, in directions that are often very different from one another. In this paper, we are particularly interested in two such directions.

The first concerns the model checking of systems whose evolution is intrinsically \emph{probabilistic}. Along this line of research, the literature has produced a rich body of results, both positive and negative, regarding the decidability and tractability of the underlying verification problem (see \cite{} \warn for a recent survey). It is worth noting that when the property to be verified is a reachability problem, or analogously termination (appropriately generalized to a probabilistic setting), the verification problem remains decidable even beyond the finite-state case, as in, for example, recursive Markov chains or pushdown automata.

But there is also another generalization of model checking, likewise obtained by considering a broader class of systems than those traditionally studied, namely so-called \emph{higher-order model checking}. It has been known for about twenty years that this problem is decidable when the underlying specification is an arbitrary MSO formula and the program is an arbitrary \emph{higher-order recursion scheme} (HORS, \cite{} \warn), a specification language equivalent to the simply-typed $\lambda$-calculus enriched with a fixpoint operator. Problems such as termination or reachability are therefore decidable for a broad class of programs encompassing higher-order types and recursion. These positive results have nonetheless revealed
%been extensively examined with respect to extensions of the language, revealing 
a certain degree of fragility: relatively minor extensions suffice to render the problem undecidable.

What, then, happens when one considers model-checking problems for probabilistic \emph{and} higher-order programs? Recently, Dal Lago et al.~\cite{DBLP:journals/lmcs/KobayashiLG20} have introduced \emph{probabilistic higher-order recursive schemes} (PHORS), a probabilistic variant of HORS, which can be seen as the natural randomized variation on the theme of HORS. Unfortunately, verifying natural probabilistic counterparts of reachability and termination properties is in general a much more difficult task and is undecidable for PHORS, contrary to what happens for HORS and recursive Markov chains. For example, in a probabilistic setting, a natural property is \emph{almost sure termination} (AST), that is, termination with probability 1. Now, while termination is the quintessential \emph{semi-decidable} property for a Turing-complete language (and is even \emph{decidable} for HORS),  AST is a $\Pi^0_2$-complete in Turing-complete probabilistic languages \cite{DBLP:journals/acta/KaminskiKM19}, and its decidability fails for PHORS, already at order 2. For these reasons, the recent literature has focused on capturing sub-classes of PHORS for which AST and, possibly, the related \emph{positive almost sure termination} (PAST) - i.e.~the finiteness of the \emph{expected} number of steps to termination - could be shown decidable, and thus possibly amenable to verification and model-checking techniques. Notably, while \cite{DBLP:journals/lmcs/KobayashiLG20} established the decidability of AST for order-1 PHORS, Li et al.~\cite{DBLP:conf/lics/LiMO22} established the decidability of both AST and PAST for the \emph{affine} PHORS, i.e.~the recursive programs which are allowed to use each of their inputs \emph{at most} once during their reductions.

Despite the aforementioned results, the nature of the problem remains poorly understood, and a precise assessment of its computational difficulty is still lacking. For instance, it is currently unknown whether the decidability problem for PHORS remains $\Pi^0_2$-complete, or whether it can, while being undercidable, be placed lower in the arithmetical hierarchy. Similarly, the fragments of the PHORS language for which decidability results are known are incomparable, and no class of PHORS encompassing all of them has been identified. We therefore believe that it is important to investigate the nature of this problem in greater depth, which is the goal of this paper. In the remainder of this introduction, we present a new perspective that will allow us to shed fresh light on the problem.


%\begin{itemize}
%\item some well-known facts about deterministic HORS $\to$ decidability, model-checking
%
%\item probabilistic case $\to$ decidability of AST fails already at order 2
%
%\item the difficulty lies in computing or estimating the probability of termination, as this requires to \emph{count} over infinitely many reductions, even though these may exhibit a somehow finitary pattern
%
%\item Known results: decidability for \emph{affine} and \emph{order 1} PHORS, but undecidability already at order 2
%\end{itemize}


\paragraph*{Probabilistic Termination via Generating Functions}

A fundamental issue underlying the difficulty of estimating the probability of termination of higher-order programs is that this requires to \emph{count} probabilities over a typically infinite set of reductions, even though such reductions might all follow a somehow finitary pattern. 

Such counting problems are common in the field of combinatorics, in which they are addressed by studying the algebraic and analytic properties of the corresponding \emph{generating functions}, i.e.~of a formal power series that one can naturally associate with the sequence of numbers that one wishes to compute or estimate.   
A famous example is provided by the well-known \emph{Catalan numbers} $C_n=\frac{1}{n+1}\binom{2n}{n}$, where $C_n$ is the number of labeled binary trees with $n$ nodes. 
The generating function $c(x)=\sum_{n=0}^{\infty}C_n x^n$ is a power series that can be characterized as the solution of the algebraic equation $xc(x)^2-c(x)+1=0$, a property that can be used to shed light over several properties of the $C_n$, and notably to replace the infinite series by a closed (and computable) expression $c(x)=\frac{1+\sqrt{1-4x}}{2x}$.

%
%In combinatorics, these kind of situations are typical of \emph{counting sequences}, that is, of sequences $(a_n)_{n\in \BB N}$ that represent the cardinality $a_n$ of a certain class of entities which can be constructed out of $n$ basic elements (e.g.~the famous 
%A standard and powerful approach to study counting sequences $(a_n)_{n\in \BB N}$ is by investigating the analytic properties of the corresponding \emph{generating functions} $F(x)=\sum_{n=0}^\infty a_n x^n$, formal power series defining analytic functions over some open subset of the complex numbers.  

Starting from Chomsky and Sch\"utzenberger's (\cite{CHOMSKY1963118}), generating functions have been widely applied in formal language theory. For example, it is well-known that for all regular languages $L$, the associated generating function $L(x)=\sum_{n=0 }^{\infty}L_nx^n$, where $L_n$ counts the words of $L$ of length $n$, is \emph{rational}, that is, it can be written as a fraction $L(x)=p(x)/q(x)$ of two polynomials.
%
%
% defined by linear recurrence equations (like e.g.~the Fibonacci numbers $f_{n+2}=f_n+f_{n+1}$), and the corresponding generating function $\widehat L(x)=\sum_{n=0}^{\infty}L_nx^n$ is thus \emph{rational}, i.e.~a fraction $\frac{\mathscr{p}(x)}{\mathscr{q}(x)}$ of two polynomials.
When $L$ is context-free, instead, $L(x)$ is \emph{algebraic}, that is, is the solution, like $c(x)$ above, of some polynomial equation $p(x,L(x))=0$. 
%Finally, when $L$ is generated by an indexed grammar (cite), a class of languages sitting in between the context-free and the context-sensitive languages (and equivalent to the order-2 HORS), $L(x)$ may fail to be algebraic
%$y=L$ of some polynomial equation of the form $\mathscr p(x,y(x))=0$ (while this property fails for \emph{indexed grammars}, a class of grammars lying in the middle between the context-free and the context-sensitive ones, and equivalent to the order-2 HORS \cite{}).

%Knowing that the generating function of some language is rational or algebraic allows one to deduce several interesting properties about the language, like asymptotic estimations for the growth of their counting sequences or disproving 
%
%to providing, in some cases, closed forms for the corresponding generating functions, and thus exact methods for 

Now, suppose $\gphors$ is a PHORS, i.e.~a higher-order recursive probabilistic program, and that $\gphors$ makes, at each of its reduction steps, an unbiased choice between two different alternatives. 
Letting $\gphors_n$ be the number of \emph{distinct} terminating reductions of $\gphors$ with exactly $n$ steps, the power series $a_\gphors(z)=\sum_{n=0}^{\infty}\frac{\gphors_n}{2^n}z^n$ captures the probabilistic behavior of $\gphors$. Notably, $a_\gphors(1)$ precisely computes its probability of termination, and, as we'll see, the derivative $a_{\gphors}'(1)$ precisely captures the \emph{expected} number of reduction steps to termination.
 
Could we study, then, the complexity of the termination problem for a PHORS $\gphors$ by studying the algebraic or analytic properties of its generating function $a_\gphors(z)$?




%
% Indeed, given some probabilistic program $M$ of ground type, one can consider the sequence $(t_n)_{n\in \mathbb N}$ counting the number of distinct reductions of $M$ to termination making $n$ reduction steps; under the simplifying assumption that $M$ makes an unbiased probabilistic choice at each reduction step, considering the generating function $F(x)=\sum_{n=0}^{\infty}\frac{t_n}{2^n}x^n$, the value $F(1)$ precisely captures the probability of termination of $M$.
%Knowing whether $F(x)$ is rational or algebraic can then lead to methods to compute the value $F(1)$ explicitly, or in any case to obtain relevant information about it.
%

%
%\begin{itemize}
%
%\item probability of termination corresponds to a counting problem: for given $n$, how many terminating reductions with $n$ unfoldings?
%Calling $a_n$ this number, then $P(M\Downarrow)=\sum_n a_n\frac{1}{2^n}$.
%
%\item manipulating power series $\to$ theory of generating functions
%
%\item well-studied for counting problems: for a language $L$, let $L_n=\sharp\{\text{words $w\in L$ of length }n\}$, then
%\begin{itemize}
%\item for $L$ regular, $L_n$ is C-finite (i.e.~satisfies a linear recurrence equation, like Fibonacci numbers $F_{n+2}=F_{n+1}+F_n$)
%\item for $L$ context-free, $L_n$ is algebraic (i.e.~the function $L(x)=\sum_n a_nL^n$ is zero of a polynomial $p(x,L(x))=0$);
%\item for $L$ indexed grammar (equivalent to order 2) $L_n$ may be not algebraic.
%\end{itemize}
%
%
%\item remark that if the termination series $a(x)=P(M\Downarrow)=\sum_n a_nx^n$ is algebraic (i.e.~$p(x,a(x))=0$), then AST is decidable via the first-order theory of the reals. 
%
%\item Yet, all this is restricted to grammars and first-order languages, and HORS theory shows that climbing up to higher-orders makes the complexity of the underlying problems increase.
%\end{itemize}

\paragraph*{Generating Functions via Linear Logic}

Generating functions have been applied in the literature both for the study of
formal grammars \cite{DBLP:books/sp/KuichS86} and 
 \emph{first-order} programming languages (also in the probabilistic case, cite people \warn).  However, extending this natural and powerful approach to functional programs is challenging, as it is not obvious how to associate \emph{higher-order} functions  with sequences of real numbers, as required to induce a generating function. 


In this work we provide a solution to this problem, by presenting the first combinatorial study of the termination problem for {higher-order} recursive programs like the PHORS.
To do this, we exploit in an essential way two ideas coming from the toolbox of linear logic.
The first fundamental ingredient is the \emph{weighted relational model} \cite{DBLP:conf/lics/LairdMMP13}, which associates proofs in linear logic, as well as PCF programs, with families of \emph{formal power series} with coefficients in a continuous semiring.
We show how this model can be used to associate each PHORS with the corresponding generating function $a_\gphors(z)$ in an elegant and compositional way.

As mentioned before, linearity has been recognized as a key factor to ensure the decidability of the AST problem for PHORS (\cite{DBLP:conf/lics/LiMO22}).
The viewpoint of generating functions provides then a \emph{novel} way to look at this result:
using our interpretation via the weighted relational model, we will show that for all affine PHORS, while the corresponding language might \emph{not} be context-free, the corresponding generating function is, in fact, always algebraic.


 Actually, the combinatorial viewpoint allows us to go beyond the linear/affine restriction and consider even \emph{non-linear} PHORS typable via the discipline of \emph{graded exponentials} (cite \warn): this discipline allows one to define types of the form $!_n A \multimap B$, intuitively expressing functions from $A$ to $B$ using their inputs \emph{at most} $n$ times. 
 
 Hence, on the one hand, the combinatorial approach leads us to capture a \emph{larger} class of PHORS than \cite{DBLP:conf/lics/LiMO22}, including PHORS admitting both bounded and unbounded forms of duplication, for which the AST and PAST problems are decidable.
At the same time, it also suggests the \emph{robustness} of the class of PHORS individuated by \cite{DBLP:conf/lics/LiMO22}, as all PHORS typable in our discipline will be shown 
 \emph{equivalent} (in terms of both probabilistic termination and the corresponding language) to some affine PHORS, even though the latter may have size more than exponential compared to the original, non-linear, one.


%
%
%
%our approach provides a \emph{new} method, grounded on algebraic combinatorics, to understand (and to re-establish) the decidability of AST/PAST for affine PHORS; on the other hand, it
% 
% each such non-linear PHORS will be shownHence, on the one hand, 

%
%
% our approach naturally leads to consider finitary exponentials, as expressed by linear types of the form $!_n A \multimap B$, intuitively expressing functions from $A$ to $B$ using their inputs \emph{at most} $n$ times.
%
%
% 
%
%, which interprets proofs in linear logic (as well as $\lambda$-terms) with matrices valued over a continuous semiring. Following a folklore ideathis model can be formulated 
%
%The fundamental tool that  such languages with the theory of generating functions 
%
%
%Notably, leveraging a well-studied denotational semantics coming from linear logic, the \emph{weighted relational semantics} \cite{}, we show that any PHORS can be associated with a formal power series whose coefficients are defined by a countable system of polynomial equations. We then explore different type disciplines capturing classes of PHORS whose corresponding power series are algebraic, and we deduce from this the decidability of their associated AST and PAST problems.
%
%While our approach provides a new method, grounded on algebraic combinatorics, to re-establish the decidability of order-1 and affine PHORS, it also encompasses programs which are \emph{both} non-linear programs and of arbitrary order.
%
%Beyond the relational semantics, that, following \cite{}, we formulate in terms of formal power series, a second important ingredient that comes from linear logic is the theory of \emph{graded linear comonads:} while linearity (or, better, affinity) had already been recognized as a key factor to ensure decidability of probabilistic termination, our approach naturally leads to consider finitary exponentials, as expressed by linear types of the form $!_n A \multimap B$, intuitively expressing functions from $A$ to $B$ using their inputs \emph{at most} $n$ times.


%
%\begin{itemize}
%
%\item\textbf{Take home sentence:} 
%the main goal of this work is to show that the theory of generating functions can be applied to the study of the termination problem for higher-order probabilistic programs. 
%Using a well-known semantics from linear logic, the weighted relational model, we show that each PHORS can be associated with a formal power series whose coefficients are defined by a countable system of polynomial equations. 
%
%
%
%
%
%
%
%
%%\item it is a folklore fact that weighted Rel interprets higher-order terms as formal power series over a continuous semi-ring.
%
%\item example: 
%$$\lambda f.\mathrm{fix}(\lambda x.fx\oplus \mathrm{True}):(\mathtt{unit}\multimap \mathtt{unit})\To \mathtt{unit}$$
%has all reduction traces $\lambda f.f^n\mathrm{True}$ (each with probability $\frac{1}{2^n}$), and 
% yields the generating function $G(f)=\sum_n \frac{1}{2^n}f^n$ with closed form $G(f)=\frac{1}{1-\frac{f}{2}}$.
% 
% 
%
%\item while such power series are not computable in general, we use the theory of generating functions to capture \emph{decidable} fragments of this semantics. 
%
%
%
%\item Notably, we capture a class of PHORS (of arbitrary order) whose termination is expressed by an algebraic power series, and for which the AST problem is thus decidable. This class comprises and extends previously studied classes of decidable PHORS, like order 1 and affine PHORS.
%
%
%\item two key ingredients:
%	\begin{enumerate}
%	\item \emph{the (weighted) relational semantics}, a well-known model of linear logic and lambda-calculus associating proofs/terms with formal power series valued over a continuous semi-ring. 
%
%	\item \emph{graded linear comonads:} while linearity (or, better, affinity) had already been recognized as a key factor to ensure decidability of probabilistic termination, our approach naturally extends to finitary exponentials, as expressed by linear types of the form $!_n A \multimap B$, intuitively expressing functions from $A$ to $B$ using their inputs \emph{at most} $n$ times.
%	
%	
%	\end{enumerate}
%
%
%
%\end{itemize}
%

\paragraph*{Contributions}
Our contributions can be resumed as follows:
\begin{varitemize}

\item First, we show that the weighted relational semantics naturally associates each PHORS with generating functions capturing probability of termination and expected number of steps to termination, whose coefficients are implicitly defined via a countable system of equations. 

\item Then, we introduce the \emph{finitely bounded PHORS}, $\Gfin$, extending the affine PHORS of \cite{DBLP:conf/lics/LiMO22} with finitely bounded non-linearity, and show that the corresponding generating functions are algebraic, leading to the decidability of AST and PAST.

\item Finally, we introduce the \emph{infinitely bounded PHORS}, $\Ginf$, which extend the $\Gfin$ with a restricted use of \emph{unbounded} non-linearity, notably enabling the \emph{composition} of different PHORS. We show that this extension preserves both algebraicity and the decidability of AST and PAST.
% 
%e then introduce \emph{finitely graded PHORS}, $\Gfin$, extending the affine PHORS of (Ong) with bounded non-linearity, and show that the corresponding generating functions are algebraic, leading to the decidability of AST and PAST.
%
%a first type system with \emph{finite} grades and fixpoints (indeed a non-linear extension of the type system of Li et al.~\cite{}), and show that the PHORS typable in this system are interpreted by power series that are algebraic over $\mathbb Q$ (yielding the decidability of the corresponding AST and PAST problems).
%%Via well-known linearization techniques, each such typable PHORS can be shown \emph{equivalent} to an affine PHORS in the sense of [Ong], yet the latter may have size more than exponential with respect to the former.
%%% While this system types possibly non-linear PHORS of arbitrary order, it types \emph{all} order1 and affine PHORS.
%
%\item We then explore an extension of the first type system which enables the introduction of \emph{infinite} grades in a controlled way, 
% allowing us to capture even algebraic PHORS which may use their inputs an \emph{unbounded} number of times. This extension relies on a \emph{parameterization} method for the weighted relational semantics which we think may lead to further results in the future. 


\end{varitemize}


\section{From PHORS to Generating Functions, via Linear Logic}

In this section we provide an overview on our approach to probabilistic termination via algebraic generating functions.

\subsection{Algebraic Power Series from Finitary PHORS}

The weighted relational semantics arises from linear logic, and is indeed based on a \emph{precise count} of the number of times that a program may use each of its inputs during any of its reductions. 
For example, the interpretation of an order-2 program $M:(o\to o)\to (o\to o)$ yields a generating function which of the form
$$
 \model{M}(y,x)=\sum_{n=0}^{\infty}a_n y^n x
$$ 
where the real coefficient $a_n$ indicates the probability that 
$M$ terminates through a reduction that uses its functional input $y$ exactly $n$ times (and $x$ once).
For instance, the program 
$$M=\mathrm{Fix}(\lambda hyx. x\oplus hy(yx))$$
which generates the probabilistic tree 
$$x\oplus (yx\oplus (y^2x\oplus (y^3x\oplus \dots$$ 
yields the power series 
$$ \model{M}(y,x)=\sum_{n=0}^\infty \frac{1}{2^{n+1}}y^n x=\frac{x}{2-y},$$
as the probability of terminating using $y$ exactly $n$ times is $\frac{1}{2^{n+1}}$.

Observe that we are silently assuming, for simplicity, that order 1 programs $P:o\to o$ are always \emph{linear}, that is, they may use their input precisely once. This implies that any such program is interpreted by a unique coefficient (the probability of $P$ terminating using its input once), and that the variable $x$ above may indeed only occur with exponent 1. While this is a simplification, it is not a very strong one, since, as we will see in Section ??,  order-1
PHORS are always \emph{affine}, that is, they may use their input variables \emph{at most} once.




%
%\begin{align*}
%Hfx&= H(B\circ f)x \oplus_{p_1} (f\circ f)(Bx)\\ 
%Bx&= x\oplus_{p_2} \Omega\\
%S&= HB e
%\end{align*}
%
%
%Language associated:
%$$
%\{ 
%a^{n}b^{2n}\mid n\in \mathbb{N}, n\geq 1
%\}
%$$
%
%\begin{align*}
%Hfgx&= H(A\circ f)(B\circ g)x \oplus_{p_1} (f\circ g\circ f)x\\ 
%Ax&= x\oplus_{p_2} \Omega\\
%Bx&= x\oplus_{p_3} \Omega\\
%S&= HAB e
%\end{align*}
%
%Language associated:
%$
%\{ 
%c^{n+1}a^{n}b^{n}a^n\mid n\in \mathbb{N}\}
%$

Consider now a program defined by the following equations:
\begin{equation}\label{eq:phors1}
\begin{aligned}
Hyx&=( H(A\circ y)x \oplus_{a}
H(B\circ y)x )\oplus_{a}y(yx)\\
Ax&= x\oplus_{b} \Omega\\
Bx&=x\oplus_{c} \Omega\\
S&=HIe
\end{aligned}
\end{equation}

This is indeed an example of PHORS: the upper case letters are called \emph{non-terminal} symbols, the execution of the program starts from the order 0 non-terminal $S$ by applying instances of the equations, read from left to right, as well as probabilistic choices, until the unit constant $e$ is, eventually, produced. 
Notice that $I$ and $\Omega$ stand, respectively, for the identity and diverging terms, and that $a,b,c,d$ stand for rational biases for the probabilistic choice operators.


We can canonically associate an infinite tree with the program above by considering binary function symbols $a,b,c$ with each choice operator. The word language (defined as in \cite{}) consisting of all the finite branches of this tree is then 
$$
\mathcal L_M=\{ a^{2|w|+2}ww\mid w\in \{b,c\}^*\}.
$$
Notice that this language is not context-free, as it contains an arbitrary word repeated twice.

In the relational interpretation, each order-2 non-terminal symbol $N$ yields a sequence of coefficients $N_n$ (and a corresponding generating function  $\model{N}(y,x)=\sum_nN_ny^n x$), and each order-1 non-terminal $N$ yields a unique real coefficient. 
The equations \eqref{eq:phors1} translate then naturally into a system of \emph{polynomial} equations over such coefficients. On the one hand, we immediately get $A\cdot x= bx$ and $ B\cdot x=cx$, that is, $A=b, B=c$; for the $H_n$ we can solve using $\model{H}(y,x)=\sum_n H_n y^n x$, which yields
\begin{align*}
H_2y^2 x&= 
\alpha
H_2y^2x
+ \beta y^2x,\\
H_ny^n x&= 0 \qquad (n\neq 2),
\end{align*}
where $\alpha=(a^2b^2+a(1-a)c^2)$ and $\beta=1-a$.
In other words, $H_n=0$ for all $n\neq 2$, while $z=H_2$ 
%is a root of the polynomial $\mathscr p\in \BB R[x,y][z]$
%$$
%\mathscr p(x,y)(z)=\delta xy^2 \cdot z+ (1-a)xy^2,
%$$
%which 
can be easily computed $H_2=\frac{a-1}{a^2b^2+a(1-a)c^2-1}$. 
Notice that this implies that a reduction of $H$, \emph{independently of the number of its unfoldings}, will always end up using its functional input precisely twice.

Now, observe that from the equations above it follows 
$$
\model{H}(y,x)= \alpha\cdot \model{H}(y,x)+\beta y^2x
$$
that is, the power series $\model{H}(y,x)$ is a solution of the polynomial equation
$\mathscr p(x,y,z(y,x))=0$, where
$\mathscr p(x,y)(z)=(\alpha-1)\cdot z+ \beta y^2x$, so it is algebraic.


This example illustrates one fact that is general and one that only works in a \emph{restricted} class of situations. The general fact is that, for \emph{any} PHORS, the relational interpretation produces a set of polynomial equations defining the real coefficients interpreting each non-terminal symbol; however, this system will in general be \emph{infinite}: already at order 2, as we saw, we obtain sequences of coefficients $(a_n)_{n\in \mathbb N}$, each with its own equation. 
We can thus generally translate a PHORS into a system of polynomial equations over \emph{countably many} unknowns.

In the example, though, we realized that, out of all the coefficients $H_n$, only $H_2$ may be different from zero. More generally, whenever we realize that, out of the countably many unknowns of the (relational interpretation of the) program, only \emph{finitely many} are non-zero, we obtain a finite system of polynomial equations, which implies that each such coefficient is an algebraic real number, and that, globally, the power series interpreting each non-terminal are themselves algebraic. 


How can we enforce, then, a program to translate into finitely many coefficients? The solution comes, again, from linear logic: as we saw, each coefficient represents reductions using the inputs a fixed number of times; what if we may provide a \emph{finite bound} on the number of uses that a program may make of each of its inputs?

A first natural idea is to restrict ourselves to linear, or even affine, programs. This is indeed the approach taken in \cite{}. However, as the example above suggests, one may well allows programs to use their inputs more than once, even a very large number of times, as soon as we can provide a fixed bound for this number (as the final number of coefficients will rely - even though exponentially, see ??? - on it). 
Indeed, a standard and well-studied way to impose finite bounds is via \emph{(affine) graded exponentials} \cite{} $!_nA$, where a program of type $M:\ !_nA\multimap B$ is forced to use its input \emph{at most} $n$ times. Interestingly, as we will see, this restriction continues to hold in presence of fixpoints: if $A$ is a finitely graded higher-order type and $M: A\to A$ is a program which may use its input $A$ \emph{unrestrictedly}, then its fixpoint $\mathrm{Fix}M:A$ is still ``finitary''.
%: its interpretation still has a number of unknowns, which can be traced as the points of some algebraic variety, as we'll see.


For example, the non-terminal $H$ in our example could be typed as 
$H:\ !_2(!_1o\multimap o)\multimap (!_1o\multimap o)$, since, as we saw, any reduction of $H$ uses its functional argument twice.



At the same time, observe that, while our example is non-linear, it can well be \emph{linearized}: we can design some affine PHORS that generates the same infinite tree, for example:
\begin{align*}\label{eq:phors2}
Ly_1y_2x&=( L(A\circ y_1)(A\circ y_2)x \oplus_{a}
L(B\circ y_1)(B\circ y_2)x )\oplus_{a}y_1(y_2x)\\
Ax&= x\oplus_{b} \Omega\\
Bx&=x\oplus_{c} \Omega\\
S&=HIIe
\end{align*}
Notice that the unique functional variable $y$, that was used twice, is now replaced by \emph{two} functional variables $y_1,y_2$, used once.
Linearization is a well-known procedure in linear logic which, intuitively, corresponds to unfolding a (finite) exponential $!_nA$ as an $n$-fold tensor $A\otimes\dots \otimes A$. 
As a consequence, the word languages that are obtained via PHORS typed via graded exponential coincide with those obtained via the affine PHORS of \cite{}. 
However, notice that the linearized PHORS may well have a number of variables that is exponential with respect to those of the original, non-linear one. 


\subsection{Deciding AST and PAST for Finitary PHORS}

We now show how \emph{almost sure termination} (AST) and \emph{positive almost sure termination} (PAST) can be decided for a PHORS typable via graded types.
Let us recall that AST and PAST are, respectively, the problems to know whether a program terminates with probability 1, and whether its expected number of steps before termination is finite. 

Given a PHORS typable via graded types, we have seen that, for each non-terminal symbol $N$, the associated power series $\model{N}$ contains finitely many non-zero terms $N_ky^k$. As all such (finitely many) coefficients are given in terms of the other ones via polynomial equations, the set of their possible values forms an algebraic variety.
As a consequence, the interpretation of the source non-terminal $S:o$, which is obtained by combining other non-terminals as well as a unique constant $e:o$, can be expressed as a polynomial combination of the coefficients. 
At this point, it is not difficult to design a formula in the decidable \emph{first order theory of the reals} \cite{} that expresses that such a polynomial combination of algebraic reals is equal to 1, and thus to decide AST.

We can also obtain a first-order formula expressing PAST as follows.
In the relational semantics a choice $M\oplus_a N$ with bias $a$ is interpreted as a convex sum $a\model M+(1-a)\model N$; in order to count the number of reduction steps we can multiply each such convex sum by a fresh parameter $w$, yielding $w(a\model M+(1-a)\model N)$. 
Logically, this corresponds to adding a fresh linear variable $w:o\multimap o$ and replacing each choice $M\oplus_a N$ by $w(M\oplus_a N)$. 
Notice then that, with this new variable in the game, the type of the source non-terminal $S$
is now $(o\multimap o)\multimap o$, that is, $S$ 
 is now interpreted as a formal power series 
$\model{S}(w)=\sum_{n=0}^\infty S_nw^n$, where $S_n$ is the probability of termination using the fresh parameter $w$ exactly $n$ times, that is, making exactly $n$ probabilistic choices. We will show in Section \ref{} that the power series $\model{S}(w)$ remains algebraic even after this \emph{parameterization} via $w$.

As a paradigmatic example, consider the order-1 PHORS
\begin{align*}
Fx&= F(Fx)\oplus_{\frac{1}{2}} x & S&=Fe,
\end{align*}
corresponding to a simple random walk. Parameterizing $ F(Fx)\oplus_{\frac{1}{2}} x$ as $w( F(Fx)\oplus_{\frac{1}{2}} x)$ yields the algebraic power series $C(w)$ given by 
$$
C(w)= \frac{1}{2}\cdot {(wC^2(w)+w)},
$$
whose solution gives precisely the generating function of the {Catalan numbers}
$C(w)=\sum_{n=0}^\infty C_nw^n = \frac{1-\sqrt{1-4w}}{2w}$ (it is indeed well-known that $C_n$ counts the number of $n$-step paths for the simple random walk)


Now, observe that the expected number of steps (i.e.~of probabilistic choices) is precisely given by $\model{S}'(1)$, where $\model{S}'$ is the \emph{derivative} of $\model{S}$:
$$
\model{S}'(w)=\sum_{n=1}^\infty n\cdot S_{n}w^{n-1}.
$$
The derivative $a'(x)$ of an algebraic power series $a(x)$ is still an algebraic power series,
and there is a well-known method to express $a'(x)$ as a polynomial in $a(x)$ with rational function coefficients \cite{}. All this allows us to express PAST (i.e.~$\model{S}'(1)<\infty$) via some first-order formula over the real numbers, and thus to decide it.




\subsection{Towards Non-Finitary PHORS via Parameterization}

As we saw, a PHORS typable via graded types may use its functional inputs only a finite number of times, independently of the number of its unfoldings. 
On the other hand, in Section ?? we explore a \emph{parameterization method} that can be used to show the algebraicity also of PHORS that may use their functional inputs an \emph{arbitrary} number of times.

Indeed, an instance of this method underlies our previous discussion of PAST, which relied on showing that the generating function $\model S(w)=\sum_{n=0}^\infty S_nw^n$ for the number of steps to termination of an algebraic PHORS is itself an algebraic power series, in which the variable $w$ may appear arbitrarily many times.

The same idea can then be used to show the algebraicity also for PHORS which are not finitary. For instance, the following order-2 PHORS
\begin{align*}
Lyx&=Ly(Lyx)\oplus_a y(yx),\\
Bx&= x\oplus\Omega,\\
S&= LBe,
\end{align*}
can be seen as a functional and non-linear variant of the simple random walk, in which the function variable $y$ may be used an arbitrary (even) number of times. 

What makes this example, as well as the previous ones, work, is the remark that the non-terminal $L$ uses the functional variable $y$ as a parameter, that is, we are never asked to compute values of $L$ over, say, $y^2$ or $2y+1$. 
Formally, this can be captured by looking at this PHORS as an algebraic formal power series 
with coefficients taken \emph{themselves} from a semiring of formal power series. In the relational semantics, this corresponds to lifting the underlying continuous semiring of weights from the (extended) positive reals $\BB R_{\geq 0}^\infty$ to the semiring of formal power series
$\fps{\BB R_{\geq 0}^\infty}{y}$ in the variable $y$, as explained for instance in \cite{}.
At the same time, we show that all such cases can be captured by extending our graded type system with a new rule that allows to introduce \emph{infinite} grades $!_\infty A$ in a controlled way. 



%
%\begin{align*}
%Hfx&= (p_1H(p_3fx)+p_2H(p_4fx) + (1-p_1-p_2)f^2x\\
%&= (p_1p_3+p_1p_4)Hfx+(1-p_1-p_2)f^2x
%\end{align*}
%
%So $y=H(f,x)$ is solution of the polynomial equation $\mathscr p(f,x,y)=0$, where
%$$
%\mathscr p(f,x,y)= (p_1p_3+p_1p_4-1)y+(1-p_1-p_2)f^2x.
%$$
%
%The solution sequence $(H_n)_{n\in\mathbb N}$ is the following: 
%\begin{align*}
%H_n&=0 \quad (n\neq 2)
%\\
%H_2&=\sum_{k=0}^\infty\sum_{k=m+n}(p_1p_3)^m(p_2p_4)^n\\
%&= \sum_{m,n}(p_1p_3)^m(p_2p_4)^n\\
% &=\sum_{m=0}^\infty(p_1p_3)^m\cdot \sum_{n=0}^\infty(p_2p_4)^n\\
% &=\frac{1}{1-p_1p_3}\cdot \frac{1}{1-p_2p_4}
%\end{align*}
%which yields the probability of termination $\sum_nH_n=\frac{1}{1-p_1p_3}\cdot \frac{1}{1-p_2p_4}$.
%\begin{itemize}
%
%\item fundamental idea: when studying the relational interpretation of a probabilistic program, we can look at it as a formal power series, whose coefficients can be seen as \emph{countably many} unknowns. These are infinite since one has to consider trajectories that may use inputs an arbitrary number of times (as in the example above)
%
%%\item fixpoints corresponds then to imposing a system of \emph{polynomial equations} over such unknowns; yet, solving a system of infinitely many polynomial equations is far beyond what can be hoped!
%
%\item yet, what it we impose a \emph{bound} on the number of uses that the program can do of each of its inputs?
%This is a well-known approach that uses \emph{(affine) graded types} $!_nA\multimap B$, meaning ``produce $B$ using an input $A$ at most $n$ times'' 
%
%\item under such constraints, the number of unknowns to find becomes \emph{finite} (although possibly very large). In a similar way, when computing the semantics of $\mathrm{fix}M:A$, where $A$ is finitely graded but $M:A\To A$ may use its input $A$ unboundedly, we show that we obtain a system of \emph{finitely} many polynomial equations, whose solution set forms then an \emph{algebraic variety}.
%
%\item Using standard algebraic reasoning, the interpretation of this fixpoint is thus shown to be a power series $F$ that is solution of a \emph{unique} polynomial equation $p(x,F(x))=0$, yielding the decidability of AST. 
%
%
%\end{itemize}



\section{PHORS}
% !TEX root = main.tex
In this section we recall Higher Order Recursion Schemes (HORS), their probabilistic counterpart, Probabilistic  Higher Order Recursion Schemes (PHORS), as well as their correspondence with the $\lambda Y$-calculus. 

\subsection{HORS, a.k.a.~the $\lambda Y$-calculus}
HORS, widely used in higher order model checking, can be seen as grammars that generate infinite ranked trees (or equivalently, infinite typed lambda terms).
In the theory of HORS, the simple types are generated by the grammar $T= \1 \mid T \to T$. We can also define the order of a type $ord(\1) =0$ and $ord(T \to S)= \max(ord(T)+1, ord(S))$.
An HORS $\gphors$ can then be defined as a 4-uple $\phors$ where $\mathcal{N}$ is a set of typed non-terminals, $\mathcal{T}$ is a set of typed terminals, $S$ is a distinguished non-terminal called the starting symbol of $\gphors$ and $\mathcal{R}$ is a set of rewriting rules, at least one for each non terminal, of the form:
$L x_1 \dots x_n = t$, where $FV(t) \subseteq \{x_1, \dots x_n\} \cup \mathcal{T}$. These rules are to be understood as recursive definition of the non-terminals that can be progressively unfolded: to $\gphors$ we will associate the rewriting system induced by the rules $L t_1 \dots t_n \to t[x_1/t_1, \dots x_n/t_n]$. The rewriting - beginning from $S$ - will generate at each step finite (simply typed) terms containing the terminals $\mathcal{T}$; the "limit" of this set of terms will be the $\textit{value tree}$ defined by the $\mathcal{G}$ (for formal definitions, see Carayol or the overview by Ong). The nodes of this tree will be labelled by non terminals and its branching factor at each node will be equal to the arity of the non terminal labelling that node.  


The \emph{branch language} $\mathcal L_{\gphors}\subseteq \mathcal T^*$ of an HORS $\gphors$ is the branch language of its value tree: its words are the sequences of terminals encountered in some branch of the tree.
 It is well known that the branch languages of order-1 PHORS coincide with context free languages, while the branch languages of order-2 PHORS coincide with the \textit{indexed languages} defined by Aho (cite) (check this, Ong says for word generating HORS  of order 2 safety is not a restriction!). 


%If there is \textit{exactly} one equation for each non-terminal, the HORS is said deterministic; otherwise, it is non deterministic.Given an HORS, its order is defined as the maximal order of its non terminals.

%
%Ranked trees can be equivalently defined as set of words over an alphabet, called the \emph{branch language} of the tree (cite Courcelle, 1981 and survey by Ong). HORS can also deal with this different approach. To see this, recall that a word can be seen as a tree with branching factor at most one; hence an HORS only involving terminals of arity 1 is called a word-generating HORS. Given an HORS, we can always build another HORS of the same order that generate the branch language of its value tree: to do this we simply replace every terminal of arity $n$ and type $A_1 \to \dots \to A_n \to A$ with $n$ distinct non terminals $f_i$ of arity 1 and type $A_i \to A$ and a non-terminal $\tilde F_i$ (of type $A_1 \to \dots \to A_n \to A$) with the (non-deterministic) set of rewriting rules $\tilde F_i x_1 \dots x_n \to x_i$ for $i=1 \dots n$.
% It is well known that the branch languages of an order 1 PHORS coincide with context free languages, while the branch languages of an order 2 PHORS coincide with the \textit{indexed languages} defined by Aho (cite) (check this, Ong says for word generating HORS  of order 2 safety is not a restriction!). 
\begin{example}
	Consider the order-2 HORS $\gphors \bydef \phors$ defined as follows: its non terminals  are $S:\1$, $L: (\1 \to \1) \to \1 \to \1$, its terminals are $e: \1$ or arity 0, $l:\1 \to \1$ of arity 1 and $h:\1 \to \1 \to \1 $ of arity 2. Its rewriting rules are:
	\begin{align*}
		&S = L r e\\
		& L f k = h k (L f (L f k))   
	\end{align*}
	The first reduction steps will be $S \to L r e \to h e (L l (L l e))$ Now we can for example unfold the external $F$ and obtain $ h e (h (L l e) (L l (L l (L l e))))$.
\end{example}


The language of HORS is equivalent to the $\lambda Y$-calculus, that is, the simply-typed $\lambda$-calculus enriched with cartesian products and a fixpoint $Y:(T\to T)\to T$. 
Given a HORS $\gphors$:
 \begin{equation*}
\label{phors010}
 	\left\{
 	\begin{aligned}
 		&L_1 x_{1,1} \dots  x_{1,k_1} = t_1\\
 		&\dots \\
 		&L_1 x_{n,1} \dots  x_{n,k_n} = t_n\\
 	\end{aligned}
 	\right.
	\end{equation*}
define the term 
\[
M_{\gphors}:= Y \ \lambda \langle L_1,\dots, L_n\rangle. \langle \bar t_1 \dots \bar t_n \rangle,
\]
where $\bar t_i \bydef \lambda x_{i,1} \dots \lambda x_{x, n_i}. t_i$. Letting  
$\Gamma_{\mathcal T}$ be the set of assignments $f:T_f$ associating each terminal $f\in \mathcal T$ with its type, we can then associate each non-terminal $L_i:T$ with a term 
$M_{L_i}:=\pi_iM_{\gphors}$ so that $
\Gamma_{\mathcal T}\vdash_{\lambda Y} M_{L_i}:T$.





\subsection{PHORS, a.k.a.~the $\lambda_{\oplus} Y$-calculus}
PHORS are a probabilistic extension of HORS: they can be obtained by adding to the syntax of HORS a set new constant symbols $\oplus_p, p \in \mathbb{Q}$, of type $\1 \to \1 \to \1$, $t_1 \oplus_p t_2$ representing a probabilistic choice that chooses with probability $p$ the first argument and with probability $1-p$ the others. Following this intuition, we allow recursive definition of non terminals to be of the form 
\[L x_1 \dots x_n \to t_L \oplus_p t_R,\]
with the two (one step) rewriting rules
\begin{align*} 
L t_1 \dots t_n & \redbigp{\mathsf l,p} t_L[x_1/t_1, \dots x_n/t_n],\\
L t_1 \dots t_n &\redbigp{\mathsf r,p} t_R[x_1/t_1, \dots x_n/t_n].
\end{align*}
% We will consider from now only PHORS which are deterministic (i.e they have one reduction rule for each non-terminal). 
 Here, we annotated the rewriting to keep track of the probability and the direction of this choice. For a reduction $R: s \redbigp{\sigma_1 ,p_1} s_1 \dots  \redbigp{\sigma_k ,p_k} s_k$, we also write $R:s \redbigp{\sigma,p} s_k$, where
 $\sigma = \sigma_1 \dots \sigma_n$ and $p = p_1 \times \dots \times p_n$, and we call 
 $a(R)=\sigma$ and
 $w(R)=p$, respectively, the \emph{address and weight of $R$}. 
 
%Following (Dal Lago Grellois), we will restrict attention to PHORS $\gphors$ with a unique terminal symbol $\mathcal T=\{e:\1\}$.
 
 Then we can define the probability of termination of a PHORS $\gphors = \phors$ :
$$\Pterm{\gphors}= \sum_{ n \: \text{normal form}}\sum_{R:S \redbigp{\sigma, p} n} p$$
end the expected number of steps to termination:
$$\ExpTime{\gphors}=  \sum_{ n \: \text{normal form}}\sum_{R:S \redbigp{\sigma, p} n} |\sigma |p $$
We can now define the two problems that will be central in this paper:
\begin{definition}
	\begin{varitemize} Given a PHORS $\gphors$,
		\item AST is the problem to decide whether $\Pterm{\gphors}=1$
		\item PAST is the problem to decide  whether $\Pterm{\gphors} < + \infty$
	\end{varitemize}
\end{definition}

\begin{example}\label{ex:phors0}
Consider the order-1 PHORS defined by
\begin{align*}
Fx&= F(Fx)\oplus_{\frac{1}{2}} x,\\
S&= Fe.
\end{align*}
As any choice either adds one $F$ or deletes one, and termination comes when $e$ is reached, it simulates a simple random walk starting from $+1$, adding $\pm 1$ at each step and terminating at $0$. We will see in the next sections that it is AST but not PAST. 
\end{example}

It is known that PAST implies AST. On the other side, in (cite Ugo, Charles, Kobayashi), it can be proved that AST is undecidable for PHORS of order at least 1, while a decision procedure exists for PHORS of order 1. In (cite Ong paper on linear PHORS), it is shown that, if we restricting to applicative terms typable in an affine type system, AST and PAST are decidable for the class of \textit{affine PHORS} thus obtained. The proof builds upon the corresponndence, devised by Clairambault and Murawski in (cite), between affine PHORS and restricted stack automata. As a consequence of the theorems of section V, we will see that this can be proved in a completely different way, without resorting to automata, but rather relying on linear logic and its relational model.



\begin{remark}
	To each PHORS we can associate an HORS whose value tree represents all the probabilistic choices that we are faced with during the reduction: to this aim, it is enough to treat every $\oplus_p$ as a terminal symbol of arity 2. Based on this translation of PHORS to HORS, we can see that the branching structure of the order-2 PHORS \eqref{eq:phors1}, yielding a non-context free branch language, 
%	:
%	\begin{align*}
%		Hyx&=( H(A\circ y)x \oplus_{a}
%		H(B\circ y)x )\oplus_{a}y(yx)\\
%		Ax&= x\oplus_{b} \Omega\\
%		Bx&=x\oplus_{c} \Omega\\
%		S&=HIe
%	\end{align*}
	cannot be simulated by any order-1 PHORS.
\end{remark}


The language of PHORS is equivalent to the $\lambda_{\oplus} Y$-calculus, that is, the extension of $\lambda Y$ with (ground) choice operators $M \oplus_p N$, with 
reduction rules
\begin{align*} 
M\oplus_p N & \redbigp{L,p} M &
M\oplus_p N &\redbigp{R,p} N,
\end{align*}
and typing rule
\[
\infer{\Gamma\vdash M\oplus_p N:\1}{\Gamma\vdash M:\1 & \Gamma\vdash N:\1}.
\]
The \emph{call-by-value reduction $\redbigp{}$} of the $\lambda_{\oplus} Y$-calculus is the closure of the reductions above as well as 
 $\beta$-reduction $(\lambda x.M)N\redbigp{\epsilon,1} M[N/x]$ and $Y$-reduction $YM\redbigp{\epsilon,1}M(YM)$ under the rule
 $M\redbigp{\sigma, p} N \ \Rightarrow \ MP \redbigp{\sigma, p} NP$. The address and weight of a reduction can be defined as for the PHORS. 
  
 
The translation of a PHORS $\gphors$ into a term $M_{\gphors}$ of the $\lambda_{\oplus} Y$-calculus works exactly as in the case of HORS. In particular, the reductions $R:S\redbigp{\sigma, p}n$ to normal form are in one-to-one correspondences with probabilistic call-by-value reductions $R:M_S\redbigp{\sigma, p}n$ to normal form in $\lambda_{\oplus} Y$, with same addresses and weights.

\begin{example}\label{ex:phors1}

The PHORS $\gphors$ from Example \ref{ex:phors0} is encoded in $\Lambda Y$ by $M_{\gphors}=Y \lambda\langle F,S\rangle. M':(\1\to \1)\times \1$, where 
$M'=
\langle \lambda  x.F(Fx)\oplus_{\frac{1}{2}} x, Fe\rangle$. 


\end{example}


%
%
% The recursive formalism of (P)HORS can be expressed in the $\lambda Y$ calculus (i.e the STLC to which for each type $T$ a fixpoint combinator $Y_G: (T \to T) \to T$) is added). In particular, given a PHORS:
% \begin{equation*}
% 	\label{phors010}
% 	\left\{
% 	\begin{aligned}
% 		&L_1 f_{1,1} \dots  f_{1,k_1} = t_1\\
% 		&\dots \\
% 		&L_1 f_{n,1} \dots  f_{n,k_n} = t_n\\
% 	\end{aligned}
% 	\right.
% \end{equation*}
% we associate to it the $\lambda Y$ term $\fix  \lambda.\langle L_1 \dots  L_n \rangle . \langle \bar t_1 \dots \bar t_n \rangle$, where $\bar t_i \bydef \lambda f_{i,1} \dots \lambda f_{i, n_i}. t$.


\section{Interpreting PHORS by Formal Power Series}

% !TEX root = main.tex
In this section, we recall the well-studied relational semantics of linear logic weighted over a continuous semiring $\Rsemiring$, in a slightly different flavor than, say, \cite{DBLP:conf/lics/LairdMMP13}: rather than using $\Rsemiring$-valued matrices, we will use formal power series over $\Rsemiring$, as in \cite{DBLP:journals/corr/abs-2501-15637}. We will then illustrate how this semantics associates each PHORS with a generating function capturing its probability of termination.

% Then, we focus on algebraic power series: this is a concept widely used in algebra and combinatorics (cite Flajolet, Kauers, Eisenbud), in the context of rings or fields. Its treatment in semi-rings is less standard, but extensively studied in the context of formal language theory (cite Kuich, Schlund). We will show that, through the relational semantics, such ideas apply naturally to the semantics of probabilistic $\lambda$-calculi.



%but we will not have to worry too much about this, as we will work with a semirings $\Rsemiring$ (usually $\Rinf$) that has a subsemiring embe\\

\subsection{Formal Power Series}
Let us first introduce formal power series (fps in short); for more details about them and their applications to combinatorics we refer to \cite{DBLP:books/daglib/0023751}, \cite{DBLP:series/tmsc/KauersP11}. Given a set $\Sigma$, 
let $!\Sigma$ be the set of finite multisets over it, i.e functions $\mu: \Sigma \to \N$ with finite support. Given a semiring $\Rsemiring$, the set of \emph{formal power series (with commuting variables) over $\Rsemiring$}, denoted $\fps{\Rsemiring}{\Sigma}$, is the set of all functions $!\Sigma \to \Rsemiring$. More concretely, if
 we introduce for each $s \in \Sigma$ a variable $x_s$ (we will denote by $x_\Sigma$ the set of all these variables), then a finite multiset $\mu \in !\Sigma$ can be seen as the monomial $x_\Sigma^\mu \bydef \Pi_{s \in \Sigma}x_s^{\mu(s)}$; then any formal power series $s\in \fps{\Rsemiring}{\Sigma}$ can then be expressed as a formal sum:
 $$s=\sum_{\mu \in !\Sigma} s_\mu x_\Sigma^{\mu}.$$ 
 We will sometimes write $s(x_\Sigma)$ to underline which variables appear in $s$.
 For each $s\in \fps{\Rsemiring}{\Sigma}$, its \emph{support} $\supp~s\subseteq \ !\Sigma$ is the set of multisets $\mu$ such that $s_{\mu}\neq 0$. 
 We denote as $\fpp{\Rsemiring}{\Sigma}\subseteq \fps{\Rsemiring}{\Sigma}$ the set of \emph{polynomials}, i.e.~of fps with finite support, which can written, as usual, as finite sums
  


% Given $\Sigma'\subset \Sigma$ and $s\in \fps{\Rsemiring}{\Sigma}$, its \emph{restriction to $\Sigma'$}, noted, 
%$s\vert_{\Sigma'}\in \fps{\Rsemiring}{\Sigma}$, is the composition of $s$ with the injection $\iota:!\Sigma'\hookrightarrow!\Sigma$, i.e.~$s\vert_{\Sigma'}(x_{\Sigma'})=\sum_{\mu\in !\Sigma'}s_{\iota(\mu)} x_{\Sigma'}^\mu$. 
%

  \begin{example}
  The power series $s(x)=\sum_{n} (1/2)^n x^n$ belongs to $\fps{\mathbb Q}{x}$.
  The power series $s(x)=\sum_{n=0 }^{\infty}\sum_{i+j=n} (1/3)^{n} x_0^{i}x_1^j$, i.e. \\
$ \sum_{\mu\in !\{0,1\}}(1/3)^{\mu(0)+\mu(1)}x^{\mu}$, belongs to $\fps{\mathbb Q}{x_0,x_1}$.
  \end{example}
 
 In $\fps{\Rsemiring}{\Sigma}$ we can define two operations: sum, performed componentwise: $\sum_{\mu \in !\Sigma} r_\mu x_\Sigma^{\mu} + \sum_{\mu \in !\Sigma} s_\mu x_\Sigma^{\mu} \bydef \sum_{\mu \in !\Sigma} (r_\mu + s_\mu)  x_\Sigma^{\mu}$ and the Cauchy product: 
 $
 \sum_{\mu \in !\Sigma} r_\mu x_\Sigma^{\mu} \cdot \sum_{\mu \in !\Sigma} s_\mu x_\Sigma^{\mu} \bydef \sum_{\kappa \in !\Sigma}\left (\sum_{\mu + \nu = \kappa} r_\mu s_\nu\right)  x_\Sigma^{\kappa}$.
 With these operations, $\fps{\Rsemiring}{\Sigma}$ is a semiring; if we take on it the pointwise partial order, it becomes a \emph{continuous} semiring: this means that all directed joins have a supremum, and such suprema commute with multiplication.
Our typical example here is the continuous semiring of positive extended reals $\Rinf$.   Continuity is essential when we want to define the \textit{composition} of formal power series. Take a power series $r \in \fps{\Rsemiring}{\Sigma}$ and let $s_\Sigma\in \fps{\Rsemiring}{\Sigma'}^{\Sigma}$ be a $\Sigma$-indexed family of power series over the set $\Sigma'$, $s_\sigma = \sum_{\nu \in !\Sigma'}s_{\sigma, \nu} y^\nu$. Then, we can define the power series $r(s_\Sigma) \in \fps{\Rsemiring}{\Sigma'}$ by the formula:

\vskip-3mm
{\small
% \begin{align*}
%\sum_{\kappa \in ! \Sigma'}  \left( \sum_{[\sigma_1^{m_1} \dots \sigma_j^{m_j}] \in !\Sigma} r_{{[\sigma_1^{m_1} \dots \sigma_j^{m_j}] }} \sum_{\nu_1 + \dots + \nu_{m_1 + \dots + m_j}= \kappa} \prod_{i=1}^{m_1 + \dots + m_j} s_{\sigma_i, \nu_i } \right)y_\Sigma^\kappa
%  \end{align*}
   \begin{align*}
   r(s_\Sigma)=
\sum_{\kappa \in ! \Sigma'}  \left( \sum_{\mu=[\sigma_1, \dots, \sigma_k] \in !\Sigma} \sum_{
{\tiny
\begin{matrix}
(\nu_1,\dots,\nu_{k})\in (!\Sigma')^k\\
\nu_1+\dots+\nu_{k}=\kappa
\end{matrix}
}
}
r_{\mu }\cdot 
 \prod_{i=1}^{k}  s_{\sigma_i, \nu_i } \right)y_\Sigma^\kappa
  \end{align*}
  }
  Observe that the if there is at least an $s_\sigma$ such that $s_{\sigma, []} \neq 0$, the sum over $!\Sigma$ will be infinite; still, by continuity, we can define its value to be the sup of the partial sums (yet, notice that, over a - non-continuous - ring, this composition would not be well-defined). If the power series $s_\Sigma$ are all constants (i.e. $s_{\sigma, \mu}=0 \: \forall \mu \neq []$), we will say that $r(s_\Sigma) \in \fps{\Rsemiring}{\emptyset}= \Rsemiring$ is the \emph{value of $r$ at the point} $(s_{\sigma, []})_{\sigma \in \Sigma}$.
With respect to these operations, $\fpp{\Rsemiring}{\Sigma}$ form a (non-continuous) sub-semiring of $\fps{\Rsemiring}{\Sigma}$. 
  
  
Since $\Qsemiring:=\fps{\Rsemiring}{\Sigma}$ is a continuous semiring, for any set $\Sigma'$ we can consider the continuous semiring $\fps{\Qsemiring}{\Sigma'}$ of formal power series whose coefficients are themselves power series (on $\Rsemiring$), and we have the isomorphism $\fps{\Qsemiring}{\Sigma'}=\fps{(\fps{\Rsemiring}{\Sigma})}{\Sigma'}\equiv\fps{\Rsemiring}{\Sigma+\Sigma'}$: this generalizes the well-known remark that a fps in two variables $s(x,y)$ can always be written as a fps $s_y(x)=\sum_n s_n(y)x^n$ in $x$ whose coefficients are fps $s_n(y)$ in $y$.
  
  
%   For a finite set $\Sigma$ and a natural number $k$, let $!_k\Sigma$ be the set of multisets $\mu$ of maximal multeplicity $k$ (i.e $\max_{x \in \Sigma} \mu(x)\leq k)$. This is clearly a subset of $!\Sigma$, corresponding to monomials over $x_\Sigma$ having degree at most $n$ in each variable. 
%   If a power series $s\in \fps{ \Rsemiring}{\Sigma}$ is such that $\supp s \subseteq !_k \Sigma$, for some $k\in \N$, we say that $s$ is a \textit{generalized} polynomial; 
%   
%    if moreover $\supp s$ is finite we say that it is a \emph{classical} polynomial. All classical polynomials are generalized polynomials, but the converse is not true unless $\Sigma$ is finite. When we say `polynomial' without any specification, we always mean a classical polynomial.
   
   
%  For a finite set $\Sigma$ and a natural number $k$, let $!_k\Sigma$ be the set of multisets $\mu$ of maximal multeplicity $k$ (i.e $\max_{x \in \Sigma} \mu(x)\leq k)$. This is clearly a subset of $!\Sigma$, corresponding to monomials over $x_\Sigma$ having degree at most $n$ in each variable. If a power series $s\in \fps{ \Rsemiring}{\Sigma}$ is such that $\supp s \subseteq !_k \Sigma$, for some $k\in \N$, we say that $s$ is a \textit{generalized} polynomial;  if moreover $\supp s$ is finite we say that it is a \emph{classical} polynomial. All classical polynomials are generalized polynomials, but the converse is not true unless $\Sigma$ is finite. When we say `polynomial' without any specification, we always mean a classical polynomial.
%  \begin{example}
%  	$\sum_{n} (1/2)^n x^n$ belongs to $\fps{\Rinf}{x}$ and it is neither a generalized nor a classical polynomial, while  $\sum_{n \geq 1} (1/2)^n x_1 \dots x_n \in \fps{\Rinf}{x_\N}$ is a generalized polynomial but not a polynomial, and  $\sum_{1 \leq n \leq 50} (1/2)^n x_1 \dots x_n \in \fps{\Rinf}{x_\N}$ is a polynomial.
%  \end{example}
%   As usual, a classical polynomial $p$ can be written as a finite sum:
%  $$p= \sum_{n_1 \dots n_{|\Sigma|} = 0}^k p_{n_1 \dots n_{|\Sigma|}} x_{\sigma_1}^{n_1} \dots x_{\sigma_{|\Sigma|}}^{n_{|\Sigma|}}$$

%A formal power series with finite support is simply called a



% It is obviously not continuous, as the supremum of a collection of polynomials can be an infinite power series. Composition of polynomials do not involve infinite sums: the coefficients of the composition $r(s_\Sigma)$ are indeed polynomials in the coefficients of $r$ and $s_\Sigma$:
%  \begin{lemma}
%  	For each $\Sigma$ and $\Sigma'$ and for each $k \in \N$ and for each  $\Sigma$-indexed family of natural numbers $k_\Sigma$, there exists a $!\Sigma'$-indexed family $p_{\Sigma'}$ of polynomials over $!_k \Sigma \sqcup \left( \bigsqcup_{\sigma \in \Sigma} !_{k_\sigma}\Sigma'  \right)$ with the following property:
%  	if $r \in \fpp{\Rsemiring}{\Sigma}$ and $s_\Sigma$ is a $\Sigma'$-indexed family of power series $\in  \fpp{\Rsemiring}{\Sigma'}$, then $\mu$ coefficient of $r(s_\Sigma) \in \fpp{\Rsemiring}{\Sigma'}$ is equal to:
%  	$$p_{\mu}((r_\kappa)_{\kappa \in !_k\Sigma}, (s_{\sigma_1, \kappa})_{\kappa \in !_{k_1}\Sigma}  \dots (s_{\sigma_n, \kappa})_{\kappa \in !_{k_n}\Sigma}) $$
%  \end{lemma}
If we do not start, as we do here, with a continuous semiring $\Rsemiring$, but rather with a ring $R$, almost all constructions still work: we can similarly define the set $\fps{R}{\Sigma}$ of formal power series over $R$, which is a ring, as well as its subring  $\fpp{R}{\Sigma}$ of polynomials over $R$. Still, the composition of formal power series will not be defined in general, as it involves an infinite sum, while the composition of polynomials remains well-defined.
  
  
  \subsection{The Weighted Relational Semantics}
  
    Now, we recall the link between formal power series and the weighted relational model of linear logic \cite{DBLP:conf/lics/LairdMMP13}. Given a continous semiring $\Rsemiring$, the category 
     $\Qrelkleisli{\Rsemiring}$ has sets as objects, and morphisms $\Qrelkleisli{\Rsemiring}(X, Y)$ are $\Rsemiring$-valued matrices indexed by $!X \times Y$. 
     $\Qrelkleisli{\Rsemiring}$ is the coKleisli category, with respect to the $!$ comonad, of the more familiar category  $\Qrel{\Rsemiring}$ of sets and $\Rsemiring$-valued matrices. 
    
     At the same time, $\Qrelkleisli{\Rsemiring}$ can be seen as a category of formal power series:    a matrix $(t_{\mu, y})_{\mu \in !X, y \in Y}$ can be identified with the $Y$-indexed family of fps $(\sum t_{\mu, y} x_X^\mu)_{y \in Y}$, the morphisms of $\Qrelkleisli{\Rsemiring}$ can be identified with (families of) formal power series, i.e.~$\Qrelkleisli{\Rsemiring}(X,Y)\equiv\fps{\Rsemiring}{X}^Y$.
%%$\Qrel{\Rsemiring}$ is symmetric monoidal closed and is thus a model of the \emph{linear} $\lambda$-calculus, in which both the tensor product and the linear function type are interpreted as $\model{A\otimes B}=\model{A\multimap B}=\model{A}\times\model{B}$.    
%The category $\Qrelkleisli{\Rsemiring}$ is the coKleisli category $\Qrelkleisli{\Rsemiring}$, with respect to the $!$ comonad: its morphisms $\Qrelkleisli{\Rsemiring} (X, Y)$ are given by $\Rsemiring$-valued matrices indexed by $!X \times Y$; since a matrix $(t_{\mu, y})_{\mu \in !X, y \in Y}$ can be seen as a $Y$-indexed family of power series $(\sum t_{\mu, y} x_X^\mu)_{y \in Y}$, the morphisms of $\Qrelkleisli{\Rsemiring}$ can be identified with (families of) formal power series, i.e.~$\Qrelkleisli{\Rsemiring}(X,Y)\equiv\fps{\Rsemiring}{X}^Y$.
    Indeed, composition in $\Qrelkleisli{\Rsemiring}$ is given by (pointwise) composition of the underlying power series, with identities $\mathrm{id}_X\in  \fps{\Rsemiring}{X}^X$ given by
    $\mathrm{id}_i(x_X)=x_i$.
%    
%     By this identification, we can from now one see $\Qrelkleisli{\Rsemiring}$ as a category whose objects are set and whose morphisms $\Qrelkleisli{\Rsemiring} (X, Y)$ are  $Y$-indexed family $s_Y$ of power series over $x_X$.\\

%The category $\Qrel{\Rsemiring}$ is symmetric monoidal closed, with monoidal products and their their adjoints both given by $X\times Y$, yielding an interpretation of the \emph{linear} $\lambda$-calculus.
$\Qrelkleisli{\Rsemiring}$ is cartesian closed, with cartesian products given by $X+Y$ (with neutral $0:=\emptyset$) and exponentials given by $!X\times Y$. 
This allows us to interpret simple types as $\model o=1:=\{\star\}$, $\model{T\times U}=\model T+\model U$ and $\model{T\to U}=!\model{T}\times \model{U}$. Intuitively, a type $T$ translates into a set of variables, and a term $\Gamma\vdash t:T$ into a $\model{T}$-indexed family of fps $\model{t}^{\Rsemiring}_i(x_\Gamma)$ in \emph{as many variables as $\model{\Gamma}$}.
Here we can observe the main challenge appearing with higher-order types, as these are interpreted by \emph{infinitely many} variables: for instance, $\model{o\to o}=!\model o\times \model o\equiv\N\times 1\equiv \N$ translates into a countable sequence of variables, and 
$\model{(o\to o)\to (o\to o)}\equiv !\N\times\N$ has one distinct variable $x_{\mu,n}$ for each $\mu\in !\N$ and $n\in \N$.
Higher-order terms correspond thus to fps in \emph{countably many} variables.


%
%The fundamental intuition about the exponential is that giving a $(!X\times Y)$-indexed family of fps $s_{\mu,y}(z_Z)\in \Qrelkleisli{\Rsemiring}(Z,!X\times Y)=\fps{\Rsemiring}{Z}^{!X\times Y}$ is the same as giving a $Y$-indexed family of fps $s(z_Z,x_{X})_y=\sum_{\mu\in !X}s_{\mu,y}(z_Z)x_X^{\mu}\in \Qrelkleisli{\Rsemiring}(Z+X,Y)\equiv \fps{\Rsemiring}{Z+X}^Y$.
%This generalizes the correspondence between a fps $\sum_{\mu\in !X} s_{\mu}x^\mu$ and its family of coefficients $(s_\mu)_{\mu\in \ !X}$.
%The interpretation of a program $M:A\to B$ is a $\Rsemiring$-matrix $\model{M}^{\Rsemiring}\in\Rsemiring^{!A\times B}=\fps{\Rsemiring}{0}^{!A\times B}$ whose entries $ (\model{M}^{\Rsemiring})_{\mu,b}$ represent ways of yielding $b$ by using each input $a\in A$ exactly $\mu(a)$ times. 


\begin{example}\label{ex:churchtwo}
Consider the program $t=\lambda x.y(yx)$, where $y:(\1\to\1)\vdash t:\1\to\1$: recalling $\model{\1\to\1}=!\1\times \1\equiv \N$, 
we have that $t$ is interpreted by a $\N$-indexed family of power series
$(b_i(y_{\N}))_{i\in\N}\in \Qrelkleisli{\Rsemiring}(\N,\N)=\fps{\Rsemiring}{\N}^{\N}$.
%, corresponding to the fps $\sum_{i=0}^{\infty}s(y_{\N})_i x^i$. 
Now, think of the variables $y_{\N}$, which interpret the function $y:\1\to\1$, as encoding the fps $a(x)=\sum_n y_n x^n\in\Qrelkleisli{\fps{\Rsemiring}{\N}}(1,1)\equiv  \fps{(\fps{\Rsemiring}{\N})}{1}$;
the term $t$ should translate then into the composition 

{\small
\[
a(a(x))=\sum_n y_n\left(\sum_m y_mx^m\right)^n=
\sum_{i=0}^{\infty}
\left(
\sum_{{\tiny\begin{matrix}(n,m_1,\dots, m_n),\\ m_1+\dots+m_n=i\end{matrix}}}y_ny_{m_1}\dots y_{m_n}\right) x^i,
%\sum_{i=0}^{\infty}s_i(y_{\N})x^i,
\] 
}
which gives $b_i(y_{\N})= \sum_{{\tiny\begin{matrix}(n,m_1,\dots, m_n),\\ m_1+\dots+m_n=i\end{matrix}}}y_ny_{m_1}\dots y_{m_n}$.
%x^i=t(y_{\N})_i.
%\sum_{{\tiny\begin{matrix}(n,m_1,\dots, m_n),\\ m_1+\dots+m_n=i\end{matrix}}}y_n(y_{m_1}x^{m_1})\dots( y_{m_n}x^{m_n})\\
%&=
% \sum_{{\tiny\begin{matrix}(n,m_1,\dots, m_n),\\ m_1+\dots+m_n=i\end{matrix}}}y_ny_{m_1}\dots y_{m_n}x^i=t(y_{\N})_ix^i.
%\end{align*}
Notice that the evaluation $tx$ of $t$ over some variable $x:o$ is then precisely interpreted by $a(a(x))\in \Qrelkleisli{\Rsemiring}(\N+1,1)=\fps{\Rsemiring}{\N+1}\equiv\fps{(\fps{\Rsemiring}{\N})}{1}$.

%In other words, since $\fps{\Rsemiring}{\N+1}\equiv\fps{\fps{\Rsemiring}{\N}}{\1}$, 
%we can look at $s(y_{\N},x)$ as a power series in $x$ whose $i$-th coefficient is the fps $t(y_{\N})_i=  \sum_{{\tiny\begin{matrix}(n,m_1,\dots, m_n),\\ m_1+\dots+m_n=i\end{matrix}}}y_ny_{m_1}\dots y_{m_n}\in \fps{\Rsemiring}{y_{\N}}$.
%
%
%
%The program $y:(\1\to \1),x:\1\vdash Mx:\1$ is then interpreted by a single power series
%$t(y_{\N},x)\in \Qrelkleisli{\Rsemiring}(\N+1,1)=\fps{\Rsemiring}{\N+1}$ given by
%\[
%t(y_{\N},x)_i=s(y_{\N})_ix^i= 
%\sum_{{\tiny\begin{matrix}(n,m_1,\dots, m_n),\\ m_1+\dots+m_n=i\end{matrix}}}y_n(y_{m_1}x^{m_1})\dots( y_{m_n}x^{m_n}).
%\]
%In other words, the fps $s(y_{\N})_{i}$ is the coefficient of the map $\1\to \1$ corresponding to 
\end{example}

%
%\begin{example}\label{ex:churchtwobis}
%Consider again the program $M=\lambda x.y(yx)$, but  
%%
%% where $y:(\1\to\1)\vdash M:\1\to\1$.
%%Letting $\model{\1}=1=\{\star\}$, and 
%%observing that $\model{\1\to\1}=!1\times 1\equiv \N$, 
%%we have that $M$ is interpreted by a $\N$-indexed family of power series
%%$(s_i(y_{\N}))_{i\in\N}\in \Qrelkleisli{\Rsemiring}(\N,\N)=\fps{\Rsemiring}{\N}^{\N}$.
%%To make calculations easier
%let us assume now that $y$ is given the linear type $y:\1\multimap \1$, so $M$ has type $M:\1\multimap \1$ too.
%Since $\model{\1\multimap \1}=\model{\1}\times\model{\1}=1\times 1\equiv 1$, $M$ is now interpreted by a fps $s(y)\in \Qrelkleisli{\Rsemiring}(1,1)=\fps{\Rsemiring}{y}$; reasoning as in the case above, think of $y$, interpreting a linear function $\1\multimap \1$, as encoding the linear map $a(x)=yx$; the term $M$ translates then into $a(a(x))=y^2x$, 
%which gives $s(y)=y^2$ (observe that $s(y)$ coincides with the linear fps $s_1(y_{\N})=y_1^2$ of the family from the previous example). 
%% 
%%
%%%, corresponding to the fps $\sum_{i=0}^{\infty}s(y_{\N})_i x^i$. 
%%Now, think of the variables $y_{\N}$, which interpret the function $y:\1\to\1$, as encoding the fps $a(x)=\sum_n y_n x^n\in\Qrelkleisli{\fps{\Rsemiring}{\N}}(\1,\1)\equiv  \fps{(\fps{\Rsemiring}{\N})}{1}$;
%%the term $M$ should translate then into the composition 
%%\[
%%a(a(x))=\sum_n y_n\left(\sum_m y_mx^m\right)^n=
%%\sum_{i=0}^{\infty}
%%\left(
%%\sum_{{\tiny\begin{matrix}(n,m_1,\dots, m_n),\\ m_1+\dots+m_n=i\end{matrix}}}y_ny_{m_1}\dots y_{m_n}\right) x^i,
%%\] 
%%which gives $s_i(y_{\N})= \sum_{{\tiny\begin{matrix}(n,m_1,\dots, m_n),\\ m_1+\dots+m_n=i\end{matrix}}}y_ny_{m_1}\dots y_{m_n}$.
%%Notice that the evaluation $y:(\1\to\1),x:\1\vdash Mx:\1$ is precisely interpreted by $a(a(x))\in \Qrelkleisli{\Rsemiring}(\N+1,1)=\fps{\Rsemiring}{\N+1}\equiv\fps{(\fps{\Rsemiring}{\N})}{1}$.
%%
%\end{example}


%The example above shows that the usual evaluation map $\Qrelkleisli{\Rsemiring}(Z,!X\times Y)\Rightarrow \Qrelkleisli{\Rsemiring}(Z+X, Y)$, i.e.~
%$\fps{\Rsemiring}{Z}^{!X\times Y}\To\fps{\Rsemiring}{Z+X}^Y$, corresponds to passing from the $!X\times Y$-indexed family of power series 
%$s(z_Z)_{\mu,y}\in \fps{\Rsemiring}{Z}$ to the $Y$-indexed family
%$t(z_Z,x_X)_y=\sum_{\mu\in !X}s(z_Z)_{\mu,y}x_X^\mu\in \fps{\Rsemiring}{Z+X}$.
%When $Z=0$, this precisely corresponds to passing from a $!X\times Y$-indexed family of scalars
%$s_{\mu,y}$ to the corresponding $Y$-indexed family of power series $\sum_{\mu\in !X}s_{\mu,y}x_X^\mu$. 
%

%
%Notice that usual curryfication $\Qrelkleisli{\Rsemiring}(X,Y)\Rightarrow \Qrelkleisli{\Rsemiring}(1,!X\times Y)$ corresponds to passing from 
%
%For instance, an arrow $s(x_{\star})\in \Qrelkleisli{\Rsemiring}(\1,\1)\equiv\fps{\Rsemiring}{x_{\star}}\equiv \ !\1\to \Rsemiring\equiv\N\to\Rsemiring$ is a power series
%$s_{x_{\star}}=\sum_n s_nx_{\star}^n$.



Beyond exponentials, 
$\Qrelkleisli{\Rsemiring}$ is endowed with Conway fixpoints \cite{DBLP:journals/mscs/Hasegawa09}: given a morphism $f \in \Qrelkleisli{\Rsemiring}(X + Y, X)$,  we define $\fix f \in \Qrelkleisli{\Rsemiring}(Y, X)$ as follows: take the sequence $f^0 \bydef  0 \times id_Y \in \Qrelkleisli{\Rsemiring}(1 \times Y ,X \times Y), \: f^{n+1} \bydef f \circ (f^n \times id_Y) \circ \langle id_\1, \Delta_Y \rangle  \in \Qrelkleisli{\Rsemiring}(1 \times Y, X)$ and finally let $\fix_{Y,X} f \bydef \sup_n f^n \in \Qrelkleisli{\Rsemiring}( Y, X)$.
   It is worth to restate this construction in terms of power series: $f$ will be represented by an $X$-indexed family $(s_x(x_X, x_Y))_{x \in X}$. Then we can define its iterates and the fixpoint as follows, for $x \in X$:
   \begin{equation}\label{eq:fixpointeq}
   \begin{aligned}
   	& r_x^{(0)}(x_Y) = 0 \in \fps{\Rsemiring}{Y}\\
   	& r^{(n+1)}_x(x_Y) =  s_x(r_X(x_Y), x_Y)\\
   	& (\fix s_X)_x(x_Y) = \sup_n r^{(n)}_x(x_Y)
   \end{aligned}
   \end{equation} 
	From this, it is clear that the power series $r_X= \fix f$ are the minimal solution of the infinite family of equations: $(r_x = s_x(r_x, x_Y))_{x \in X}$. 
	
	
Using the cartesian closed structure and the fixpoints, any term 
$\Gamma\vdash t:T$
 of $\LY$ yields in $\Qrelkleisli{\Rsemiring}$
a family of fps $\model{t}^{\Rsemiring}\in \fps{\Rsemiring}{\model{\Gamma}}^{\model{T}}$.
% : letting types be interpreted by $\model{o}=1:=\{\star\}$, $\model{T\times U}=\model T+\model U$, $\model{T\to U}=!\model{T}\times \model U$, we obtain, for all term $\Gamma\vdash t:T$, a $\model{T}$-indexed family of fps
%$\model{t}^{\Rsemiring}\in \Qrelkleisli{\Rsemiring}{\model\Gamma,\model T}\equiv
%\fps{\Rsemiring}{\model\Gamma}^{\model T}$.
To interpret $\PLY$, we restrict our attention to semirings of the form $\Rsemiring=\fps{\Rinf}{\{z\}+\Sigma}$, where $z$ denotes a distinguished variable. This allows us to interpret probabilistic choice (with bias $p$) as 
\[
\model{M\oplus_p N}^{\Rsemiring}=p z\cdot\model{M}^{\Rsemiring}+(1-p) z\cdot\model{N}^{\Rsemiring}.
\]
As shown by Proposition \ref{prop:proba} below, the variable $z$ plays the role of a \emph{counter} for each probabilistic choice: each reduction $t\redbigp{\sigma, p}e$ will produce a monomial $pz^{|\sigma|}$ in the semantics.
%
%	
%	
%To interpret probabilistic choice, first, observe that in all $\Qrelkleisli{\Rsemiring}$ it is possible to interpret \emph{weighted choices} $q_0\cdot M+q_1\cdot N$, where $q_0,q_1\in \Rsemiring$, by letting
%  $\model{q_0\cdot M+q_1\cdot N}=q_0\model{M}+q_1\model{N}$.
%Starting from this, we will consider two different ways to interpret probabilistic choice:
%\begin{varitemize}
%\item taking $\Rsemiring=\Rinf$, we interpret a choice $M\oplus_p N$ with bias $p\in [0,1]$ letting $q_0=p$ and $q_1=1-p$, i.e.~
%$\model{M\oplus_p N}^{\Rinf}=p\model M+(1-p)\model N$; 
%\item taking $\Rsemiring=\fps{\Rinf}{z}$, we interpret a choice $M\oplus_p N$ via $q_0=p z$ and $q_1=(1-p) z$, i.e.~
%$\model{M\oplus_p N}^{\fps{\Rinf}{z}}=p z\cdot\model M+(1-p) z\cdot\model N$.



%\end{varitemize}

\begin{remark}\label{rem:tropical}
\cite{DBLP:journals/corr/abs-2501-15637} considers the interpretation of
\emph{parametric} choices  $M\oplus_x N$, i.e.~choices according to some unknown bias $x$, by taking the semiring 
  $\Qsemiring=\fps{\Rsemiring}{x,\overline x}$ and letting $\model{M\oplus_p N}^{\Qsemiring}=x\cdot M+\overline{x}\cdot N$. \end{remark}
% $\Qrelkleisli{\Rsemiring}$ has thus all the relevant structure to define an interpretation $\model{-}^{\Rsemiring}$ of higher-order languages with fixpoints, like the $\lambda Y$ calculus or PCF (see Laird Manzonetto for all details).

%
%For suitable semirings, \emph{probabilistic choices} can be interpreted via non-determinism and weights: if $\Rsemiring$ contains $\Rinf$ as a sub-semiring, we can interpret a choice $M\oplus_p N$ with bias $p\in [0,1]$ by rewriting it as $p\cdot M + (1-p)\cdot N$; 
%taking instead $\Qsemiring=\fps{\Rsemiring}{x,\overline x}$, it is even possible to interpret 
%\emph{parametric} choices $M\oplus_x N$, i.e.~choices according to some unknown bias $x$, by rewriting it as $x\cdot M+\overline{x}\cdot N$ (see Barbarossa and Pistone).

%One can interpret $\PLY$ (or even probabilistic PCF) in all $\Qrelkleisli{\Rsemiring}$ with $\Rsemiring$ of the form $\fps{\Rinf}{\Sigma}$. 

All this leads to the following definitions:
\begin{definition}[probabilistic generating function of a PHORS]
For every PHORS $\gphors=(\nonterm, \C R,S)$ and non-terminal $L_i:T_1\to\dots\to T_n\to o\in\nonterm$, letting $\Sigma=\model{T_1}+\dots+\model{T_n}$, we define:
\[
a_{L}(z)(x_{\Sigma}):=
 \model{\pi_i(Yt_{\gphors})f_1\dots f_n}^{\fps{\Rinf}{z}}\in \fps{(\fps{\Rinf}{z})}{\Sigma}.
\]  
In particular, the fps $a_{\gphors}(z):=a_S(z)\in \fps{\Rinf}{z}$ is called
the
 \emph{probabilistic generating function of $\gphors$}.

\end{definition}
Given the interpretation $\model{\gphors}^{\Rsemiring}\in\fps{\Rsemiring}{\model\nonterm}^{\model\nonterm}$ of the simply typed term $t_\gphors:\nonterm\to\nonterm$, observe that $a_{L_i}(z)(x_\Sigma)=\pi_i\left(\fix \model{{\gphors}}^{\Rsemiring}\right)$ is the minimal solutions of the equational system $(r_x=(\model{\gphors}^{\Rsemiring})_x(r_x,z))_{x\in \Sigma}$. 

The generating function $a_\gphors(z)$ precisely captures the call-by-name probabilistic execution of closed terms, as stated below:%the interpretation $\model{M}^{\Rinf}$ of a term $M:\1$ of ground type consists in a real number, since $\model{M}^{\Rinf}\in\Qrelkleisli{\Rsemiring}(0,\1)\equiv\fps{\Rsemiring}{0}\equiv\Rsemiring$, which coincides with the probability of termination of $M$, computed as the sum of the weights $w(R)\in \Rinf$ of all call-by-name probabilistic reductions $R:M{\to^*}e$:
\begin{proposition}\label{prop:proba}
For any PHORS $\gphors$,
\begin{align}
%\model{S}^{\Rinf}&=\mathbb P(\gphors\downarrow),
%%=\sum_{R:M\to^* e}w(R),
%\label{eq:prob1} \\
a_{\gphors}(z)&=\sum_{i=0}^{\infty}\mathbb P(\gphors\downarrow_i)z^i,\label{eq:prob2}
\end{align}
where $\mathbb P(\gphors\downarrow_i)=\sum_{R:t\to^i e}w(R)$ is the probability that $S$ terminates after exactly $i$ probabilistic steps. In particular, $a_\gphors(1)=\mathbb P(\gphors\downarrow)$. 
\end{proposition}
\begin{proof}
From \cite{DBLP:conf/lics/LairdMMP13, DBLP:journals/jacm/EhrhardPT18,DBLP:journals/corr/abs-2501-15637} we know that each reduction $S\redbigp{\sigma,p}e$ precisely adds up one monomial $pz^{|\sigma|}$, hence $a_{|\sigma|}z^{|\sigma|}$ in $a_\gphors(z)$ is the sum of the probabilities of all reductions of length $|\sigma|$.
\end{proof}
%While \eqref{eq:prob1} indicates that the interpretation in $\Rinf$ precisely captures the probability of termination, \eqref{eq:prob2} provides a \emph{finer} result: 
%By marking with the variable $z$ each probabilistic reduction step, the coefficients of $a_\gphors(z)$ count the probability of terminating in exactly $i$ steps. 
From Proposition \ref{prop:proba} we also deduce that the \emph{derivative} of $a_\gphors(z)$ captures the expected number of steps to termination:

\begin{corollary}\label{cor:expected}
%\begin{equation}\label{eq:expected1}
$a'_\gphors(1)=\sum_{i=1}^{\infty} i\cdot \mathbb P(\gphors\downarrow_i)=\mathbb E(\gphors\downarrow)$.
%\end{equation}

\end{corollary}
%\begin{proof}
%We have $\left(\model{M}^{\fps{\Rinf}{z}}\right)'(z)=\sum_i (i+1)\mathbb P(M\downarrow_i)z^{i}$
%
%\end{proof}


%\begin{remark}
%That $\model{M}^{\Rinf}$ precisely counts the probability of termination by summing the weights of all reductions is best seen from the parametric interpretation $\model{M}^{\Rsemiring}$, with $\Rsemiring=\fps{\Rinf}{x,\overline x}$, of (Barbarossa, Pistone) recalled above: in that case 
%$\model{M}^{\Rsemiring}$ is a power series of the form
%\begin{equation}\label{eq:prob2}
%\model{M}^{\Rsemiring}(x,\overline x)=\sum_{m,n}\sharp(m,n)x^m {\overline x}^n,
%\end{equation}
%where $\sharp(m,n)\in \N$ is the number of reductions $R:M\to^* e$ making $m$ left choices and $n$ right choices. One recovers \eqref{eq:prob1} by \emph{evaluating} the power series \eqref{eq:prob2} over the actual biases $x:=p, \overline x:=1-p$. 
%\end{remark}



%\begin{example}\label{ex:churchtwo2}
%Proposition \ref{prop:affine} drastically simplifies the computation of 
%
%\end{example}

\begin{example}\label{ex:phors2}
Consider the PHORS from Example \ref{ex:phors0}, with $t_F=F(Fx)\oplus_{\frac{1}{2}}x$ and $t_S=Fe$.
% and the corresponding $\PLY$-term $M_{\gphors}$ (cf.~Example \ref{ex:phors1}).
%
% Its interpretation $\model{M}^{\Rinf}\in 
% \Qrelkleisli{\Rinf}(1, \N+1)\equiv
% (\Rinf)^{\N +1}\equiv (\Rinf)^{\N}\times \Rinf$ (using $\model{\1\to \1}=\ !1\times 1\equiv\N$) is given by a pair $(f,s)$, where $f:\N \to \Rinf$ and $s\in \Rinf$.%
The interpretation in $\Qrelkleisli{\fps{\Rinf}{z}}$ 
%$\model{\lambda \langle F,S\rangle.M'}^{\Rinf}$ 
 of $\lambda \langle F,S\rangle. t_F:((\1\to \1)\times 1)\to (\1\to \1)$ is a $\N$-indexed family of fps 
$s^F_i(y_{\N+1})\in \fps{(\fps{\Rinf}{z})}{\N +1}$, given, for $i\in \N$, and recalling $b_i(y_{\N})$ from Example \ref{ex:churchtwo}, by
\[
s^F_i(y_{\N+1})=
\begin{cases}
\frac{1}{2}zb_1(y_{\N})+\frac{1}{2}z=\frac{1}{2}(z
y_1^2+z)
& \text{ if }i=1,\\
\frac{1}{2}zb_i(y_{\N})
&\text{ if }i\in\N, i\neq 1,
%\\
%\sum_{n}y_n&\text{ if }i=\star.
\end{cases}
\]
The interpretation of $\lambda \langle F,S\rangle. t_S:((\1\to \1)\times 1)\to \1$ is a unique fps 
$s^S(y_{\N +1})\in \fps{\Rinf}{\N +1}$ given by $s^S(y_{\N +1})=\sum_{i\in \N} y_i$. 
 
Computing $ \mathsf{fix}\  \langle s^F,s^S \rangle$ means finding a minimal solution $(a_i(z))_{i\in \N+1}\in(\fps{\Rinf}{z})^{\N+1}$ of the fixpoint equations $a_{i}(z)=s^F_i(a_{\N+1}(z))$ ($i\in\N$) and $a_{\gphors}(z)=s^S(a_{\N +1}(z))$.
One can easily see that this yields $a_{i}(z)=0$, for $i\neq 1,\star$, while $a_\gphors(z):=a_\star(z)=a_1(z)\in \Rinf$ can be found as minimal solution of the polynomial equation
\begin{equation}\label{eq:alg1}
a_\gphors(z)=\frac{1}{2} \left (za_\gphors^2(z)+z\right).
\end{equation}
%%It is not difficult to check that the unique solution here is $a_1(y_{\N+1})=1$, which yields
%%$\model{t_{\gphors}}^{\Rinf}_1=\model{t_{\gphors}}^{\Rinf}_\star=1$ and 
%%$\model{t_{\gphors}}^{\Rinf}_{i+1}=0$, i.e.~that $\mathbb P(S\downarrow)=1$, i.e.~AST holds, and moreover  
%%$\mathbb P(Fx\to^* x)=1$.
%% ( terminates with probability $1$, and that $\Fx$.
%% is then given by the fixpoint of the family $t_{\N+1}$, i.e.~by the minimal solution $\mathsf{fix}\  s_{\N +1}$ of the family of equations
%%$(\mathsf{fix}\  s_{\N +1})_i=t((\mathsf{fix}\  s_{\N +1})_{\N +1})$. In the next section we will show that this is given by
%%
%%\[
%%(\mathsf{fix}\  s_{\N +1})_i
%%=
%%\begin{cases}
%%1
%%& \text{ if }i=1,\star,\\
%%0&\text{ if }i\in\N, i\neq 1.
%%\end{cases}
%%\]
%%The sequence $\model{\pi_1M_{\gphors}}^{\Rinf}=(\mathsf{fix}\  s_{\N +1})_{\N}$ interprets the term $M_F:=\pi_1M_{\gphors}:\1\to \1$ via the affine map $\sum_{i=0}^{\infty} (\mathsf{fix}\  s_{\N +1})_i x^i = x$; using Proposition \ref{prop:affine}, this translates into the fact that $\mathbf{P}(Fx\to^* x)=1$; 
%%the scalar $\model{\pi_2M_{\gphors}}^{\Rinf}=(\mathsf{fix}\  s_{\N +1})_\star=1$ interprets 
%% $M_S:=\pi_2M_{\gphors}$, and translates into $\mathbf P(S\to^* e)=1$.
%
%\end{example}
%
%\begin{example}\label{ex:phors3}
%Consider now the interpretation of the same PHORS in $\Qrelkleisli{\fps{\Rinf}{z}}$. 
%By a similar argument we are led to find a minimal solution $a_1(z)\in \fps{\Rinf}{z}$
%of the fixpoint equation
%\begin{equation}\label{eq:alg2}
%a_1(z)=\frac{za_1^2(z)+z}{2}.
%\end{equation}
We will see in the next section that this solution is given by the series
$a_\gphors(z)=\sum_{i=0}^{\infty}\frac{C_i}{2^{2i+1}} z^{2i+1}$, where $C_i$ is the $i$-th Catalan number.
\end{example}

  
 For order-1 PHORS, we have the following useful lemma:
\begin{lemma}\label{lemma:affine}
For all order-1 PHORS $\gphors=(\nonterm, \C R,S)$ and non-terminal symbol $L\in \nonterm$, the fps $a_L\in \fps{(\fps{\Rsemiring}{z})}{x_1,\dots, x_n}$ is \emph{affine}:
%
%first-order $\PLY$-term $x_1:\1,\dots, x_n:\1\vdash_{\PLY}t:\1$, the fps 
%$\model{t}^{\Rinf}(x_1,\dots, x_n)\in \fps{\Rinf}{x_1,\dots,x_n}$ is affine:
 there exists $w_0,w_1,\dots, w_n\in\fps{\Rinf}{z}$ such that
\[
a_{L}(z)(x_1,\dots, x_n)
=w_0(z)+w_1(z)x_1+\dots+w_n(z)x_n,
\]
where $w_0(1)=\mathbb P[L\vec x\to^* e]$ and
$w_{i+1}(1)=\mathbb P[L\vec x\to^* x_{i+1}]$.
\end{lemma}
This result translates the fact that in a reduction of $Lx_1\dots x_n$ to normal form, a ground variable $x_i:\1$ can occur in head position \emph{at most once}: as soon as it does, we must have $Lx_1\dots x_n\to^* x$, that is, the reduction has terminated.



%  Observe that, if $X=!X_1 \times X_2$ is an exponential object,
%  for all $f\in \Qrelkleisli{\Rsemiring}(Y,X)$, the usual evaluation arrow 
%  $
%%  Eval \circ f \times id_{Y_2}: 
%  Y \times Y_1 \to Y_2$ is represented by the family of series $(\sum_{\kappa \in !Y_1} (\sum t_{\mu, (\kappa, y_2)} x_X^\mu) x_{Y_1}^\kappa)_{y_2 \in Y_2}$\\
%  
  
  
  

{\color{red}

- formal power series over a commutative semiring. 

- recalls on polynomial equations and algebraic power series

- Proposition 1: $n$ polynomial equations over $n$ variables yield an algebraic fps 

- weighted relational model given in terms of formal power series (as we do with Davide)

- graded linear comonad $!_n$ ($n\in \mathbb N\cup\{\infty\}$)

- Proposition 2: fixpoint over finite set is an algebraic fps (by Proposition 1)




}



\section{Algebraic PHORS via Finite Grades}


{\color{red}

- rules of the type system

- interpretation in the model

- Theorem1: typable terms yields fps with finitely many non-zero coefficients, each coefficient is algebraic over $\mathbb Q$ 

- Theorem2: typable terms yields an algebraic power series

- Theorem3: AST is decidable (describe first-order formula for AST)

- Example1: order1 random walk, 

- Example2: $0^n1^n0^n$: order 2 non context-free but algebraic (both linear and non-linear)

- Example3: non-algebraic $0^n 1^{2^n}$ is not typable

}





\section{Algebraic PHORS via Infinite Grades}
{\color{red}

- discuss order 2 random walk: algebraic equations with power series coefficients correspond to admitting variables with infinite grade, but in a controlled way

- parametric polynomial equations and their solution

- Proposition 2. $n$ parametric polynomial equations in $n$ fps variables yield an algebraic fps

- extended type system

- Theorems 1/2/3 for the new type system, via Proposition 2

}










\section{Conclusion}
% !TEX root = main.tex
\section{Related Work}

An extensive literature in analytic combinatorics has explored different kinds of combinatorial structures having an algebraic generating function, in particular with respect to their asymptotic behaviour (for a comprehensive treatment, see \cite{DBLP:books/daglib/0023751}).
In the setting of imperative programming, the idea of using generating functions to analyse probabilistic programs is developed in \cite{Klinkenberg2021} and \cite{Klinkenberg2024}; notably, 
of particular interest, with respect to our work is the characterization of while loops giving rise to rational generating functions in \cite{Klinkenberg2021}.
On the side of formal language theory, there exists a huge amount of work about generating functions; particularly relevant to this work is the paper \cite{DBLP:journals/ita/AdamsFM13}, in which they study generating functions of Aho's indexed languages (corresponding to order-2 safe HORS, thus going beyond the usual framework of context-freeness) and they give an explicit procedure to compute these functions under some restrictions, but do not give a characterization of grammars with an algebraic generating function.

The connection between HORS and the relational semantics of linear logic first appears in  Melliès and Grellois' work \cite{DBLP:conf/csl/GrelloisM15}, \cite{DBLP:conf/mfcs/GrelloisM15}. Subsequently, in \cite{DBLP:journals/pacmpl/ClairambaultGM18} a linear-non linear typing system for HORS was introduced, allowing for a finer complexity analysis of model checking; its affine fragment is studied in \cite{DBLP:conf/mfcs/ClairambaultM19}. None of these works involve probabilistic behavior.

In \cite{DBLP:journals/lmcs/KobayashiLG20}, the problem of AST for PHORS is studied by directly translating a PHORS into (possibly higher-order) real functional equations: only in the case of order 1 PHORS, this produces a fixpoint algebraic system analogous the those considered in this work, thus implying a decidability result (while in the same work, undecidability from order-2 is proved). Techniques to approximate from above the probability of termination are also considered. In \cite{DBLP:conf/lics/LiMO22}, the decidability of AST/PAST for affine PHORS is proved . With respect to our approach, they perform the translation from PHORS to real equations via an in intermediate automata-theoretic step, based on the equivalence between affine HORS and restricted pushdown automata proved in \cite{DBLP:conf/mfcs/ClairambaultM19}.



\section{Conclusion}

In this work we have shown that the combinatorial method of generating functions can be adapted, via the weighted relational semantics of linear logic, to the study of higher-order probabilistic languages. We think that the main value of this work resides in opening the possibility of applying well-established methods from algebraic and analytic combinatorics to the study of higher-order languages. We provided a first demonstration of this by showing how  the decidability of termination for certain classes of PHORS (proved in the literature with syntactic methods like automata theory or intersection types) can be re-established, and actually \emph{extended}, in a relatively elementary way, via the notion of algebraic power series.

At the same time, several other directions of application of combinatorial methods can be mentioned: on the one hand, could one capture a class of PHORS, extending the algebraic ones, giving rise to \emph{D-finite} power series \cite{DBLP:series/tmsc/KauersP11}, i.e.~fps defined by linear \emph{differential} equations, and could this be related to the recently explored connections between {higher-order differentiation} and fixpoints in the weighted relational model \cite{DBLP:conf/lics/GalalL24}?
On the other hand, could the vast body of work on \emph{asymptotic estimations} \cite{DBLP:books/daglib/0023751} of generating functions be applied to extract approximated  information about the probabilistic behavior of even larger classes of PHORS?
%
%which extends the algebraic power series
%
%
%{\color{red}
%
%
%Here some perspectives, suggesting that our result is just the first and most obvious one.
%
%Extensions, e.g.~holonomic power series and restriction, e.g.~rational ones
%
%Asymptotic estimations: a huge literature, can we use them to provide probability estimations?
%
%Extracting properties of programs from the corresponding generating functions: factorization, recurrence properties, differential equations
%
%
%}





\end{document}
\endinput

