In this paragraph, first we introduce the weighted relational semantics of linear logic over a continous semiring $\Rsemiring$, in a slightly different flavour than the one used in (cite Laird, Manzonetto): rather than using $\Rsemiring$-valued matrices, we will use formal power series over $\Rsemiring$, as in (cite Paolo and Davide's work about tropical lambda calculus). Secondly, we will recall the notion of algebraic power series and then we will prove a proposition about the algebraic characterization of the fixpoint of an operator over power series.\\
First, we recall some notions about formal power series; for more details about them and their applications to combinatorics we refer to (cite the concrete tetrahedron). Given a finite set $\Sigma$, 
let us call $!\Sigma$ the set of finite multisets over it, i.e functions $\mu: \Sigma \to \NAT$ with finite support. If $\Rsemiring$ is a semiring , we call $\fps{\Rsemiring}{\Sigma}$ the set of functions $!X \to \Rsemiring$. More concretely, if
 we introduce for each $s \in \Sigma$ a variable $x_s$ (we will denote by $x_\Sigma$ the set of all this variables), then a multiset $\mu \in !\Sigma$ can be seen as the monomial $x_\Sigma^\mu \bydef \Pi_{s \in \Sigma}x_s^{\mu(s)}$; then a formal power series can be expressed as a formal sum:
 $$\sum_{\mu \in !\Sigma} r_\mu x_\Sigma^{\mu}  $$; we will also write $s(x_\Sigma)$ to underline which variables appear in $s$.\\
 In $\fps{\Rsemiring}{\Sigma}$ we can define two operations: sum, performed componentwise: $\sum_{\mu \in !\Sigma} r_\mu x_\Sigma^{\mu} + \sum_{\mu \in !\Sigma} s_\mu x_\Sigma^{\mu} \bydef \sum_{\mu \in !\Sigma} (r_\mu + s_\mu)  x_\Sigma^{\mu}$ and the Cauchy product: $$\sum_{\mu \in !\Sigma} r_\mu x_\Sigma^{\mu} \cdot \sum_{\mu \in !\Sigma} s_\mu x_\Sigma^{\mu} \bydef \sum_{\kappa \in !\Sigma} (\sum_{\mu + \nu = \kappa} r_\mu s_\nu))  x_\Sigma^{\kappa}$$
 With this operations, $\fps{\Rsemiring}{\Sigma}$ is a semiring ; if we take on it the pointwise (partial order), it becomes a continuous semiring. Continuity is essential when we want to define the \textit{composition} of formal power series. Take a power series $r \in \fps{\Rsemiring}{\Sigma}$ and let $s_\Sigma$ be a $\Sigma$-indexed family of power series over the set $\Sigma'$, $s_\sigma = \sum_{\nu \in !\Sigma'}s_{\sigma, \nu} y^\nu$. Then, we can define the power series $r(s_\Sigma) \in \fps{\Rsemiring}{\Sigma'}$ by the formula:
 \begin{align*}
  &r(s_\Sigma) =\\ &\sum_{\kappa \in ! \Sigma'} y_\Sigma^\kappa \left( \sum_{[\sigma_1^{m_1} \dots \sigma_j^{m_j}] \in !\Sigma} r_{{[\sigma_1^{m_1} \dots \sigma_j^{m_j}] }} \sum_{\nu_1 + \dots + \nu_{m_1 + \dots + m_j}= \kappa} \prod_{i=1}^{m_1 + \dots + m_j} s_{\sigma_i, \nu_i } \right)
  \end{align*}
  Observe that the if there is at least an $s_\sigma$ such that $s_{\sigma, []} \neq 0$, the sum over $!\Sigma$ will be infinite; still, being in a continuous semiring, we can define its value to be the sup of the partial sums. If the power series $s_\Sigma$ are all constants (i.e. $s_{\sigma, \mu}=0 \: \forall \mu \neq []$), we will say that $r(s_\Sigma) \in \fps{\Rsemiring}{\emptyset}= \Rsemiring$ is the value of $r$ at the point $(s_{\sigma, []})_{\sigma \in \Sigma}$.\\
  For a finite set $\Sigma$ and a natural number $k$, let $!_nX$ be the family of multisets $\mu$ of maximal multeplicity $k$ (i.e $\max_{s \in \Sigma} \mu(i)\leq k)$. This is clearly a finite subset of $!\Sigma$, corresponding to monomials over $x_\Sigma$ having degree at most $n$ in each variable. If a power series $p: !\Sigma \to \Rsemiring$ is such that $ \exists k \in \NAT \: \supp p \subseteq !_k \Sigma$, we say that $p$ is a polynomial. In this case, $p$ can be seen as a finite sum:
  $$p= \sum_{n_1 \dots n_{|\Sigma|} = 0}^k p_{n_1 \dots n_{|\Sigma|}} x_{\sigma_1}^{n_1} \dots x_{\sigma_{|\Sigma|}}^{n_{|\Sigma|}}$$
  Polynomial form a sub- semiring of $\fps{\Rsemiring}{\Sigma}$, that we will denote as $\fps{\Rsemiring}{\Sigma}$. It is obviously not continuous, as the supremum of a collection of polynomials can be an infinite power series. 
  Let us say something on what happens if we do not start with a continuous semiring $\Rsemiring$, but rather with a ring $R$. In this case, almost all of the constructions we made will still work: we will be able to define the set $\fps{R}{\Sigma}$ of formal power series over $R$, which will be a ring and its subring  $\fps{R}{\Sigma}$ of polynomials over $R$. Still, the composition of formal power series will not be defined in general, as it involves an infinite sum.\\
  As we saw, power series are infinitary objects and their manipulation can then involve the concept of limit, making it generally uncomputable. Moreover, the continuous semiring we will generally use in the context of probabilistic semantics will be $\Rinf$ and the semiring of formal power series over $\Rinf$: hence, the computation with the coefficients of power series themselves could be infinitary. To our rescue, we will use a way to give a finitary specification of power series: this will be the concept of algebraicity. First, we recall it in the context of rings:
  \begin{definition}
  	Let $R$ be a ring and let $R'$ be a sub ring of it.
  	With a little abuse of notation, we say that a formal power series $s \in \fps{R}{\Sigma}$ is \textbf{algebraic} over $R'$ if there exists a polynomial $p(w, x_\Sigma) \in \fpp{R'}{\Sigma \cup w}$ such that $p(r, x_\Sigma)=0$.
  \end{definition}
  \begin{example}
  	\begin{itemize}
  		\item Let $s \in \fps{\R}{x}$ be $\sum_{n \geq 0}x^n$. Then if we take the polynomial $p(w,x) = w(1-x)-1$, we get $p(s, x)=0$. Hence, $s$ is algebraic over $\Q$. Indeed, $s$ is the multiplicative inverse $(1-x)^{-1}$ of $(1-x)$ in $\fps{\R}{x}$. In general, every $\Q$ rational power series (i.e every power series of the form $r_1(x_\Sigma)r_2(x_\Sigma)^{-1}$ for $r_1, r_2 \in \fps{\Q}{\Sigma}$) is algebraic over $\Q$ 
  		\item Let $s= \sum_{n \geq 0} \frac{1}{n+1} \binom{2n}{n} x^{n} \in \fps{\R}{x}$. One can verify that this is the Taylor expansion around $0$ of the (analytic) function $\frac{1-\sqrt{1-4z}}{2z}$. From this, it is easy to see that, taking $p(w, x)= zw^2 - w + 1$, that $p(s, x)=0$. Observe that computing $s(1) = sum_{n \geq 0} \frac{1}{n+1} \binom{2n}{n}$ as a limit is hard, but, using the equation $p(s(x), x)=0$, we can easily deduce that $s(1)$ is the (only) positive root of $w^2-w+1$. 
  	\end{itemize}
  \end{example}	
  	