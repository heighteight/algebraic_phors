% !TEX root = main.tex

We introduce a second class of PHORS that extends the $\Gfin$ by admitting infinite grades $!_\infty$, corresponding to parameters that may be used an \emph{unbounded} number of times. In this way, thanks to the composability of the underlying algebraic power series, the $\Ginf$ are proved to be closed under \emph{composition}.


 
 
\subsection{Syntax of $\Ginf$}


While the type discipline of $\Gfin$ ensures that all variables are used in a bounded way,
the variable $z$ in $a_\gphors(z)$, which counts the number of probabilistic choices (e.g.~think again of the HORS \eqref{eq:introphors2}), can be thought of as an input that is used \emph{unboundedly} during reductions, without compromising algebraicity.

The key insight here is that we are describing PHORS via families of power series with coefficients taken in $\fps{\Rinf}{z}$: in other words, the variable $z$ is treated as a \emph{formal parameter} all along the way.
Our goal is then to capture those situations in which a variable is used as a parameter and can, thus, be duplicated unrestrictedly.


\begin{example}\label{ex:phorsLfx}
Consider the following PHORS $\gphors$:
\begin{align*}
Lfgx&= f(Lfg(Lfgx))\oplus_{a}gx\\
Ax&=x\oplus_b \Omega\\
Bx&=x\oplus_c \Omega\\
S&=LAB e
\end{align*}
Define the parenthesis symbols ``$($'' and ``$)$'' as, respectively, $ab$ and $ac$; then the branch language $\C L(\gphors)$ is formed by all words of the form $w)\in \{(,)\}^*$, where $w$ is ``well-parenthesized'', like e.g.~$w=(()(()))$, $w=(()(())())$, i.e.~it essentially corresponds to the \emph{Dyck language}.
Notice that $\gphors$ is not a $\Gfin$: 
if we try to type the term $t_L=\lambda fgx. f(Lfg(Lfgx))\oplus_{a}gx$ with finite grades, setting
$\nonterm(L)=!_{m}\tau\multimap!_n\tau\multimap\tau$, we obtain
$\nonterm\mid \emptyset\vdash t_L: \varphi$, where $\varphi= !_{2m+1}\tau\multimap !_{2n}\tau\multimap \tau\not\sqsubseteq\nonterm(L)$.
The only way out is to assign infinite grades $n,m=\infty$, i.e.~to let 
$\nonterm(L)=!_\infty \tau\multimap !_\infty\tau\multimap\tau$.

The interpretation of $L$ will be some power series $a_L(z,f,g,x)=\sum_{ijk}L_{ijk}f^ig^jxz^k$.
Now, observe that $a_L(z,f,g,x)$ can be rewritten as a power series $a_L(z,f,g)(x)$ with coefficients in $\fps{\Rinf}{z,f,g}$, that is a solution (letting e.g.~$a=\frac{1}{2}$) of the algebraic equation  
\begin{equation}\label{eq:catalan2}
a_L(z,f,g)=\frac{zf a_L(z,f,g)^2+zg}{2}.
\end{equation}
This equation is reminiscent of \eqref{eq:alg1} and can indeed be solved in a similar way, yielding $a_L(z,f,g,x)=\sum_{i=0}^{\infty}\frac{C_i}{2^{2i+1}}f^ig^{i+1}xz^{2i+1}$: there are 
$C_i$ well-parenthesized words of length $2i+1$, each using $($ $i$ times and $)$ $i+1$ times, each of them produced by a reduction making $2i+1$ choices, and thus having probability $\frac{1}{2^{2i+1}}$.
\end{example}

We now introduce a type discipline that captures situations like the one just sketched.
We use two different kinds of types:
\begin{align*}
\varphi,\psi&:= o^n\mid !_k\varphi\multimap \varphi\qquad (k\in \N)\\
\Phi,\Psi&:=\varphi\mid !_k\varphi\multimap \Phi\qquad (k\in \N\cup\{\infty\})
\end{align*}
The types $\varphi,\psi$ are called \emph{finitary} and are precisely those of $\Gfin$; the types $\Phi,\Psi$ are called \emph{infinitary}. The order relation on infinitary type $\Phi\sqsubseteq\Psi$ is inherited from the one for infinite types together with the new rule 
$\varphi\sqsubseteq\varphi',\Psi\sqsubseteq\Psi'\Rightarrow !_k\varphi\multimap \Psi\sqsubseteq !_\infty\varphi'\multimap\Psi'$

The typing rules will use three kinds of contexts:
\begin{varitemize}
\item the non-linear context $\nonterm$, made of infinitary bindings of the form $L:\Phi$;
\item a new parameters context $\C P$, made of finitary bindings of the form $w:_\infty \varphi$;
\item a graded context $\Delta$, made of finitary bindings of the form $x:_k \varphi$, with $k<\infty$.
\end{varitemize}
The rules are illustrated in Fig.~\ref{fig:typerulesinf}.
The rules for finitary types are as in the previous section; the new rules are
ax$_{\C P}$ and, notably, the infinitary abstraction and application rules $\lambda_\infty$ and $@_\infty$. In the rule $@_\infty$, it is assumed that the contexts $\nonterm$ and $\nonterm'$ are \emph{disjoint}. This rule states that an infinitary function $t:!_\infty\varphi\multimap\Psi$ can only be applied to either a \emph{fresh} non-terminal or another parameter. 
Recall that in a PHORS each non-terminal $L$ is equated with a corresponding term $t_L$: so, when we apply a function $t:!_\infty \varphi\to \Psi$ to the symbol $L$, we are actually applying $t$ to the corresponding term $t_L$.
% The validity of this rule relies indeed in the possibility of \emph{composing} algebraic power series.


\begin{figure}
\fbox{
\begin{minipage}{.43\textwidth}
\resizebox{\linewidth}{!}{
\begin{minipage}{\linewidth}
\begin{align*}
		\infer[ax_\Delta]{\nonterm\mid\C P\mid \Delta\vdash x:\varphi }{x:_p\varphi\in \Delta,\quad 1 \leq p} \qquad
		\infer[ax_{\C P}]{\nonterm\mid\C P\mid\Delta \vdash w:\varphi }{ w:_\infty\varphi  \in \C P}
		 \qquad
		\infer[ax_\nonterm]{\nonterm\mid\C P\mid\Delta \vdash L:\Phi }{ L:\Phi  \in \nonterm}
\end{align*}
\begin{align*}
		\infer[\lambda]{\nonterm\mid\C P\mid \Delta \vdash \lambda x. t : !_k \varphi \multimap \Psi }{\nonterm\mid\C P \mid \Delta, x:_k \varphi \vdash t : \Psi}
		&  &
		\infer[@]{\nonterm\mid\C P\mid \Delta + k\Delta' \vdash tu : \Psi}
		{\nonterm\mid\C P\mid\Delta \vdash t :!_k\varphi \multimap \Psi \qquad \nonterm\mid\C P\mid \Delta' \vdash u :\varphi }
\end{align*}
\begin{align*}
		\infer[\lambda_\infty]{\nonterm\mid\C P\mid \emptyset \vdash \lambda w. t : !_\infty \varphi \multimap \Psi }{\nonterm\mid\C P, w:_\infty \varphi \mid \emptyset \vdash t : \Psi}
		&  &
		\infer[@_\infty]{\nonterm\cup\nonterm'\mid\C P\mid \Delta \vdash tu : \Psi}
		{\nonterm\mid\C P\mid\Delta\vdash t :!_\infty\varphi \multimap \Psi \qquad u:\varphi\in \nonterm'\cup\C P}
\end{align*}
\begin{align*}
		\infer[\langle \rangle]{\nonterm\mid\C P\mid \Delta \vdash \langle t_1, \dots,t_n \rangle : \1^n}{\nonterm\mid\C P\mid \Delta \vdash t_i : \1 &  i=1,\dots, n}&  &
		\infer[\pi_i]{\nonterm\mid\C P\mid \Delta \vdash \pi_i t: \1}{\nonterm\mid\C P\mid \Delta \vdash t: \1^n &  i=1,\dots, n}
	\end{align*}
	\begin{align*}
		 \infer[\oplus]{\nonterm\mid\C P\mid \Delta \vdash t \oplus t' : \1 }{\nonterm\mid\C P\mid \Delta \vdash t:\1 \quad \nonterm\mid\C P\mid \Delta \vdash t':\1} 
	\end{align*}
%		\begin{align*}
%			\infer[ Y_{\mathrm{fin}}]{\nonterm\mid\emptyset \vdash^{\mathrm{fix}}  Y \lambda \langle L_1,\dots, L_n\rangle.t:\psi }{\nonterm,\langle L_1,\dots, L_n\rangle:\varphi\mid \emptyset \vdash t :  \psi \quad \psi \sqsubseteq \varphi  }
%	\end{align*}
	\end{minipage}
	}
\end{minipage}
}
\caption{Typing rules of $\Gfin$.}
\label{fig:typerulesinf}
\end{figure}

\begin{definition}[$\Ginf$]
A \emph{(parametric) bounded PHORS} (noted $\Ginf$) is a tuple $\gphors=(\nonterm, \C P,\C D, S)$, where $\nonterm$ is a finite set of (infinitarily) typed non-terminals $L:\Phi$, $\C P$ is a finite set of (finitarily) typed parameters $w:_\infty\varphi$, $S\in\nonterm$ is the {start symbol} such that $\nonterm(S)=\1$, and $\C D$ is a function that associates each $L\in\nonterm$ with a derivation $\nonterm\mid\C P\mid\emptyset\vdash \lambda x_1.\dots. x_n.t:\Phi$ such that $\Phi\sqsubseteq \nonterm(L)$, $t$ contains no $\lambda$-abstraction and $\nonterm\mid x_1,\dots, x_k\vdash t:\1$. 

\end{definition} 

Notice that a $\Ginf$ $\gphors$ may contain \emph{open} parameters $w:_\infty\varphi\in \C P$. We call $ \gphors$ \emph{closed} when $\C P=\emptyset$. 


The PHORS from Example \eqref{ex:phorsLfx} is a closed $\Ginf$, with $\nonterm(L)=!_\infty\tau\multimap!_\infty\tau\multimap\tau$. 
Instead, the non-algebraic PHORS from Example \eqref{ex:nonalg} is not a $\Ginf$:
while we can give type $\nonterm(L)=!_\infty \tau\to \tau$ to $L$, we cannot be apply it to the function $f\circ f:\tau$, as the latter is neither a parameter nor a non-terminal. 




As for the $\Gfin$, we interpret each $\Ginf$ $\gphors=(\nonterm,\C P, \C D,S)$ via its
induced simply typed PHORS $(\st\nonterm,\st{\C P},\st{\C D},S)$.
This yields then a family of fps
${\model{{\gphors}}}^{\Rsemiring}\in \fps{\Rsemiring}{\model{\st\nonterm}+\model{\st{\C P}}}^{\model{\st{\nonterm}}}$, that we will look at as parametric on $\C P$, that is, as 
in $\fps{(\fps{\Rsemiring}{\model{\st{\C P}}})}{\model{\st\nonterm} }^{\model{\st\nonterm}}$.
%
% Now, this family can be equivalently described as a family in $\fps{(\fps{\Rsemiring}{\model S})}{\model T }^{\model T}$, that is, as given by fps with variables in $\model T$ and coefficients in $\fps{\Rsemiring}{\model S}$. 





Two PHORS can be composed through their open parameters:
\begin{definition}[composition of $\Gfin$]
Given two $\Ginf$ $\gphors_1=(\nonterm_1,\C P\cup\{w:_\infty\nonterm(L)),\C D_1,S_1)$ and
$\gphors_2=(\nonterm_2,\C P,\C D_2,S_2)$, where $\nonterm_1\cap\nonterm_2=\emptyset$
and $L\in \nonterm_2$. Their \emph{composition through $L$} is the $\Ginf$ $\gphors_1 \circ_L\gphors_2=(\nonterm_1\cup\nonterm_2, \C P, \C D_1[L/w]\cup\C D_2,S_1)$.
\end{definition}


The composition of two PHORS is interpreted via the composition of the associated formal power series, as shown below:

\begin{lemma}\label{lemma:composition}
$\model{\gphors_1\circ_L\gphors_2}^{\Rsemiring}\left(w_{\model{\C P}}\right)=
\model{\gphors_1}^{\Rsemiring}\left(\model{\gphors_2}^{\Rsemiring}\left(w_{\model{\C P}}\right),w_{\model{\C P}}\right)$.
%
%
%	Let  $\Ginf$ $\gphors_1=(\nonterm_1,\C P\cup\{w:_\infty\varphi),\C D_1,S_1)$ and
%	$\gphors_2=(\nonterm_2,\C P,\C D_2,S_2)$ be two $\Ginf$ with $\nonterm_1 \cap \nonterm_2= \emptyset$ and $\nonterm_2(L) = \phi$. Let $\gphors_1(\gphors_2)$ be their composition along $L$. Then we have for each $L_1 \in \nonterm_1 $,
%	$$\model{\gphors_1(\gphors_2)}_{L_1}^\Rsemiring(y_{\model{\C P}}) = \model{\gphors_1}^\Rsemiring_{\model{L_1}}(\model{\gphors_2}^\Rsemiring_{\model{L}}(y_{\model{\C P}}),y_{\model{\C P}} )$$
%	and for each $L_2 \in \nonterm_2$,
%	$$ \model{\gphors_1(\gphors_2)}_{L_2}^\Rsemiring(y_{\model{\C P}}) = \model{\gphors_2}_{L_2}^\Rsemiring(y_{\model{\C P}}) $$
\end{lemma}

Composability enables a \emph{modular} analysis of the generating functions of a PHORS: we may for instance first focus on certain subsets of non-terminals, replacing all others with parameters, and only later put the pieces together via composition.


\subsection{From $\Ginf$ to FAS}



To show that the $\Ginf$ have algebraic generating functions, we will proceed in two steps.
First, we will consider finitary PHORS with parameters, which can be shown algebraic with the same approach as in Section 6. Then, we will show that any $\Ginf$ can be reduced to an equivalent parametric $\Gfin$, which will result in the algebraicity of its generating function.


\paragraph*{Finitary Parametric PHORS}
Consider a $\gphors=(\nonterm, \C P,\C D,S)$ that is parametric but \emph{finitary}, i.e.~all its non-terminal symbols have a finitary type. In other words, the derivations $\C D(L_i)$ are of the form $\nonterm\mid \C P\mid \emptyset \vdash t_{L_i}:\varphi$, where $\varphi\sqsubseteq \nonterm(L_i)$ and all types in $\nonterm$ and $\C P$ are finitary, but recall that the parameters $w$ in $\C P$ have infinite grade. 

By arguing as in the previous section, but reasoning over the semiring $\fps{\Rsemiring}{\model S}$, we can conclude that the pair of finite variable sets $(\model \nonterm,\model \nonterm)$ is stable and polynomial, yielding:

\begin{proposition}
For all finitary parametric $\Ginf$ $\gphors=(\nonterm, \C P,\C D, S)$, for each $L\in \nonterm$, the fps $a_L(z,w_{\model{\C P}})\in \fps{\Rinf}{z,w_{\model{\C P}}}$ is algebraic. 
\end{proposition}

\begin{example}\label{ex:Lfxpara}
Consider the finitary $\Ginf$, with two parameters $f,g:_\infty\tau$, obtained from the one in Example \ref{ex:phorsLfx} by introducing two new non-terminals
 $L_{[f,g]}:\tau$ and $S_{[f,g]}$ with the equations 
\begin{align*}
L_{[f,g]}x&= f(L_{[f,g]}(L_{[f,g]}x))\oplus_a gx\\
S_{[f,g]}&= L_{[f,g]}e
\end{align*}
The interpretation of $L_{[f,g]}:\tau$ is then the fps $a_{F}(z,f,g)(x)\in \fps{(\fps{\Rinf}{z,f,g})}{x}$ given by Equation \eqref{eq:catalan2}.
\end{example}


\paragraph*{Reducing $\Ginf$ to Parametric $\Gfin$}
Following the idea of Example \ref{ex:Lfxpara}, we show how any closed $\Ginf$ can be turned into a parametric $\Gfin$. 
The idea is to replace each non-terminal $L$ with possibly infinitely graded inputs $w_1,\dots, w_n$ with as many non-terminals $L_{[A_1,\dots,A_n]}$ as all possible ways of \emph{filling in} such inputs with other non-terminals from $\Ginf$. 

Let then $\gphors=(\nonterm,\emptyset, \C D,S)$ be a closed $\Ginf$. 
For any non-terminal $L$ whose type can be written as $\nonterm(L)=!_\infty \phi_1\multimap\dots\multimap !_\infty\phi_k\multimap \phi_L$,
let $\C P_L$ be the parameter context formed by all $w_i:_\infty\phi_i$, and let $T(L)$  be the sets of \emph{sostituable terms}, defined as:	
\[
T(L) \bydef \{[t_1, \dots t_n] \in (\nonterm\cup\C P_L)^* \: \mid \: \forall i \, (\nonterm \cup \mathcal{P}) (t_i) = \phi_i)  \} 
\]
Let $\widetilde{\nonterm}$ be a new set of finitely typed non-terminals containing, for each $L\in \nonterm$ as above and $\gamma\in T(L)$, a non-terminal $\tilde{L}_\gamma:\phi_L$.
Finally, let $\C P$ be the concatenation of all $\C P_L$.

For each $L\in \nonterm$ and $\gamma\in T(L)$, define the following terms:
\begin{varitemize}
\item an applicative term $\nonterm\mid\C P\mid\emptyset \vdash \alpha_{L,\gamma}:\phi_L$ defined by 
$\alpha_{L,\gamma}:=L\gamma$;

\item a finitary term $\widetilde{\nonterm}\mid\C P \mid\emptyset\vdash \tilde{t}_{L,\gamma}:= \phi_L$ defined by 
$\tilde t_{L,\gamma}=( t_L\gamma)\big[ \tilde G_{\delta}/G\delta\big]_{G\in \nonterm, \delta\in T(G)}$.



\end{varitemize}

We have obtained in this way a finitary parametric PHORS $\gphors_{\mathrm{fin}}=(\widetilde\nonterm, \C P, \C D',S_{[]})$, and we have the equalities:
\begin{equation}\label{eq:alphaeq}
\alpha_{L,\gamma}\big[ t_L/  L] = t_L\gamma   =
\tilde{t}_{L,\gamma}\big[\alpha_{G,\delta}/\tilde{G}_\delta\big]_{G\in \nonterm,\delta\in T(G)}
\end{equation}



The following technical lemma relates then the minimal solutions of the systems of $\gphors$ and $\gphors_{\mathrm{fin}}$.

\begin{lemma}\label{lemma:fixpoint}
Given sets $\nonterm,\widetilde{\nonterm},\C P$ and morphisms
$t\in \Qrelkleisli{\Rsemiring}(\nonterm, \nonterm)$,
$\tilde t\in \Qrelkleisli{\Rsemiring}(\widetilde{\nonterm}+\C P, \widetilde\nonterm)$,
$\alpha\in \Qrelkleisli{\Rsemiring}(\nonterm+\C P, \widetilde{\nonterm})$,
if we have:
\begin{varitemize}
\item $\alpha\circ \langle 0,1_\C P \rangle=0\in \Qrelkleisli{\Rsemiring}(\C P, \widetilde{\nonterm})$, 
\item $\alpha\circ \langle t\circ \pi_1,\pi_2  \rangle=
\tilde t\circ \langle \alpha\circ \langle \pi_1,\pi_2 \rangle,\pi_2\rangle\in \Qrelkleisli{\Rsemiring}(\nonterm+\C P, \widetilde{\nonterm})
$,
\end{varitemize} 
then
$\alpha\circ \langle {\mathsf{fix}}_{0,\nonterm} t,1_{\C P}\rangle={\mathsf{fix}}_{\C P,\widetilde{\nonterm}} \tilde t\in \Qrelkleisli{\Rsemiring}(\C P, \widetilde{\nonterm})
$. 

\end{lemma}

Notice that the first condition of Lemma \ref{lemma:fixpoint}
clearly holds for the interpretation of $\alpha=\langle \model{\alpha_{L,\gamma}}^{\Rsemiring}\rangle_{L,\gamma}$, and that the second condition is precisely \eqref{eq:alphaeq}. 
%In the above, given $s\in \Qrelkleisli{\Rsemiring}(0, !X\times Y)$ and 
% $t\in \Qrelkleisli{\Rsemiring}(Z, X)$, we write $s\cdot t\in  \Qrelkleisli{\Rsemiring}(Z,  Y)$ for the evaluation of $s$ on $t$.


\begin{theorem}
For any closed $\Ginf$ $\gphors$ there exists a parametric $\Gfin$ $\gphors_{\mathrm{fin}}$, $\|\gphors_{\mathrm{fin}}\|\leq \|\gphors\|^{|\nonterm|}$, such that $a_{\gphors}(z)=a_{\gphors_{\mathrm{fin}}}(z,1)$ and $\C L(\gphors)=\C L(\gphors_{\mathrm{fin}})$.
\end{theorem}
\begin{proof}[Proof sketch]
By inspecting the connected component formed by all non-terminals $\tilde L_{\gamma}$
``reachable'' from the source $S_{[]}$, one can check that none of these contains free parameters in $\gamma$. For the closed finitary PHORS $\tilde{t}^*$ obtained by restricting to such non-terminals, Lemma \ref{lemma:fixpoint} yields then the equality 
$\mathsf{fix}_{\nonterm} \model{t_\gphors}=\mathsf{fix}_{\widetilde\nonterm}\model{\tilde t^*}$.
\end{proof}



%( be transformed into an equivalent order-1 $\Gfin$ defined by replacing $L$ by an order-1 $L_{[A,B]}$ non-terminal with equation
%$L_{[A,B]}x=A(L_{[A,B]}(L_{[A,B]}x))\oplus_a Bx$. More generally, any closed $\Ginf$ can be turned into an equivalent $\Gfin$:
%)



We thus immediately obtain:
\begin{corollary}
For all closed $\Ginf=(\nonterm, \emptyset, \C D,S)$, and non-terminal $L\in \nonterm$, the  generating function $a_L(z)$ is algebraic. Moreover, the AST and PAST problems are decidable.
\end{corollary}
\begin{example}
If we apply this technique to the $\Ginf$ from Example \ref{ex:phorsLfx},
we must introduce all order-1 non-terminals $L_{[\gamma_1,\gamma_2]}$ with equations
$L_{[\gamma_1,\gamma_2]}x=\gamma_1(L_{[\gamma_1,\gamma_2]}(L_{[\gamma_1,\gamma_2]}x)\oplus_a \gamma_2x$, where $\gamma_1,\gamma_2$ are either non-terminals or parameters. With $\gamma_1,\gamma_2=f,g$, we precisely get the equation in Example \ref{ex:Lfxpara}. By restricting ourselves to the connected component of $S_{[]}$ we obtain then a \emph{closed} order-1 $\Gfin$ with non-terminals $S_{[]},A,B,L_{[A,B]}$.
\end{example}
%\begin{example}
%Let $\sigma$ be a generator of the permutation group $\mathfrak S(n)$. Take a $\Ginf$
%with equation $Lw_1\dots w_nx=w_1(Lw_{\sigma(1)}\dots w_{\sigma(n)})\oplus x$, where $L:(o\to o)^n\to(o\to o)$, as well as non-terminals $A_1,\dots, A_n$ and $S=LA_1\dots A_ne$. Intuitively, this PHORS computes all words of the form $w=a_1a_{\sigma(1)}\dots a_{\sigma^{|w|}(1)}$. Then the corresponding $\Gfin$ must contain, beyond $S$ and the $A_i$, $n!$ non-terminals $L_{[A_{\tau(1)},\dots, A_{\tau(n)}]}$, with $\tau\in \mathfrak S(n)$.
%
%\end{example}


%
%The interpretation of $\Ginf$ in the relational model works as for the $\Gfin$.
%The refinement relation is extended via $\varphi\tri T, \Psi\tri U \Rightarrow !_\infty\varphi\multimap \Psi\tri T\to U$. 
%The interpretation is extended by $\model{!_\infty\varphi\multimap\Psi}:=!\model{\varphi}\times\model{\Psi}$.
%
%
%For any derivation of $\nonterm\mid\C P\mid\Delta\vdash t:\Phi$, 
%$\Rsemiring{\model{\C P}}$ will be the semiring of power series in the \emph{finitely many} variables $\vec w$ arising from the interpretation of the finite types in $\C P$. 
%Given $\nonterm\tri\Gamma_\nonterm, \Delta\tri\Gamma_\Delta$ and $\Phi\tri T$, we obtain the 
%\[
%\model{t:\Phi\tri T}^{\Rsemiring}\in \Qrelkleisli{(\fps{\Rsemiring}{\model{\C P}})}(\model{\Gamma_\nonterm}+\model{\Gamma_\Delta},{\model{\Phi}}).
%\]
%As before, $\model{t:\Phi\tri T}^{\Rsemiring}$ coincides with the restriction of the corresponding $\LY$-derivation $\model{t:T}^{\Rsemiring}$. More precisely, 
%we have 
%\[
%\alpha_{\C P,\model{\nonterm},\model{\Delta},\model{\Phi}}\circ \model{t:\Phi\tri T}^{\Rsemiring}=\left(\model{t:T}^{\Rsemiring}\right)\Big \vert^{\model{\Phi}}_{\model{\nonterm}+\model{\C P}+\model{\Delta}},
%\]
%where 
%$\alpha_{Z,X,Y,W}$ is the isomorphism
%$\fps{(\fps{\Rsemiring}{Z})}{X+Y}^W\equiv\fps{\Rsemiring}{X+Z+Y}^W$.
%As before, all this ensures that, for any closed $\Ginf$ $\gphors=(\nonterm, \emptyset,\C D, S)$, the generating function $a_\gphors(z):=\model{S:o\tri o}^{\fps{\Rinf}{z}}=\sum_{i=0}^{\infty}\mathbb P(\gphors\downarrow_i)z^i$ captures probabilistic termination. 
%
%Similarly to the previous section, although now in a parametric sense, we obtain
%\begin{proposition}
%For every $\Ginf$ $\gphors=(\nonterm, \C P, \C D,S)$, the fps $
%\model{t_\gphors}^{\Rsemiring}\in\fps{(\fps{\Rsemiring}{\model{\C P}})}{\model{\C N}}^{\model{\C N}}$ is finitary.
%\end{proposition}
%\begin{theorem}[$\Ginf$ are algebraic ]
%For all closed $\Ginf$ $\gphors=(\nonterm, \C P, \C D,S)$, the 
%$\mathbb P(\gphors\downarrow)$, $\mathbb E(\gphors\downarrow)$ are $\mathbb Q^+$-algebraic, $\sum_i\mathbb P(\gphors \downarrow_i)^i$ is $\fps{\mathbb Q^+}{z}$-algebraic.
%\end{theorem}
%
%Also in this case, we can deduce decidability of AST and PAST from algebraicity.
%\begin{corollary}
%For all closed $\Ginf$, the AST and PAST problems are decidable.
%\end{corollary}
%\begin{proof}[Proof sketch]
%
%
%\end{proof}
%
%
%The system  $\mathbf{G}_{\mathrm{fin}}$ only types PHORS of finitary graded types. For sure, the class of PHORS whose non-terminals are interpreted in Qrel by an algebraic power series is much wider than this. In this paragraph we will show how this technique and the correlated results about the decidability of AST/PAST can be extended to a system $\mathbf{G}_{\mathrm{\infty}}$ that allows infinitary types. To understand how this work, it is important to think about the following example.
%\begin{example}
%	Let us consider an order 2 non-terminal $L$ of simple type $(\1 \to \1) \to \1 \to \1$, defined by the equation $Lfk = fk \oplus Lf(Lfk)$. Its branch language will be the set of the trajectories of a random walk on $\N^+$ starting from 1 and ending in 0 (just like the order 1 PHORS $Fx=x+FFx$). The smallest (in the sense of $\sqsubseteq$) type that we can give to it is $!_\infty(!_1 \1 \multimap \1) \multimap !_1\1 \multimap \1$. If we call $f_0, f_1 $ (resp. $x$) the variables interpreting $f$ (resp. $k$) in Qrel, one can see that the interpretation of $L$ will then be a power series $\sum_{n \geq 0} L_n f_1^nk$: all other coefficients will be 0 as $L$ will always terminate by using at least once $f$ applied to $k$. Now we could see this infinite power series as a polynomial $p = \tilde{L}(f)k\in \fpp{\fps{\R}{f}}{k}$. This allows us to apply the ideas of paragraphs IV and V and write the equation for coefficients (which now are power series): $\tilde{L}(f) = f/2 + \tilde{L}^2(f)$ From this we see that $\tilde{L}(f)$ (and thus $\sum_{n \geq 0} L_n f_1^nk$) is algebraic.\\
%	To understand better what is happening here, observe that our $L$ can be seen as a parametric version of the order 1 random walk: we can think of considering the 'open' PHORS $Fx = f x \oplus F F x$ where $f$ is a variable of type $!_1 1 \multimap 1$; for each value we substitute to $f$, we then obtain a value for the non terminal $F$: we can then think of $F$ as depending over $f$, thus transforming $F: !_1 1 \to \1 $ into  $L: !_1 1 \to \1 $ 
%\end{example}
%The discussion of the last example gives us an outline of our strategy to treat infinitary PHORS: first, we need to introduce PHORS on an open parameter; secondly we need an introduction rule to transform a non-terminals $L_1: \tau$, whose definition dependins over a parameter of finite type $f: \phi$ into a non terminal $\lambda L(f): !_k$. 
%\subsection{Parametric PHORS}
%We now introduce the notion of a \emph{parametric finitely graded PHORS} (noted $\Gfin_{Par}$) 
%\begin{definition}
% A \emph{parametric finitely graded PHORS} is a triple $\gphors=(\nonterm, \C D, P, S)$, where $\nonterm$ is a finite set of typed non-terminals, $S\in\nonterm$ is the {start symbol} such that $\nonterm(S)=\1$, $P$ is a context $ f_1:_{k_1} \psi_1, \dots f_n:_{k_n}: \psi_n$ ($f_1, \dots f_n$ are called the paramters of the PHORS), and $\C D$ is a function that associates each $L\in\nonterm$ with a derivation $\nonterm\mid P : \psi_n \vdash \lambda x_1.\dots. x_n.t:\varphi$ such that $\varphi\sqsubseteq \nonterm(L)$, $t$ contains no $\lambda$-abstraction and $\nonterm\mid P , x_1,\dots, x_k\vdash t:\1$. 
%\end{definition}  
%In the case of a parametric PHORS, the interpretation of the $\lambda Y$ term $t_\gphors$ is a power series in $\fps{\Rsemiring}{\model{\nonterm} + \model{P}}^{\model{\nonterm}}$. All the reasonements of paragraph 6.2 continue to hold and we obtain an analogue of proposition 6.4:
%\begin{proposition}
%	For every $\Gfin_{Par}$ $\gphors=(\nonterm,\C, P R,S)$, the fps
%	$\model{t_{\gphors}}^{\Rsemiring}\in \fps{\Rsemiring}{\model{\nonterm}}^{\model{\nonterm} + \model{P}}$ is finitary.
%\end{proposition} 
%From this and  Proposition \ref{prop:fintoalg}, we obtain then:
%\begin{theorem}[$\Gfin_{Par}$ are algebraic]
%	For all $\Gfin$ $\gphors=(\nonterm, \C D, P, S)$, $\model{S}^{\Rinf}\in \Rinf$ is $\fps{\mathbb Q^{+}}{\model{P}}$-algebraic and 
%	$\model{S}^{\fps{\Rinf}{z}}\in \fps{\Rinf}{z}$ is $\fps{\mathbb Q^{+}}{\model{P}, z}$-algebraic.
%\end{theorem}
%\begin{example}
%	Take the $\gphors$ with non terminals $S: \1, F: !_1 \to 1$ and free parameter $f: !_1 \to \1$:
%	\begin{equation*}
%	\begin{cases}
%		S = F e\\
%		F k = f k \oplus F F k
%	\end{cases}
%	\end{equation*}
%	We see that the in this case $\model{f: !_1 \1 \to \1} = (f_0, f_1) \in (\Rinf)^2 $,   $\model{ fk : !_1 \1 \to \1} = f_1 k \in (\Rinf)^2 $ and  $\lambda \langle F,S\rangle. t_F:((\1\to \1)\times 1)\to (\1\to \1)$ is interpreted as the family of fps:
%	\[
%	s^F_i(y_{\N+1}, f_0, f_1)=
%	\begin{cases}
%		\frac{1}{2} (y_0 + y_0 y_1 + f_0)  
%		& \text{ if }i=1,
%		\\
%		\frac{1}{2}(
%		y_1^2+f_1)
%		& \text{ if }i=1,\\
%		\frac{1}{2}s_i(y_{\N})
%		&\text{ if }i\in\N, i\neq 0, 1
%		%\\
%		%\sum_{n}y_n&\text{ if }i=\star.
%	\end{cases}
%	\]
%	while $\model{\lambda \langle F,S\rangle. t_S:((\1\to \1)\times 1)\to \1}$ is a unique fps 
%	$s^S(y_{\N +1}  f_0, f_1)\in \fps{\Rinf}{\N +1, f_0, f_1}$ given by $s^S(y_{\N +1, f_0, f_1})=\sum_{i\in \N} y_i$. The minimum fixpoint solution is then $s_1(y_\N, f_0, f_1)= 1- \sqrt{1-f_1^2}$, $s_i= f_0/\sqrt{1- f^2_1}$
%\end{example}
%\subsection{Abstraction of a parameter}
%Given a parametric finitely graded PHORS $\gphors=(\nonterm, \C D, P, S)$, the interpretation of its non terminals $\model{L_i}^{\Rsemiring}$ is an algebraic power series in the variables $\model{P}$. It is then quite natural -at least semantically - to see $L_i$ as of type $!_\infty \psi_1 \multimap !_\infty \psi_ 2\dots !_\infty \psi_n \multimap \nonterm(L_i)$ This motivates the introduction of a new rule that allows this kind of transformation. First of all, given a type  $\phi$ and a context $\Delta= f_1:_{k_1} \psi_1, \dots f_1:_{k_n} \psi_n $ we define $I_\Delta(\phi)$ to be the type  $!_\infty \psi_1 \multimap !_\infty \psi_ 2\dots !_\infty \psi_n \multimap \nonterm(L_i) $. 
%Then we introduce a rule:
%\[
%\infer[\lambda_\infty]{\nonterm, L'_i: I_\Delta(\phi) \mid \infty \Delta \vdash \lambda f_1 \dots f_n  t(L_i /L'_i f_1, \dots f_n): I_\Delta(\phi') }{\nonterm, L_i: \phi \mid \emptyset \vdash t: \phi' \quad \phi' \sqsubseteq \phi }
%\]
%To understand better how it works, let us focus at first on the case of a single non terminal $L \bydef L_i$. Here, the fixpoint of the premise is a familiy of power series $l \in (\fpp{\Rsemiring}{\model{N},P})^{\model{L}}$ solving the system of equations (over $\fpp{\Rsemiring}{P})$):
%$$(l_i = \model{t}(l_{\model{L}}, x_{\model{P}}))_{i \in \model{L}}$$
%Since this is an equation between fps, we can write it componentwise: in particular, we can write $l_i$ as $\sum_{\mu \in !P}l_{i,\mu} x_P^\mu $. Then let  $\model{t}'(x_{\model L, ! P}, x_P) \bydef \model{t}(\sum_{\mu \in !P}x_{i,\mu} x_P^\mu, x_{\model{P}})$. Then, the coefficients $l_{i_\mu} \in \Rsemiring$ are solution to:
%$$(l_{i,\mu} = \model{t}'(l_{\model{L}, !\model{P}}, x_{\model{P}}))_{(i,\mu) \in \model{L} \times ! \model{P}}$$
%but this is exactly the interpretation of the fixpoint in the consequence. We then get that the fixpoint of the consequence is equal to the fixpoint of the premise (up to an abstraction).  Moreover we see that if the consequence is such that $I_\Delta(\phi') \sqsubseteq \phi$, the premise is such that $\phi' \sqsubseteq \phi$:it is then a paramteric finitely graded $PHORS$, hence $\model{L}^{\Rinf}$ is an algebraic function and so is $\model{L}^{\Rinf}$. 
%In the case of multiple non terminals, we can apply the previous reasonement to $Yt_{\gphors}$
%\subsection{Infinitary application}
%{\color{red}
%
%- discuss order 2 random walk: algebraic equations with power series coefficients correspond to admitting variables with infinite grade, but in a controlled way
%
%- parametric polynomial equations and their solution
%
%- Proposition 2. $n$ parametric polynomial equations in $n$ fps variables yield an algebraic fps
%
%- extended type system
%
%- Theorems 1/2/3 for the new type system, via Proposition 2
%
%}
%


