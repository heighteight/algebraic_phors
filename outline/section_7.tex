\subsection{Related works}

The connection between HORS and the relational semantics of linear logic appears at first in P.A. Melliès and C. Grellois' work (cite finitary semantics of linear logic and higher-order model-checking, indexed linear logic and
higher-order model checking, Relational semantics of
linear logic and higher-order model checking). Subsequently, in (cite
Linearity in Higher-Order Recursion Schemes), a linear-non linear typing system for HORS is introduced, allowing for a finer complexity analysis of model checking and the behaviour of the purely affine fragment of it is studied in []. None of these work involve probabilistic behaviour.\\
In (cite Ugo , Charles, Koabayashi), the problem of AST for PHORS is studied by directly translating a PHORS into (a possibly higher order) real functional equations: only in the case of order 1 PHORS, this produces a fixpoint algebraic system analogous the the ones considered in this work, thus implying a decidability result (while in the same work, undecidability from order 2 is proved). Also, techniques to approximate from above the probability of termination are considered. In (cite Ong), the decidability of AST/PAST for affine PHORS is proved . With respect to this work, they perform the translation from PHORS to real equation via an in intermediate automata-theoretic step, based on the equivalence between affine HORS and restricted pushdown automata (illustrated in (cite Pierre Clairembaut)).\\
On the side of formal language theory, there exists a huge amount of work about generating functions; particularly relevant to this work is the paper (cite FROM INDEXED GRAMMARS TO GENERATING
FUNCTIONS), in which they study generating functions of Aho's indexed languages (corresponding to order 2 safe HORS, thus going beyond the usual framework of context-freeness) and they give an explicit procedure to compute these functions under some restrictions, but do not give a characterization of grammars with an algebraic generating function.\\
Finally, it is worthwhile to notice that in analytic combinatorics a lot of work has been done about different kinds of combinatorial structures having an algebraic generating function, in particular with respect to their asymptotic behaviour (cite Flajolet, Branderier)

{\color{red}


Here some perspectives, suggesting that our result is just the first and most obvious one.

Extensions, e.g.~holonomic power series and restriction, e.g.~rational ones

Asymptotic estimations: a huge literature, can we use them to provide probability estimations?

Extracting properties of programs from the corresponding generating functions: factorization, recurrence properties, differential equations


}
